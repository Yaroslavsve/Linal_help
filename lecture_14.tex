\documentclass[a4paper, 12pt]{article}

\usepackage{import}

% Корректность отображения всех шрифтов, кодировок и мат. символов
\usepackage[T2A]{fontenc}
\usepackage[utf8]{inputenc}
\usepackage[english, russian]{babel}
\usepackage{amssymb, amsmath, amsthm, mathtools}

% Отображение содержания
\usepackage{tocloft}

% Вставка картинок
\usepackage{graphicx}
\usepackage{tikz}
\usepackage{tkz-euclide}
\usepackage{asymptote}

\usepackage{wrapfig}        % Огибание картинок текстом
\usepackage{cancel}         % Зачёркивания
\usepackage{indentfirst}    % Отступ у первого абзаца
\usepackage{xcolor}         % Цвета
\setlength{\parskip}{.5ex}  % Отступы между абзацами
\usepackage{enumitem}       % Работа со списками
% \usepackage{minted}       % Вставка блоков кода

\usepackage{hyperref}       % гиперссылки
\definecolor{linkcolor}{HTML}{225ae2} % Цвет ссылок
\definecolor{urlcolor}{HTML}{225ae2} % Цвет гиперссылок
\hypersetup{
    pdfstartview=FitH, 
    linkcolor=linkcolor,
    urlcolor=urlcolor,
    colorlinks=true}
\setlength{\arrayrulewidth}{0.5mm} %Толщина линейки в таблицах
\setlength{\tabcolsep}{18pt} %Разделение между столбцами в таблице

% Отступы на странице
\usepackage[left=2cm, right=1.5cm, top=2cm, bottom=2cm]{geometry}

\usepackage{cmap}            % Русский поиск в PDF документе
\usepackage{etoolbox}
\usepackage{soul}            % Разряженный текст \so{} и подчеркивание \ul{}
\usepackage{soulutf8}        % Поддержка UTF8 в soul

\usepackage{titlesec}        % Форматирование заголовков
\titleformat{\section}{\LARGE \bfseries}{\thesection}{1em}{}
\titleformat{\subsection}{\Large\bfseries}{\thesubsection}{1em}{}
\titleformat{\subsubsection}{\large\bfseries}{\thesubsubsection}{1em}{}

\newcommand{\R}{\mathbb R}
\newcommand{\Q}{\mathbb Q}
\newcommand{\Z}{\mathbb Z}
\newcommand{\N}{\mathbb N}
\newcommand{\CC}{\mathbb C}
\newcommand{\F}{\mathbb F}
\newcommand{\aug}{\fboxsep=-\fboxrule\!\!\!\fbox{\strut}\!\!\!}
\newcommand{\sgn}{\operatorname{sgn}}
\newcommand{\id}{\mathrm{id}}
\renewcommand{\phi}{\varphi}
\renewcommand{\epsilon}{\varepsilon}

\newsavebox{\boxedalignbox}
\newenvironment{boxedalign*}
  {\begin{equation*}\begin{lrbox}{\boxedalignbox}$\begin{aligned}}
  {\end{aligned}$\end{lrbox}\fbox{\usebox{\boxedalignbox}}\end{equation*}}

\newcommand\tab[1][.5cm]{\hspace*{#1}}

% Подписи для матриц
\newcommand\undermat[2]{\makebox[0pt][l]{$\smash{\underbrace
{\phantom{\begin{matrix}#2\end{matrix}}}_{\text{$#1$}}}$}#2}
\newcommand\overmat[2]{\makebox[0pt][l]{$\smash{\overbrace
{\phantom{\begin{matrix}#2\end{matrix}}}^{\text{$#1$}}}$}#2}

% Значек "пусть"
\newlength{\tempheight}  
\newcommand{\Let}[0]{  
\mathbin{\text{\settoheight{\tempheight}{\mathstrut}\raisebox{0.5\pgflinewidth}{%
\tikz[baseline,line cap=round,line join=round] \draw (0,0) --++ (0.4em,0) --++ (0,1.5ex) --++ (-0.4em,0);
}}}}


% \newcounter{lemcount}
% \newcounter{thcount}
% \newcounter{offercount}
% \newcounter{concount}
% \newcounter{subthcount}
% \newcounter{defcount}

\theoremstyle{definition}
\newtheorem*{definition}{Определение}
% \newtheorem{definitionnum}[defcount]{Определение}
\newtheorem*{example}{Примеры}
\newtheorem*{example1}{Пример}
\newtheorem*{exercise}{Упражнение}


\theoremstyle{plain}
\newtheorem*{theorem}{Теорема}
% \newtheorem{theoremnum}[thcount]{Теорема}
\newtheorem*{consequense}{Следствие}
\newtheorem*{consequenses}{Следствия}
% \newtheorem{consequensenum}[concount]{Следствие}
\newtheorem*{lemma}{Лемма}
% \newtheorem{lemmanum}[lemcount]{Лемма}
\newtheorem*{subtheorem}{Утверждение}
% \newtheorem{subtheoremnum}[subthcount]{Утверждение}
\newtheorem*{algorithm}{Алгоритм}
\newtheorem*{properties}{Свойства}
\newtheorem*{properties1}{Свойство}


\theoremstyle{remark}
\newtheorem*{remark}{Замечание}
\newtheorem*{offer}{Предложение}
% \newtheorem{offernum}[offercount]{Предложение}
\begin{document}
  \newpage
  \begin{definition}
    Билинейная функция называется симметрической, если $\forall x$, $y\in V$ : $b(x, y)=b(y,x)$.
  \end{definition}
  \begin{definition}
    Билинейная функция называется кососимметрической (при $char\F\neq2$), если $\forall x$, $y\in V$ : $b(x,y)= -b(y,x)$.
  \end{definition}
  \begin{subtheorem}(1)
    Любая билинейная функция над полем $\F$ : $char\F\neq2$, единственным образом представляется в виде $b(x,y) = b_+(x,y)+b_-(x,y)$, где $b_+(x,y)$ - симметрическая функция, а $b_-(x,y)$ - кососимметрическая функция. 
  \end{subtheorem}
  \begin{proof}
    Если есть равенство $\begin{cases}
      b(x,y) = b_+(x,y)+b_-(x,y)\\
      b(y,x) = b_+(x,y)-b_-(x,y)
    \end{cases}$ $\Longrightarrow$\\
    \begin{center}
      $b_+(x,y) = \frac{b(x,y)+b(y,x)}{2}$, $b_-(x,y) = \frac{b(x,y)-b(y,x)}{2}$.
    \end{center}
  \end{proof}
  \begin{subtheorem}
    Билинейная функция $b(x,y)$ симметрична (кососимметрична)$\Longleftrightarrow$ в любом базисе $e$ : $B_e^T=B_e$ ($B_e^T=-B_e$).
  \end{subtheorem}
  \begin{proof} (Докажем для симметрической, для кососимметрической аналогично)
    $\underline{\Longrightarrow}$ Пусть $B = (b_{ij})$, тогда $b_{ij}=b(e_i, e_j)$. Если $\forall x$, $y\in V$, $b(x,y)= b(y,x)$, то $b(e_j, e_i) = b(e_i, e_j)$.\\
    $\underline{\Longleftarrow}$ $b(x,y)= X^TBY$, $b(y,x) = Y^TBX = (X^TB^TY)^T = (X^TBY)^T = b(x,y)$.
  \end{proof}
  Утверждение (1) $\Longleftrightarrow$ $\forall$ матрицы $B$ некоторой билинейной функции верно, что $B = B_++B_-$, где $B_+$ - матрица симметрической билинейной функции, а $B_-$ - матрица кососимметрической билинейной функции.
  \begin{definition}
    Квадратичная функция, порождённая билинейной функцией $b(x,y)$ - это функция на $V$, обозначаемая $k(x):=b(x,x)$, если $k(x)\not\equiv0$.
  \end{definition}
  Если $b$ - кососимметрическая функция, то $b(x,x)=0$ $\Longrightarrow$ $k(x)\equiv0$. В общем случае существует бесконечно много билинейных функций, порождающих одну и ту же квадратичную, таких, что если $b(x,y)=b_+(x,y)+b_-(x,y)$, то $b(x,x)=b_+(x,x)$.
  \begin{theorem}
    $\forall$ квадратичной функции $\exists!$ симметрическая билинейная функция, которая её порождает.
  \end{theorem}
  \begin{proof}
    Допустим, что $b(x,y) = b(y,x)$ - симметрическая билинейная функция и $k(x) = b(x,x)$. Тогда $\forall x, y\in V$\\ 
    $k(x+y) = b(x+y, x+y) = b(x,x)+b(x,y)+b(y,x)+b(y,y)=\\ = b(x,x)+2b(x,y)+b(y,y) = k(x)+2b(x,y)+k(y)$.
    \\ Так как $char\F\neq2$, то\\ $b(x,y)=\frac{k(x+y)-k(x)-k(y)}{2}$.
  \end{proof}
  \begin{definition}
    Билинейная функция $b(x,y) = \frac{k(x+y)-k(x)-k(y)}{2}$ называется поляризацией квадратичной функции $k$.
  \end{definition}
  Далее будем считать матрицу квадратичной формы матрицей её полярной симметрической билинейной функции $b(x,y)$.\\
  $b(x,y)=\sum\limits_{i=1}^nb_{ii}x_iy_i+\sum\limits_{i<j}b_{ij}x_iy_j+\sum\limits_{i>j}b_{ij}x_iy_j$, $\forall i$, $j$ $b_{ij}=b_{ji}$ $\Longrightarrow$\\
  $b(x,x)=k(x)=\sum\limits_{i=1}^nb_{ii}x_i^2+\sum\limits_{1\leqslant i<j\leqslant n}b_{ij}x_ix_j$.  (1)
  \begin{example1}
    Пусть $k(x_1,x_2, x_3)=3x_1^2+2x_1x_2-x_1x_3+x_2^2+6x_2x_3-7x_3^2$, тогда\\
    \begin{center}  
      $B=\begin{pmatrix}
        3 & 1 & -\frac{1}{2}\\
        1 & 1 &  3\\
        -\frac{1}{2} &  3 & -7
      \end{pmatrix}$
    \end{center}
  \end{example1}
  \begin{definition}
    Пусть $b(x,y)$ - симметрическая или кососимметрическая билинейная функция и $\emptyset\neq L\leqslant V$. Ортогональным дополнением к $L$ относительно билинейной формы $b(x,y)$ называется $L^{\perp}:=$\{$y\in V$ | $b(x,y)=0$, $\forall x\in L\}$.
  \end{definition}
  \begin{remark}
    Запись $x\perp y$ означает, что $b(x,y)=0$.
  \end{remark}
  \begin{definition}
    $V^{\perp}=\{y\in V$ | $b(x,y)=0$, $\forall x\in V\}$ - ядро формы.
  \end{definition}
  \begin{definition}
    Билинейная функция $b(x,y)$ называется невырожденной, если $Kerb=V^{\perp}=\{0\}$.
  \end{definition}
  \begin{exercise}
    $b(x,y)$ - невырожденная функция $\Longleftrightarrow$ $\det B\neq0$.
  \end{exercise}
  \subsection{Квадратичные формы}
  \begin{definition}
    Квадратичная форма в некотором базисе называется диагональной, если в этом базисе $k(x_1, \ldots, x_n)=\sum\limits_{i=1}^n\alpha_ix_i^2$, где $\alpha_i\in\F$.
  \end{definition}
  \begin{theorem}
    В конечномерном пространстве $V$ ($char\F\neq2$) $\exists$ базис, в котором эта форма диагональна.
  \end{theorem}
  \begin{proof} (Алгоритм Лагранжа (метод выделения полных квадратов))
    По формуле (1) $k(x)=\sum\limits_{i=1}^nb_{ii}x_i^2+\sum\limits_{i<j}b_{ij}x_ix_j$.\\
    1. Основной случай: $\exists i$ : $b_{ii}\neq0$ $\Longrightarrow$ можно перенумеровать неизвестные $x_1$, $\ldots$, $x_n$, так что $b_{11}\neq0$. Выделим в $k(x)$ все одночлены, содержащие $x_1$\\
    $k(x)=\sum\limits_{i=1}^nb_{11}x_1^2+2x_1\sum\limits_{i=2}^nb_{1i}x_i+\widetilde{k}(x_2,\ldots, x_n)$ и дополним выражение до квадрата $\Longrightarrow$\\
    \begin{center}
      $k(x) = b_{11}(x_1^2+2x_1\sum\limits_{i=2}^n\frac{b_{1i}}{b_{11}}x_i+(\sum\limits_{i=2}^n\frac{b_{1i}}{b_{11}}x_i^2))-\frac{(\sum\limits_{i=2}^nb_{1i}x_i)^2}{b_{11}}+\widetilde{k} = $\\
      $=b_{11}(x_1+\sum\limits_{i=2}^n\frac{b_{1i}}{b_{11}}x_i)^2+k_2(x_2, \ldots,x_n)$.
    \end{center}
    Затем для формы $k_2(x_2,\ldots, x_n)=\sum\limits_{i=2}^nb_{ii}'x_i^2+\sum\limits_{2\leqslant i<j\leqslant n}b_{ij}'x_ix_j$ найдём коэффициент $b_{jj}'\neq0$ и выделим квадрат как на предыдущем шаге. На каждом шаге число переменных уменьшается на единицу, а значит, за конечное число шагов (а именно $\leqslant n-2$) форма приобретёт диагональный вид.\\
    2. Особый случай: $\forall i$ $b_{ii}=0$, но так как $k(x)\not\equiv0$ $\Longrightarrow$ $\exists$ индексы $i$ и $j$ такие, что $b_{ij}\neq0$, то есть в выражение $k(x_i, x_j)$ входит одночлен $2b_{ij}x_ix_j$.\\
    Пусть $x_i=x_i'+x_j'$ и $x_j=x_i'-x_j'$, тогда $x_ix_j = x_i'^2-x_j'^2$, то есть появился квадрат с коэффициентом, не равным нулю $\Longrightarrow$ можно перейти к общему случаю. 
  \end{proof}
  \begin{remark}
    В благоприятном случае, когда на первом шаге коэффициент при $x_1$ не равен нулю, на втором шаге коэффициент при $x_2$ не равен нулю и т.д., матрица замены будет иметь вид:\\
    \begin{center}
      $C_{e\rightarrow e'}^{-1}=\begin{pmatrix}
        1 & \frac{b_{12}}{b_{11}} & \ldots & \frac{b{1n}}{b_{11}}\\
        0 & 1 & \ldots & \frac{b_{1n}}{b_{22}}\\
        \vdots & \null & \null & \vdots\\
        0 & 0 & \ldots & 1
      \end{pmatrix}$ - матрица с 1 на диагонали $\Longrightarrow$ $|C_{e\rightarrow e'}^{-1}|=1\neq0$.
    \end{center}
  \end{remark}
  \begin{definition}
    Форма $k(x_1,\ldots,x_n)$ называется канонической(нормальной), если:\\
    1. (над $\R$) в диагональном виде $\forall \alpha_i$ принимает только такие значения: -1, 0, 1.\\
    2. (над $\CC$) в диагональном виде $\forall \alpha_i$ принимает только такие значения: 0, 1.
  \end{definition}
  1. Пусть $\F=\R$ и $k(x)=b_{11}x_1^2+b_{22}x_2^2+\ldots+b_{nn}x_n^2 = \alpha_1x_1^2+\alpha_2x_2^2+\ldots+\alpha_nx_n^2$.\\ 
  Если $rkB=r$, то $k(x)=\alpha_1x_1^2+\alpha_2x_2^2+\ldots+\alpha_rx_r^2(\alpha_{r+1}=\ldots=\alpha_n=0)$.\\
  Если $\alpha_i>0$, то введём обозначение $\widehat{x_i}=\sqrt{\alpha_i}x_i$ $\Longrightarrow$ $k=\widehat{x_1}^2+\ldots+\widehat{x_p}^2-\widehat{x_{p+1}}^2-\ldots-\widehat{x_r}^2$, где $p$ - количество коэффициентов $\alpha_i>0$.\\
  Если $\alpha_i < 0$, то $\widehat{x_i} = -\sqrt{\alpha_i}x_i$.\\
  2. Пусть $\F=\CC$, тогда $\forall i=\overline{1,r}$ $\widehat{x_i}=\sqrt{\alpha_i}x_i$ $\Longrightarrow$ $k=\widehat{x_1}^2+\ldots+\widehat{x_r}$.
\end{document}
  