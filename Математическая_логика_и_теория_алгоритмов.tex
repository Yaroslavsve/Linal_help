\documentclass[a4paper, 12pt]{article}

\usepackage{import}

% Корректность отображения всех шрифтов, кодировок и мат. символов
\usepackage[T2A]{fontenc}
\usepackage[utf8]{inputenc}
\usepackage[english, russian]{babel}
\usepackage{amssymb, amsmath, amsthm, mathtools}

% Отображение содержания
\usepackage{tocloft}

% Вставка картинок
\usepackage{graphicx}
\usepackage{tikz}
\usepackage{tkz-euclide}
\usepackage{asymptote}

\usepackage{wrapfig}        % Огибание картинок текстом
\usepackage{cancel}         % Зачёркивания
\usepackage{indentfirst}    % Отступ у первого абзаца
\usepackage{xcolor}         % Цвета
\setlength{\parskip}{.5ex}  % Отступы между абзацами
\usepackage{enumitem}       % Работа со списками
% \usepackage{minted}       % Вставка блоков кода

\usepackage{hyperref}       % гиперссылки
\definecolor{linkcolor}{HTML}{225ae2} % Цвет ссылок
\definecolor{urlcolor}{HTML}{225ae2} % Цвет гиперссылок
\hypersetup{
    pdfstartview=FitH, 
    linkcolor=linkcolor,
    urlcolor=urlcolor,
    colorlinks=true}
\setlength{\arrayrulewidth}{0.5mm} %Толщина линейки в таблицах
\setlength{\tabcolsep}{18pt} %Разделение между столбцами в таблице

% Отступы на странице
\usepackage[left=2cm, right=1.5cm, top=2cm, bottom=2cm]{geometry}

\usepackage{cmap}            % Русский поиск в PDF документе
\usepackage{etoolbox}
\usepackage{soul}            % Разряженный текст \so{} и подчеркивание \ul{}
\usepackage{soulutf8}        % Поддержка UTF8 в soul

\usepackage{titlesec}        % Форматирование заголовков
\titleformat{\section}{\LARGE \bfseries}{\thesection}{1em}{}
\titleformat{\subsection}{\Large\bfseries}{\thesubsection}{1em}{}
\titleformat{\subsubsection}{\large\bfseries}{\thesubsubsection}{1em}{}

\newcommand{\R}{\mathbb R}
\newcommand{\Q}{\mathbb Q}
\newcommand{\Z}{\mathbb Z}
\newcommand{\N}{\mathbb N}
\newcommand{\CC}{\mathbb C}
\newcommand{\F}{\mathbb F}
\newcommand{\aug}{\fboxsep=-\fboxrule\!\!\!\fbox{\strut}\!\!\!}
\newcommand{\sgn}{\operatorname{sgn}}
\newcommand{\id}{\mathrm{id}}
\renewcommand{\phi}{\varphi}
\renewcommand{\epsilon}{\varepsilon}

\newsavebox{\boxedalignbox}
\newenvironment{boxedalign*}
  {\begin{equation*}\begin{lrbox}{\boxedalignbox}$\begin{aligned}}
  {\end{aligned}$\end{lrbox}\fbox{\usebox{\boxedalignbox}}\end{equation*}}

\newcommand\tab[1][.5cm]{\hspace*{#1}}

% Подписи для матриц
\newcommand\undermat[2]{\makebox[0pt][l]{$\smash{\underbrace
{\phantom{\begin{matrix}#2\end{matrix}}}_{\text{$#1$}}}$}#2}
\newcommand\overmat[2]{\makebox[0pt][l]{$\smash{\overbrace
{\phantom{\begin{matrix}#2\end{matrix}}}^{\text{$#1$}}}$}#2}

% Значек "пусть"
\newlength{\tempheight}  
\newcommand{\Let}[0]{  
\mathbin{\text{\settoheight{\tempheight}{\mathstrut}\raisebox{0.5\pgflinewidth}{%
\tikz[baseline,line cap=round,line join=round] \draw (0,0) --++ (0.4em,0) --++ (0,1.5ex) --++ (-0.4em,0);
}}}}


% \newcounter{lemcount}
% \newcounter{thcount}
% \newcounter{offercount}
% \newcounter{concount}
% \newcounter{subthcount}
% \newcounter{defcount}

\theoremstyle{definition}
\newtheorem*{definition}{Определение}
% \newtheorem{definitionnum}[defcount]{Определение}
\newtheorem*{example}{Примеры}
\newtheorem*{example1}{Пример}
\newtheorem*{exercise}{Упражнение}


\theoremstyle{plain}
\newtheorem*{theorem}{Теорема}
% \newtheorem{theoremnum}[thcount]{Теорема}
\newtheorem*{consequense}{Следствие}
\newtheorem*{consequenses}{Следствия}
% \newtheorem{consequensenum}[concount]{Следствие}
\newtheorem*{lemma}{Лемма}
% \newtheorem{lemmanum}[lemcount]{Лемма}
\newtheorem*{subtheorem}{Утверждение}
% \newtheorem{subtheoremnum}[subthcount]{Утверждение}
\newtheorem*{algorithm}{Алгоритм}
\newtheorem*{properties}{Свойства}
\newtheorem*{properties1}{Свойство}


\theoremstyle{remark}
\newtheorem*{remark}{Замечание}
\newtheorem*{offer}{Предложение}
% \newtheorem{offernum}[offercount]{Предложение}
\begin{document}
  \tableofcontents
  \fontsize{14pt}{20pt}\selectfont
  \newpage
  \section {Множества}
  Построение множеств:
  \begin{definition}
    Множество - $\{x|A(x)\}$, где $A(x)$ - некоторое свойство.
  \end{definition}
  \begin{definition}
    $A\subseteq B$, если все элементы $A$ принадлежат множеству $B$
  \end{definition}
  \begin{definition}
    $A=B$, если множества $A$ и $B$ состоят из одинаковых элементов.
  \end{definition}
  \begin{definition}
    $P(A)=\{B|B\subseteq A\}$ - множество всех подмножеств.
  \end{definition}
  \begin{example}
    Парадокс Рассела: $R=\{x|x\notin x\}$ $R\in R$ и  $R\notin R$.
  \end{example}
  \begin{definition}
    $(a,b):=\{\{a\}, \{a, b\}\}$ - упорядоченная пара.
  \end{definition}
  \begin{definition}
    $A\times B:=\{(a, b)|a\in A\wedge b\in B\}$ - декартого произведение множеств $A$ и $B$.
  \end{definition}
  \begin{definition}
    $R\subseteq A\times A$ - бинарное отношение на множестве $A$.
  \end{definition}
  \begin{definition}
    \begin{enumerate}
        \item Рефлексивность: $\forall x\in A$ $xRx$
        \item Симметричность: $\forall x, y\in A$ ($xRy\to yRx)$
        \item Транзитивность: $\forall x, y, z\in A$ ($xRy\wedge yRz\to xRz)$
        \item Антисимметричность: $\forall x, y\in A$ ($xRy\wedge  yRx\to x=y)$ 
    \end{enumerate}
  \end{definition}
  \begin{definition}
    $R$ - отношение частичного порядка, если оно рефлексивно, транзитивно и антисимметрично.
  \end{definition}
  \begin{definition}
    $R$ - отношение эквивалентности, если оно рефлексивно, транзитивно и симметрично.
  \end{definition}
  \begin{definition}
    $R(x):=\{y|xRy\}$ - класс эквивалентности элемента $x$.\\
    $A/R:=\{R(x)|x\in A\}$ - фактормножество.
  \end{definition}
  \begin{definition}
    $\Gamma$ - разбиение множества $A$, если:
    \begin{enumerate}
        \item $\Gamma\subseteq P(A)$
        \item $\forall B\in \Gamma$ $B\neq\varnothing$
        \item $\forall x\in A$ $\exists!$ $B\in\Gamma$ : $x\in B$
    \end{enumerate}
  \end{definition}
  \begin{theorem}
    $A/R$ - разбиение. Если $\Gamma$ - разбиение $A$, то $\exists!$ $R$ : $\Gamma=A/R$.
  \end{theorem}
  \begin{proof}
    \begin{enumerate} 
      \item $A/R\subseteq P(A)$
      \item $\forall B\in A/R$ $B\neq\varnothing$, потому что $\forall x\in A$ $xRx$ $\Longrightarrow$ $x\in R(x)$
      \item $\forall x\in A $ $\exists B\in A/R$ : $x\in B$. Из-за транзитивности класс $B$ единственный.
    \end{enumerate}
  \end{proof}
  \begin{definition}
    $f\subseteq A\times B$ - функция из множества $A$ в множество в $B$, если $\forall x\in A$ $\exists!$ $y\in B$ : $(x, y)\in f$.\\
    Если $\forall x,x'\in A$ $(f(x)=f(x')\to x=x')$, то $f$ - инъекция.\\
    Если $\forall y\in B$ $\exists$ $x\in A$ : $y=f(x)$, то $f$ - сюръекция.\\
    $f$ - биекция $\Longleftrightarrow$ $f$ - инъекция и сюръекция.
  \end{definition}
  \begin{definition}
    Множества $A$ и $B$ равномощны, если существует биекция $f:A\longrightarrow B$. Обозначение: $A\sim B$.
  \end{definition}
  \begin{subtheorem}
    Ровномощность является отношением эквивалентности.
  \end{subtheorem}
  \begin{proof}
    \begin{enumerate}
        \item $A\sim A$, так как $f$ : $f(x)=x$ - биекция
        \item $A\sim B$, тогда $\exists$ биекция $f:A\longrightarrow B$ $\Longrightarrow$ $f^{-1}$ - биекция из $B$ в $A$
        \item Биекция из $A$ в $C$ - композиция биекций
    \end{enumerate}
  \end{proof}
  \begin{definition}
    Множество $A$ вложимо в $B$, если существует инъекция $f:A\longrightarrow B$.
  \end{definition}
  \begin{theorem}
    Данное отношение является отношением порядка.
  \end{theorem}
  \begin{proof}
    \begin{enumerate}
        \item $A\preceq A$ (так как существует биекция, то существует и инъекция)
        \item $A\preceq B\wedge B\preceq C\to A\preceq C$, так как композиция инъекций является инъекцией
        \item $A\preceq B\wedge B\preceq A\to A\sim B$ (Теорема Кантора - Бернштейна)
    \end{enumerate}
  \end{proof}
  Определим натуральные числа так:\\
  $0=\varnothing$\\
  $1=\{0\}$\\
  $\vdots$\\
  $n+1=n\cup\{n\}$
  $\omega=\{0,1,2,3,\ldots\}$ - множество натуральных чисел.
  Принцип индукции
  Пусть $X\subseteq\omega$, и пусть выполнены свойства:
  \begin{enumerate}
    \item $0\in X$
    \item $\forall n(n\in X\to (n+1)\in X)$
  \end{enumerate}
  Тогда $X=\omega$.
  \begin{theorem}
    Всякое не пустое подмножество $\omega$ имеет наименьший элемент.
  \end{theorem}
  \begin{proof}
    Пусть $\varnothing\not= A\subseteq\omega$ и в $A$ нет наименьшего элемента, тогда
    \begin{enumerate}
      \item $0\notin A$, иначе $0$ - наименьший элемент
      \item $\{0,1,2,\ldots,n\}\cap A=\varnothing$, тогда $n+1\notin A$, иначе $n+1$ - наименьший элемент
    \end{enumerate}
    Следовательно, по принципу математической индукции $\{0,1,2,\ldots,n\}\cap A=\varnothing$ $\forall n$ $\Longrightarrow$ $A=\varnothing$ - противоречие.
  \end{proof}
  \begin{definition}
    Множество называется конечным, если равномощно некоторому натуральному числу.
  \end{definition}
  \begin{lemma}
    Если $m\in n$, то $n\setminus\{m\}\sim(n-1)$
  \end{lemma}
  \begin{proof}
    База индукции:\\
    $n=0$, доказывать нечего.\\
    Пусть верно для $n$, докажем для $n+1$. Возьмём $m\in (n+1)=n\cup\{n\}$. Если $m=n$, то $(n+1)\setminus \{m\}=n$. Если $m\in n$, то применяем предположение индукции, что $n\setminus\{m\}\sim (n-1)$, то есть существует биекция между $n\setminus\{m\}$ и $(n-1)$. Отобразим $n\to (n-1)$, тем самым продлив биекцию до $(n+1)\setminus\{m\}\to n$.
  \end{proof}
  \begin{theorem}(Принцип Дирихле)
    $\forall m, \ n\in\omega(m\sim n\to m=n)$.
  \end{theorem}
  \begin{proof}
    $\Phi(n):=\forall m\{m\sim n\to m=n)\}$.
    База индукции:\\
    $n=0$ $\Longrightarrow$ $m\sim\varnothing$ $\Longrightarrow$ $m=\varnothing$.
    Пусть верно $\Phi(n)$, докажем $\Phi(n+1)$. Предположим $(n+1)\sim m$, тогда существует биекция $f$ : $n\cup\{n\}\to m$. Пусть $k=f(n)$, тогда $g$ : $n\to m\setminus\{k\}$ - биекция $\Longrightarrow$ $n\sim m\setminus\{k\}$. По лемме $m\setminus\{k\}\sim(m-1)$ $\Longrightarrow$ по транзитивности $n\sim m-1$. По предположению индукции $n=m-1$ $\Longrightarrow$ $n+1=m$.
  \end{proof}
  \begin{definition}
    Для конечного множества $x$ полагаем, что $|x|=n$, если $x\sim n$. 
  \end{definition}
  \begin{definition}
    Для $m, n\in\omega$ $m<n:=m\in n$, $m\leqslant n:=m<n\vee m=n$.
  \end{definition}
  \begin{properties}
    \begin{enumerate}
      \item $m<n\wedge n<k\to m<k$(транзитивность)
      \item $m\leqslant n\to m\subseteq n$
      \item $n\not< n$(иррефлексивность)
      \item $n<m\leftrightarrow n+1\leqslant m$ (дискретность вверх)
      \item $n<m\vee m<n\vee m=n$
    \end{enumerate}
  \end{properties}
  \begin{lemma}
    Если $A$ конечно, то $A\cup\{x\}$ конечно. 
  \end{lemma}
  \begin{proof}
    Если $x\in A$, то $A\cup\{x\}=A$ конечно.
    Пусть $x\notin A$, $A\sim n$, тогда $A\cup\{x\}\sim n+1$ 
  \end{proof}
  \begin{lemma}
    Если $A$ конечно, то $\forall n\in\omega$ $A\cup n$ конечно.
  \end{lemma}
  \begin{proof}
    База индукции: $n=0$, $A\cup\varnothing=A$ конечно.
    Пусть $A\cup n$ конечно. Запишем $A\cup(n+1)=A\cup(n\cup\{n\})=(A\cup n)\cup\{n\}$ - конечное множество по предположению индукции и по предыдущей лемме, последний переход по транзитивности.
  \end{proof}
  \begin{lemma}
    Если множества $A$ и $B$ конечны и $A\cap B=\varnothing$, то $A\cup B$ конечно.
  \end{lemma}
  \begin{proof}
    Так как $B$ конечно, то $B\sim n\in\omega$, то есть существует биекция $f:B\to n$ $\Longrightarrow$ существует биекция $g: A\cup B\to A\cup n$, тождественная на $A$ и совпадает с $f$ на $B$. По транзитивности и предыдущй лемме имеем, что $A\cup B$ конечно.
  \end{proof}
  \begin{theorem}
    Если множества $A$ и $B$ конечны, то $A\cup B$ конечно.
  \end{theorem}
  \begin{proof}
    $A\cup B=(A\setminus B)\cup B$. Докажем, что $A\setminus B$ - конечное множество. $A\setminus B\subseteq A$ и докажем, что любое подмножество $C$ конечного множества конечно.\\
    База индукции: $|A|=0$, тогда $|C|=0$. Пусть верно для $|A|=n$. Рассмотрим $|A|=n+1$, если $C=A$, то $C$ конечно. Пусть $C\subset A$, тогда $\exists a\in A$ и $a\notin C$ $\Longrightarrow$ $|A\setminus\{a\}|=n$ $\Longrightarrow$ $C\subseteq A\setminus\{a\}$ $\Longrightarrow$ по предположению индукции $C$ конечно. Значит, $A\setminus B$ конечно. По последней лемме $A\cup B$ конечно.
  \end{proof}
  \begin{theorem}
    Если множества $A$ и $B$ конечны, то $A\times B$ конечно.
  \end{theorem}
  \begin{proof}
    База индукции: $|B|=0$, тогда 
  \end{proof}
  \begin{definition}
    $m+n:=|A\cup B|$, где $|A|=m$, $|B|=n$, $A\cap B=\varnothing$.
    $m\cdot n:=|A\times B|$, где $|A|=m$, $|B|=n$.
    $B^A:=\{f|f\textup{ - функция из A в B}\}$.
  \end{definition}
  \begin{lemma}
    $\forall n\in\omega$ $B^{n+1}\sim B^n\times B$.
  \end{lemma}
  \begin{proof}
    Пусть $f: n\to B$, $b\in B$, тогда построим функцию $g:(n+1)\to B$, которая совпадает с $f$ на множестве $n$ и переводит $n+1$ в $b$. Она задаёт биекцию между $B^n\times B\to B^{n+1}$.
  \end{proof}
  \begin{lemma}
    Если множество $B$ конечно, то $\forall n\in\omega$ $B^n$ конечно.
  \end{lemma}
  \begin{proof}
    По последней теореме и предыдущей лемме по индукции получаем данное утверждение.
  \end{proof}
  \begin{lemma}
    Если $A\sim C$, то $B^A\sim B^C$.
  \end{lemma}
  \begin{proof}
    Пусть дана биекция $g:A\to C$. По функции $f:C\to B$ строим композицию $(f\cdot g): A\to B$. Это задаёт биекцию $B^C$ на $B^A$.
  \end{proof}
  \begin{theorem}
    Если множества $A$ и $B$ конечны, то $B^A$ конечно.
  \end{theorem}
  \begin{proof}
    Из двух последних лемм получаем данное утверждение.
  \end{proof}
  \begin{definition}
    $m^n:=|B^A|$, где $|B|=m$, $|A|=n$.
  \end{definition}
  \begin{definition}
    Характеристической функцией для любого подмножества $A$ множества $X$ называется функция $f$ такая, что $f(x)=\begin{cases}
      1\textup{, если } x\in A\\
      0\textup{, если } x\notin A
    \end{cases}$ $\forall x\in X$
  \end{definition}
  \begin{theorem}
    $2^A\sim P(A)$.
  \end{theorem}
  \begin{proof}
    Между подмножествами множества $A$ и их характеристическими функция существует биекция, а множество всех характеристических функций равно $2^A$.
  \end{proof}
  \begin{theorem}{Теорема Кнтора}
    $|A|<|P(A)|$.
  \end{theorem}
  \begin{proof}
    $A\preceq P(A):$ инъекция $a\to\{a\}$. Предполодим, что $f:A\to P(A)$ - биекция, и рассмотрим $B=\{a|a\notin f(a)\}$.
  \end{proof}
  \begin{theorem}
    Даны множество $Y$, $y_0\in Y$, функция $h:Y\to Y$. Тогда существует единственная функция $f:\omega\to Y$ такая, что
    \begin{enumerate}
      \item $f(0)y_0$
      \item $\forall n$ $f(n+1)=h(f(n))$
    \end{enumerate} 
  \end{theorem}
  \begin{proof}
    Назовём $m$-функцией такую же функцию, как в формулировке теоремы, но заданную на множестве $m+1$.\\
    База индукции: $\{(0, y_0)\}$ единственная 0-функция.\\
    Шаг индукции: если $g$ - $m$-функция, то, добавив $\{(m+1,h(g(m)))\}$, получим $m+1$-функцию.\\Объединив все $m$-функции, получим искомую $\omega$-функцию $f$.
  \end{proof}
  \begin{definition}
    Множество $A$ называется счётным, если $A\sim\omega$.
  \end{definition}
  \begin{theorem}
    \begin{enumerate}
      \item Если $A$ конечно, $B$ счётно, то $A\prec B$.
      \item Если $A$ счётно, $B\subseteq A$, то $B$ конечно или счётно.
      \item Если $A$ счётно, $B$ конечно или счётно, то $A\cup B$ счётно.
      \item Если $A$ счётно, $B$ конечно или счётно, то $A\times B$ счётно.
    \end{enumerate}
  \end{theorem}
  \begin{proof}
    \begin{enumerate}
      \item По транзитивности $A\preceq B$. Пусть $A\sim B$, тогда $\omega\sim n$ по транзитивности. Из того что $n+1\subseteq\omega$ следует, что $n+1\preceq n$, что противоречит принципу Дирихле.
      \item Пусть $A=\omega$, $B\subseteq A$ бесконечно. По рекурсии построим биекцию $f:\omega\to B$. Сначала построим функцию $F$ такую, что $F(0)=\{min(B)\}$, $F(n+1)=F(n)\cup min(B\setminus F(n))$.
      \item Пусть $A$ и $B$ счётны и не пересекаются, тогда предъявим обход по элементам как было в курсе математического анализа в первом семестре.
      \item $\omega\times\omega\sim\omega$: канторовская нумерация.
    \end{enumerate}
  \end{proof}
  \begin{definition}
    Множество бесконечно по Дедекинду (D-бесконечно), если оно равномощно какому-нибудь своему собственному подмножеству.
  \end{definition}
  \begin{theorem}
    \begin{enumerate}
      \item Конечное множество D-конечно.
      \item Счётное множество D-бесконечно.
      \item Если $A$ счётно, $A\subseteq B$, то $B$ D-бесконечно.
      \item $A$ D-бесконечно $\Longleftrightarrow$ $A$ содержит счётное подмножество.
      \item Если $A$ D-бесконечно, $B$ конечно, то $A\cup B\sim A\setminus B\sim A$.
      \item Если $A$ D-бесконечно, $B$ счётно, то $A\cup B\sim A$.
    \end{enumerate}
  \end{theorem}
  \begin{proof}
    \begin{enumerate}
      \item По принципу Дирихле.
      \item Пусть $A$ счётно, тогда $\omega\subseteq A$ и $A\sim\omega$ $\Longrightarrow$ по определению $A$ D-бесконечно.
      \item Построим отображение из $B$ в $A$,
      \item $\underline{\Longleftarrow}$ По утверждению 3.\\
      $\underline{\Longrightarrow}$ Пусть $f:A\to B$ - биекция, где $B\subset A$. Если $A\in A\setminus B$, то рассмотрим $\{f(a),f(f(a)),\ldots\}$. Таким образом получили искомое счётное множество.
      \item Если $C\subseteq A$ - счётное множество, $B\cap A=\varnothing$ и $B\sim n$, то строим биекцию из $A\cup B$ на $A$: сдвигаем все элементы $C$ на $n$, а на освободившиеся места отображаем $B$. $A=(A\setminus B)\cup B$, тогда по утверждению 3 $A\setminus B$ D-бесконечно, так как оно содержит счётное подмножество $C\setminus B$, а по уже доказанному в утверждении пункту получаем, что $A\sim A\setminus B$.
      \item Если $C\subseteq A$ счётно, $B\cap A=\varnothing$ и $B\sim\omega$, то строим биекуию из $A\cup B$ на $A$: удваиваем номера всех элементов $C$, а на нечётные места отображаем $B$.
    \end{enumerate}
  \end{proof}
  \begin{theorem}
    Всякое бесконечное множество D-бесконечно.
  \end{theorem}
  \begin{proof}
    Построим инъекцию между бесконечным множеством и $\omega$, тогда по утверждению 4 данное множество D-бесконечно.
  \end{proof}
  \begin{definition}
    $c:=2^{\omega}$ - континуум.
  \end{definition}
  \begin{definition}
    Пусть $X\not=\varnothing$, $\leqslant$ - бинарное отношение на $X$
    \begin{enumerate}
      \item $\leqslant$ - (частичный) порядок, если $\leqslant$ - рефлексивно, транзитивно и антисимметрично;
      \item $\leqslant$ - линейный порядок, если $\leqslant$ - частичный порядок, а также $\forall x, y\in X$ $(x\leqslant y\vee y\leqslant x)$;
      \item $\leqslant$ - полный порядок, если $\leqslant$ - частичный порядок, а также любое непустое подмножество $X$ имеет наименьший элемент.
    \end{enumerate}
    Обозначение: $x<y:=x\leqslant y\wedge x\neq y$.
    Соответствующие пары $(X,\leqslant)$ называются:
    \begin{enumerate}
      \item (частично) упорядоченным множеством
      \item линейно упорядоченным множеством
      \item вполне упорядоченным множеством (ВУМ)
    \end{enumerate}
  \end{definition}
  \begin{definition}
    Пусть $\alpha=(X,\leqslant)$, $\beta=(X',\leqslant')$ - вполне упорядоченные множества. Изоморфизмом $\alpha$ на $\beta$ называется отображение $f:X\to X'$, которое является биекцией и сохраняет порядок, то есть $x\leqslant y\to f(x)\leqslant f(y)$.
  \end{definition}
  \begin{lemma}{Свойства изоморфизмов}
    \begin{enumerate}
      \item $x<y\to f(x)<f(y)$
      \item Если $f$ - изоморфизм $\alpha$ на $\beta$, то обратная функция $f^{-1}$ - изоморфизм $\beta$ на $\alpha$
      \item Композиция изоморфизмов - изоморфизм
    \end{enumerate}
  \end{lemma}
  \begin{definition}
    Пусть $\alpha$, $\beta$ - вполне упорядоченные множества, тогда $\alpha$ изоморфно $\beta$, если существует изоморфизм $\alpha$ на $\beta$. Обозначение: $\alpha\cong\beta$.
  \end{definition}
  \begin{lemma}
    \begin{enumerate}
      \item $\alpha\cong\alpha$
      \item Если $\alpha\cong\beta$, то $\beta\cong\alpha$
      \item Если $\alpha\cong\beta$ и $\beta\cong\gamma$, то $\alpha\cong\gamma$
    \end{enumerate}
  \end{lemma}
  \begin{definition}
    Пусть $\alpha=(X,\leqslant)$ - вполне упорядоченное множество. Начальный отрезок $\alpha$ - подмножество $Y\subseteq X$, замкнутое по убыванию: $\forall y\in Y$ $\forall x\in X$ $(x<y\to x\in Y)$.
  \end{definition}
  \begin{lemma}
    Пусть $Y$ - собственный начальный отрезок вполне упорядоченного множества $(X,\leqslant)$. Тогда $Y=\{x|x<a\}$ для некоторого $a\in X$.
  \end{lemma}
  \begin{proof}
    
  \end{proof}
  \begin{lemma}
    Пусть $\alpha=(X,\leqslant)$ - вполне упорядоченное множество, $f:X\to X$ - строго возрастающая функция. Тогда $\forall x\in X$ $x\leqslant f(x)$.
  \end{lemma}
  \begin{proof}
    Рассмотрим множество $Y:=\{x\in X|f(x)<x\}$. Предположим, что $Y\neq\varnothing$, и пусть $x$ - наименьший элемент в $Y$. Тогда: $f(f(x))<f(x)$, так как $f$ строго возрастает. Но так как $x$ - минимальный элемент, то $f(x)\leqslant f(f(x))$ - противоречие. Значит, $Y=\varnothing$.
  \end{proof}
  \begin{theorem}
    Никакие вполне упорядоченное множество не изоморфно своему начальному отрезку.
  \end{theorem}
  \begin{proof}
    Предположим противное, пусть $f$ - изоморфизм ($X,\leqslant$) на $(Y=\{x\in X|x < a\},\leqslant'$, где $\leqslant'$ - ограничение $\leqslant$ на $Y$. Тогда $f(a)<a$, что противоречит предыдущей лемме.
  \end{proof}
  \begin{theorem}{Теорема Кантора о сравнении вум}
    Пусть $\alpha$, $\beta$ - вполне упорядоченные множества, тогда верно ровно одно из 3 утверждений:
    \begin{enumerate}
      \item $\alpha$ изоморфно собствееному начальному отрезку $\beta$
      \item $\beta$ изоморфно собствееному начальному отрезку $\alpha$
      \item $\alpha\cong\beta$
    \end{enumerate}
  \end{theorem}
  \begin{proof}
    Пусть $\alpha=(X,\leqslant)$, $\beta=(Y,\leqslant')$. Рассмотрим отношение $f:=\{(x,y)|x\in X, y\in Y, \{x_1\in X|x_1<x\}\cong\{y_1\in Y|y_1<y\}\}$
    \begin{enumerate}
      \item По последней теореме для любого $x$ существует не более одного $y$ такого, что $(x, y)\in f$, и для любого $y$ существует не более одного $x$ такого, что $(x,y)\in f$. Следовательно, $f$ - биекция $X_0\subseteq$ на $Y_0\subseteq Y$.
      \item Докажем, что $X_0$ - начальный отрезок $\alpha$ и $Y_0$ - начальный отрезок $\beta$. Пусть $x\in X_0$ и $\{x_1\in X| x_1<x\}\cong\{y_1\in Y|y_1<y\}$. Пусть $h$ - данный изоморфизм. Возьмём $z<x$, тогда, сужая $h$ на $z$ получим $\{x\in X| x < z\}\cong\{y\in Y|y < h(z)\}$. Так как $h$ сохраняет порядок, потому что $h$ - изоморфизм, и у любого элемента множества $\{y\in Y|y < h(z)\}$ есть прообраз, то $z\in X_0$.
      \item Докажем, что $f$ сохраняет порядок. Пусть $x\in X_0$, $\{x_1\in X| x_1<x\}\cong\{y_1\in Y|y_1<y\}$, $h$ - данный изоморфизм и $z < x$. Так же как и в прошлом пункте ограничим $h$ на $z$, тогда $\{x_1\in X| x_1<z\}\cong\{y_1\in Y|y_1<h(z)\}$, то есть $f(z)=h(z)$. $h(z)< y=f(x)$ $\Longrightarrow$ $f(z)<f(x)$, то есть $f$ сохраняет порядок в силу произвольности $z$.
      \item Из предыдущих пунктов имеем, что $f$ - изоморфизм начального отрезка в $\alpha$ на начальный отрезок в $\beta$. Если оба этих отрезка собственные, то по предпоследней лемме $X_0=\{x_1\in X| x_1<x\}$ и $Y_0=\{y_1\in Y|y_1<y\}$ для некоторых $x$ и $y$. Тогда $(x,y)\in f$ по определению $f$, а значит, $x\in X_0=\{x_1\in X|x_1 < x\}$ - противоречие. Следовательно, один из этих отрезков не собственный, то есть остаётся только 3 варианта, перечисленные в условии теоремы, а по предыдущей теореме они не изоморфны друг другу.
    \end{enumerate}
  \end{proof}
  \subsection{Ординальные числа}
  \begin{definition}
    Ординальное число (по Кантору) представитель класса эквивалентности по $\cong$.
  \end{definition}
  \begin{theorem}
    В любом не пустом множестве $S$ ординальных чисел существует наименьшее.
  \end{theorem}
  \begin{proof}
    Выберем $\alpha\in S$. Рассмотрим $\{\beta\in S|\beta\leqslant\alpha\}$. Среди элементов данного множества есть наименьший, так как в непустом множестве начальных отрезков числа $\alpha$, которым изоморфны эти $\beta$, есть наименьший.
  \end{proof}
  \begin{theorem}{Теорема о трансфинитной индукции}
    Пусть дано свойство $\Phi(\alpha)$ ординальных чисел, которое наследуется, то есть $\forall\alpha$ $(\forall\beta < \alpha$ $\Phi(\beta)\to\Phi(\alpha))$. Тогда $\forall\alpha\Phi(\alpha)$.
  \end{theorem}
  \begin{proof}
    Предположим противное, пусть для некоторого $\alpha_0$ $\Phi(\alpha_0)$ не верно. Рассмотрим $S=\{\beta<\alpha_0|\Phi(\beta)$ не верно$\}$. По предыдущей теореме найдётся наименьшее ординальное число $\beta_0$, тогда $\forall\beta<\beta_0$ $\Phi(\beta)$ верно. Поскольку $\Phi$ наследуется, получаем, что $\Phi(\beta_0)$ верно - противоречие.
  \end{proof}
  \begin{definition}
    $\alpha+1$ получается прибавлением наибольшего элемента к $\alpha$.
  \end{definition}
  \begin{definition}
    Предельное ординальное число не имеет предыдущего.
  \end{definition}
  \begin{subtheorem}
    Пусь $X$ - произвольное множемтво ординальных чисел, тогда существует орддинальное число $\alpha$ большее любого $\beta\in X$. Если в $X$ нет наибольшего, то существует $\alpha=supX$.
  \end{subtheorem}
  \begin{theorem}{Парадокс Бурали-Форти}
    Все ординальные числа не образуют ординального числа.
  \end{theorem}
  \begin{proof}
    Так как отношение вложимости ординалов является отношением линейного порядка по теореме сравнения, то из существования наименьшего элемента имеем, что этот порядок полон. Тогда он должен задавать ординальное число, которое больше всех ординальных чисел, но самого большого ординального числа не существует, так как любое число можно увеличить на 1.
  \end{proof}
  \begin{theorem}{теорема о трансфинитной рекурсии}
    Пусть $\alpha_0$ - ординальное число, $H$ - функция на ординальных числа. Тогда $\exists!$ функция $F$ на ординальных числах такая, что:
    \begin{enumerate}
      \item $F(0)=\alpha_0$
      \item $F(\alpha+1)=H(F(\alpha))$
      \item $F(\alpha)=sup\{F(\beta)|\beta<\alpha\}$, если $\alpha$ - предельное ординальное число.
    \end{enumerate}
  \end{theorem}
  Введём операции на ординальных числах:
  \begin{enumerate}
    \item Сложение: $\alpha+0=\alpha$, $\alpha+(\beta+1)=(\alpha+\beta)+1$, $\alpha+\beta=sup\{\alpha+\gamma|\gamma<\beta\}$, если $\beta$ - предельное ординальное число.
    \item Умножение: $\alpha\cdot0=0$, $\alpha\cdot(\beta+1)=(\alpha\cdot\beta)+\alpha$, $\alpha\cdot\beta=sup\{\alpha\cdot\gamma|\gamma<\beta\}$, если $\beta$ - предельное ординальное число.
    \item Возведение в степень: $\alpha^0=1$, $\alpha^{\beta+1}=\alpha^{\beta}\cdot\alpha$, $\alpha^{\beta}=sup\{\alpha^{\gamma}|\gamma<\beta\}$, если $\beta$ - предельное ординальное число.
  \end{enumerate}
  \begin{definition}
    Ординал - множество специального вида, обобщение натуральных чисел Фон Неймана.
  \end{definition}
  \begin{theorem}{Теорема счёта}
    Любое вум изоморфно некоторому ординалу.
  \end{theorem}
\end{document}