\documentclass[12pt]{article}
\usepackage[left=1cm,right=1cm,top=0.5cm,bottom=1.5cm,bindingoffset=0cm]{geometry}
\usepackage[T1,T2A]{fontenc}
\usepackage[utf8]{inputenc}
\usepackage[english,russian]{babel}
% Пакет, необходимый для корректной работы babel
\usepackage{csquotes}
\usepackage{graphicx}
% Описание пути к изображениям.
\graphicspath{ {./pics} }
\usepackage{amsmath}
\usepackage{amsfonts}
\usepackage{amssymb}
\usepackage{amsthm}
\usepackage{mathtools}
% Пакет для удобной работы с системами уравнений
\usepackage{systeme}
% Пакет, позволяющий рисовать красивые картинки прямо внутри пдф файла
\usepackage{tikz}
\usepackage{hyperref}
\usepackage{xcolor}
\definecolor{linkcolor}{HTML}{225ae2}
\definecolor{urlcolor}{HTML}{225ae2}
\hypersetup{
    pdfstartview=FitH, 
    linkcolor=linkcolor,
    urlcolor=urlcolor,
    colorlinks=true
}
% Выделение ссылок
\newcommand\tab[1][.5cm]{\hspace*{#1}}
%отступ после начала окружения
\usepackage{bbm}
\usepackage[normalem]{ulem}
\theoremstyle{plain}
\newtheorem{theorem}{Теорема}[section]
\newtheorem{consequence}{Следствие}[theorem]
\newtheorem{subtheorem}{Утверждение}[section]
\newtheorem{algorithm}{Алгоритм}[section]
\newtheorem{axiom}{Аксиома}[section]
\newtheorem*{exercise}{Упражнение}
\newtheorem{propertes}{Свойства}
\newtheorem{lemma}{Лемма}[section]
\newtheorem{example}{Пример}
\newtheorem{examples}{Примеры}
% Окружение типа ``определение''
\newtheorem{definition}{Определение}[section]
\newtheorem{remark}{Замечание}
% Замечение без нумерации
\renewcommand\qedsymbol{$\blacksquare$}
\newcommand{\mP}{\mathbf{P}}
\newcommand{\mE}{\mathbf{E}}
\newcommand{\mD}{\mathbf{D}}
\newcommand{\skob}[1]{\left(#1\right)}
\newcommand{\fig}[1]{\left\{#1\right\}}
\newcommand{\br}[1]{\left[#1\right]}
\newcommand{\R}{\mathbb R}
\newcommand{\Q}{\mathbb Q}
\newcommand{\Z}{\mathbb Z}
\newcommand{\N}{\mathbb N}
\newcommand{\CC}{\mathbb C}
\newcommand{\acts}{\curvearrowright}
\newcommand{\F}{\mathbb F}
\newcommand{\aug}{\fboxsep=-\fboxrule\!\!\!\fbox{\strut}\!\!\!}
\newcommand{\sgn}{\operatorname{sgn}}
\newcommand{\id}{\mathrm{id}}
\renewcommand{\phi}{\varphi}
\renewcommand{\epsilon}{\varepsilon}
\newcommand{\E}{\mathcal E}
\begin{document}
    \tableofcontents
    \fontsize{14pt}{20pt}\selectfont
    \section{Кривые}
    \subsection{Кривые на плоскости}
    Здесь и далее координаты векторов обозначаем верхними индексами, производные обозначаем так: $\dot{a}$.
    \begin{definition}
        Гладкая элементарная регулярная вложенная плоская кривая --- множество точек $\gamma$ на плоскости, задаваемое параметрическими уравнениями
        $$z^1=u(t)$$
        $$z^2=v(t)$$
        $$z(t)=(u(t),v(t))$$
        причём
        \begin{enumerate}
            \item $t\in[a, \ b]$
            \item $z(t)\in C^{\infty}[a,\ b]$ (гладкость)
            \item $\forall t\in[a, \ b] \ \dot{z}(t)=(\dot{u}(t), \ \dot{v}(t))\neq0$ (вектор скорости кривой не обращается в 0 на отрезке $[a, \ b]$ (регулярность))
            \item если $z(t_1)=z(t_2)$, то $t_1=t_2$
        \end{enumerate}
    \end{definition}
    \begin{example}
        $$z^1(t)=t^3$$
        $$z^2(t)=t^2$$
        $$t\in[-1, \ 1]$$
        
    \end{example}
    \begin{exercise}
        Пусть $\gamma$ --- прямой угол, необходимо задать $z=r(t)$, где $r(t)\in C^{\infty}$.\\
    \end{exercise}
    \begin{definition}
        Замена параметра вида: $t(\tau)$, где $\tau\in[\alpha,\ \beta]$, $t(\tau)\in C^{\infty}[\alpha, \ \beta]$, $t'(\tau)\neq 0$, называется допустимой заменой.
    \end{definition}
    \begin{definition}
        Касательной к кривой $\gamma$ в точке $r(t_0)$ называется прямая, проходящая через эту точку в направлении вектора $\dot{r}(t_0)$. $z=r(t_0)+\tau\cdot\dot{r}(t_0)$.
    \end{definition}
    \begin{definition}
        Нормалью к кривой $\gamma$ в точке $r(t_0)$ называется прямая, проходящая через эту точку и ортогональная касательной.
    \end{definition}
    \begin{definition}
        Длина дуги кривой между точками $t_1$ и $t_2$ называется число $$l=\int\limits_{t_1}^{t_2}|\dot{r}(t)|dt$$
    \end{definition}
    \begin{exercise}
        Доказать, что $l$ не зависит от допустимой параметризации.
    \end{exercise}
    Рассмотрим кривую $\gamma \ : \ z=r(t)$, рассмотрим точку $t_0$ на этой кривой и функцию $s(t)=\int\limits_{t_0}^t|\dot{r}(t_1)|dt_1$. Зададим параметризацию $t=t(s)$.
    \begin{definition}
        Такая параметризация называется натуральной параметризацией, а $s$ называется натуральным параметром.
    \end{definition}
    \begin{remark}
        Если кривая $\gamma$ задана натуральной параметризацией, то $$l=|s(t_2)-s(t_1)|$$
    \end{remark}
    \begin{remark}
        $v(s)=1$, обозначение: $\rho'(s):=v(s)$.
    \end{remark}
    Пусть даны две кривые $\gamma$ и $\tilde{\gamma}$, $s_0$ --- их общая точка.
    $$\gamma \ : \ z=\rho(s)$$
    $$\tilde{\gamma} \ : \ z=\tilde{\rho(s)}$$
    $$\rho(s_0)=\tilde{\rho}(s_0)$$
    \begin{definition}
        Кривые $\gamma$ и $\tilde{\gamma}$ имеют в точке $s_0$ касание порядка $k$, если натуральные параметры можно выбрать так, что $\rho(s)-\tilde{\rho}(s)=(s-s_0)^k$.
    \end{definition}
    \begin{subtheorem}
        Кривые $\gamma$ и $\tilde{\gamma}$ имеют в точке $s_0$ касание порядка $k$ тогда и только тогда, когда натуральные параметры можно выбрать так, что $$\rho(s_0)=\tilde{\rho}(s_0), \ \rho'(s_0)=\tilde{\rho}'(s_0), \ \ldots, \ \rho^{(k)}(s_0)=\tilde{\rho}^{(k)}(s_0)$$.
    \end{subtheorem}
    \begin{proof}
        По формуле Тейлора с остаточным членом в форме Пеано $$\rho(s)-\tilde{\rho}(s)=\sum\limits_{j=0}^k\frac{\rho^{(j)}(s_0)-\tilde{\rho}^{(j)}(s_0)}{j!}\cdot(s-s_0)^j+\overline{\overline{o}}((s-s_0)^k)$$
    \end{proof}
    \begin{lemma}
        \hypertarget{d1}{}
        Пусть $\alpha(t)\in\R^n$, $\alpha(t)$ ---бесконечно дифференцируемая вектор-функция и $l=const$, тогда $\alpha\perp\dot{\alpha}$.
    \end{lemma}
    \begin{proof}
        $l^2=(\alpha, \ \alpha)=const$ $\Longrightarrow$ $|\alpha|^2=const$ $\Longrightarrow$ $(|\alpha|^2)'=2(\alpha, \ \dot{\alpha})=0$.
    \end{proof}
    \begin{consequence}
        Если $z=\rho(s)$ --- нормальная параметризация кривой $\gamma$, то $(\ddot{\rho}, \ \dot{\rho})=0$.
    \end{consequence}
    \begin{remark}
        Далее вместо $\dot{\rho}$ пишем $\rho'$, имея ввиду, что $\rho$ --- натуральная параметризация.
    \end{remark}
    \begin{theorem}
        Пусть в некоторой точке $s_0$ кривой $\gamma$ $\rho''(s_0)\neq0$, тогда существует единственная окружность, имеющая касание второго порядка в точке $s_0$, причём центр этой окружности находится на нормали к кривой $\gamma$ в направлении $\rho''(s_0)$, а её радиус $R=\frac{1}{|\rho''(s_0)|}$.
    \end{theorem}
    \begin{proof}
        Рассмотрим окружность $\tilde{\gamma}$ с центром в точке $z_0$ и радиуса $R=\frac{1}{|\rho''(s_0)|}$. Зададим эту окружность так:
        $$\tilde{\gamma} \ : \ z=z_0+\begin{pmatrix}
            a\cos(\frac{s}{a})\\
            a\sin(\frac{s}{a})
        \end{pmatrix}$$
        Тогда $\tilde{\rho}'(s)=\begin{pmatrix}
            -\sin(\frac{s}{a})\\
            \cos(\frac{s}{a})
        \end{pmatrix}$
        $$\tilde{\rho}''(s)=\begin{pmatrix}
            -\frac{1}{a}\cos(\frac{s}{a})\\
            -\frac{1}{a}\sin(\frac{s}{a})
        \end{pmatrix}$$
        Значит, $|\tilde{\rho}''(s)|=\frac{1}{a}$.
    \end{proof}
    \begin{definition}
        Построенная окружность называется соприкасающейся окружностью. Её центр называется центром кривизны, её радиус $R$ --- радиусом кривизны, а величина $k=\frac{1}{R}$ --- кривизной кривой.
    \end{definition}
    \begin{remark}
        Пусть $\ddot{a}(s_0)\neq0$, тогда пара $\{v(s_0), \ n(s_0)=\frac{\ddot{a}(s_0)}{|\ddot{a}(s_0)|}\}$ --- ортонормированный репер с центром в точке $s_0$, $n(s_0)$ --- вектор нормали.
    \end{remark}
    \begin{definition}
        Такой репер называется репером Френе.
    \end{definition}
    \begin{subtheorem}[Плоские формулы Френе]\tab
        \hypertarget{2}{}
        \begin{enumerate}
            \item $v'=\frac{1}{R}\cdot n$
            \item $n'=-\frac{1}{R}\cdot v$
        \end{enumerate}
    \end{subtheorem}
    \begin{proof}\tab
        \begin{enumerate}
            \item $n=\frac{\rho''}{|\rho''|}=\frac{v'}{\frac{1}{R}}$, то есть $v'=\frac{1}{R}\cdot n$.
            \item \hyperlink{d1}{По лемме 1.1} $n'\perp n$, тогда $n' \ \| \ v$ $\Longrightarrow$ $n'=\lambda v$. Так как $(v, \ n)=0$, то\\$(v, \ n')+(v', \ n)=0$. По пункту 1. $v'=\frac{1}{R}\cdot n$, тогда $\frac{1}{R}(n, \ n)+\lambda(v, \ v)=0$.\\Векторы $\vec{n}$ и $\vec{v}$ имеют единичную длину, значит, $\lambda=-\frac{1}{R}$, то есть $n'=-\frac{1}{R}\cdot v$.
        \end{enumerate}
    \end{proof}
    \begin{remark}
        Пусть плоскость ориентирована и $v=\begin{pmatrix}
            v_1\\
            v_2
        \end{pmatrix}$, тогда рассмотрим\\$\tilde{n}=\begin{pmatrix}
            -v_2\\
            v_1
        \end{pmatrix}$.\\В таком случае можно не требовать, что вектор скорости нигде не обращается в ноль. При этом $\tilde{k}=\begin{cases}
            k, \ \textup{при } \tilde{n}=n\\
            -k, \ \textup{при } \tilde{n}=-n\\
            0, \ \textup{при } \ddot{a}=0
        \end{cases}$
    \end{remark}
    \begin{subtheorem}\tab
        \begin{enumerate}
            \item Формулы Френе остаются верными при замене $n$ на $\tilde{n}$ и $k$ на $\tilde{k}$.
            \item $\tilde{k}\in C^{\infty}$
        \end{enumerate}
    \end{subtheorem}
    \begin{proof}\tab
        \begin{enumerate}
            \item Так как $\tilde{n}=n$ либо $\tilde{n}=-n$, и знаки $\tilde{n}$ и $\tilde{k}$ совпадают, и $|k|=|\tilde{k}|$, то $kn=\tilde{k}\tilde{n}$.
            \item $|\tilde{n}|=1$, рассмотрим $(v', \ \tilde{n})$. \hyperlink{2}{По первой плоской формуле Френе} $(v', \ \tilde{n})=\tilde{k}\cdot(\tilde{n}, \ \tilde{n})=\tilde{k}=v_1\cdot v_2'-v_2\cdot v_1'$. Так как $v$ --- бесконечно непрерывно дифференцируемая функция, то $\tilde{k}$ тоже.
        \end{enumerate}
    \end{proof}
    \begin{theorem}[о восстановлении кривой по функции кривизны]\tab
        \begin{enumerate}
            \item Пусть $k_0\in C^{\infty}[0, \ l]$, тогда существует кривая $\gamma$ такая, что $k_0$ --- её функция кривизны.
            \item Пусть $\gamma_1$ и $\gamma_2$ --- две кривые такие, что $\tilde{k}_1=\tilde{k}_2$, тогда существует движение $f$ плоскости, переводящее $\gamma_1$ в $\gamma_2$, то есть $f(\gamma_1)=\gamma_2$.
        \end{enumerate}
    \end{theorem}
    \begin{proof}\tab
        \begin{enumerate}
            \item Рассмотрим $\alpha(s)=\int\limits_0^sk_0(s_1)ds_1+\alpha_0$, тогда $\alpha'=k_0$. Рассмотрим $v(s)=\begin{pmatrix}
                \cos(\alpha(s))\\
                \sin(\alpha(s))
            \end{pmatrix}$, тогда $r(s)=\int\limits_0^sv(s_1)ds_1$. Рассмотрим кривую $\gamma \ : \ z=r(s)$. Эта кривая гладкая, так как $v(s)$ --- гладкая функция. Так как $v(s)$ --- бесконечно непрерывно дифференцируемая функция, то $r(s)$ тоже. Из того что $|v(s)|=1\neq 0$, следует, что $\gamma$ регулярна, а так же $r''=v'=\alpha'\cdot\begin{pmatrix}
                -v_2\\
                v_1
            \end{pmatrix}=k_0\cdot\tilde{n}$, то есть $k_0$ --- кривизна кривой $\gamma$.
            \item Соединим начала данных кривых при помощи сдвига. Затем, применим поворот, чтобы их векторы скорости в начальной точке совпали. Так как $\tilde{k}_1=\tilde{k}_2$, то $v_1'=v_2'$, значит, $v_1=v_2+const$. Так как вектора скорости в начальной точке совпадают, то $const=0$. Аналогично $r_1(s)=r_2(s)$. Композиция использованных сдвига и поворота является искомым движением.
        \end{enumerate}
    \end{proof}
    \begin{remark}
        Возможны точки самопересечения, в таких случаях эти точки считаем различными.
    \end{remark}
    \subsection{Кривые в трёхмерном пространстве}
    \begin{definition}
        Гладкая элементарная регулярная вложенная кривая --- множество точек $\gamma\subset\R^3$, задаваемое параметрическими уравнениями
        $$z^1=z^1(t)$$
        $$z^2=z^2(t)$$
        $$z^3=z^3(t)$$
        $$z(t)=(z^1(t), \ z^2(t), \ z^3(t))$$
        причём
        \begin{enumerate}
            \item $t\in[a, \ b]$
            \item $z(t)\in C^{\infty}[a,\ b]$ (гладкость)
            \item $\forall t\in[a, \ b] \ \dot{z}(t)=(\dot{z}_1(t), \ \dot{z}_2(t), \ \dot{z}_3(t))\neq0$ (вектор скорости кривой не обращается в 0 на отрезке $[a, \ b]$ (регулярность))
            \item если $z(t_1)=z(t_2)$, то $t_1=t_2$
        \end{enumerate}
    \end{definition}
    \begin{definition}
        Касательной к кривой $\gamma$ в точке $r(t_0)$ называется прямая, проходящая через эту точку в направлении вектора $\dot{r}(t_0)$. $z=r(t_0)+\tau\cdot\dot{r}(t_0)$.
    \end{definition}
    \begin{definition}
        Нормальная плоскость к кривой $\gamma$ в точке $r(t_0)$ --- прямая, проходящая через эту точку и ортогональная касательной. $(z-r(t_0), \ \dot{r}(t_0))=0$.
    \end{definition}
    \begin{remark}
        Длина дуги, натуральная параметризация,  порядок касания, соприкасающиеся окружности, центр и радиус кривизны и кривизна определяются аналогично плоскому случаю.
    \end{remark}
    Пусть $z=r(s)$ и $r(s_0)\neq0$ для некоторого $s_0$, тогда определены $v=r'(s_0)$ и $n=r''(s_0)$
    \begin{definition}
        Вектор $n$ --- вектор главной нормали к кривой. 
    \end{definition}
    Рассмотрим вектор $b=[v, \ n]$.
    \begin{definition}
        Вектор $b$ называется вектором бинормали. Тройка векторов $\{v, \ n, \ b\}$ является ортонормированным репером с центром в точке $s_0$.
    \end{definition}
    \begin{lemma}\hypertarget{3}{}
        Пусть $A(t)$ и $B(t)$ --- это $n\times n$ гладкие матричные функции, заданные на отрезке $[a, \ b]$, причём $\dot{Q}=A\cdot Q$, тогда
        \begin{enumerate}
            \item если $\forall t\in[a, \ b] \ Q(t)\cdot Q^T(t)=E$, то $\forall t\in[a, \ b] \ A^T(t)=-A(t)$.
            \item если $\forall t\in[a, \ b] \ A(t)^T=-A(t)$ --- ортогональная матрица и $Q(t)$ --- ортогональная матрица хотя бы в одной точке, тогда $\forall t\in[a, \ b] \ Q(t)$ --- ортогональная матрица.
        \end{enumerate}
    \end{lemma}
    \begin{proof}
        Рассмотрим $$(Q^T\cdot Q)'=\dot{Q}^T\cdot Q+Q^T\cdot\dot{Q}=(A\cdot Q)^T\cdot Q+Q^T\cdot A\cdot Q=Q^T(A^T+A)Q$$ $Q^T(A^T+A)Q=0 \ \Longleftrightarrow \ A^T+A=0$, то есть $Q^T\cdot Q=const \ \Longleftrightarrow \ A^T+A=0$.
        \begin{enumerate}
            \item Из условия и равносильности выше следует, что $A^T=-A$.
            \item Так как для некоторого $t\in[a, \ b]$ $Q^T(t)\cdot Q(t)=E$, то из равносильности выше это равенство верно для любого $t$.
        \end{enumerate}
    \end{proof}
    \begin{theorem}[\hypertarget{4}{Пространственные формулы Френе}]
        Пусть $\forall s \ r''(s)\neq0$, тогда существует бесконечно непрерывно дифференцируемая функция $\kappa(s)$ такая, что:
        \begin{enumerate}
            \item $v'=k\cdot n$
            \item $n'=-k\cdot v-\kappa\cdot b$
            \item $b'=\kappa\cdot n$
        \end{enumerate}
    \end{theorem}
    \begin{proof}
        Рассмотрим матрицу $Q(s)=\begin{pmatrix}
            v\uparrow & n\uparrow & b\uparrow
        \end{pmatrix}$, тогда $$Q'(s)=\begin{pmatrix}
            v'\uparrow & n'\uparrow & b'\uparrow
        \end{pmatrix}=A\cdot Q$$\hyperlink{3}{По первому пункту леммы 1.2} $A$ --- кососимметрическая матрица.\\Пусть $A=\begin{pmatrix}
            0 & a_{12} & a_{13}\\
            -a_{12} & 0 & a_{23}\\
            -a_{13} & -a_{23} & 0
        \end{pmatrix}$. $n=\frac{r''}{|r''|}=\frac{v'}{k}$, то есть $v'=k\cdot n$. Тогда $a_{12}=k$ и $a_{13}=0$. Таким образом, матрица $A$ имеет вид: $A=\begin{pmatrix}
            0 & k & 0\\
            -k & 0 & a_{23}\\
            0 & -a_{23} & 0
        \end{pmatrix}$. Обозначим $-a_{23}=\kappa$. Тогда из равенства $Q'=A\cdot Q$ следуют пункты 2 и 3.\\
        Рассмотрим $-(n', \ b)=k\cdot(v, \ b)+\kappa\cdot(b, \ b)=\kappa$
    \end{proof}
    \begin{definition}
        Функция $\kappa(s)$ называется фуннкцией кручения кривой $\gamma$.
    \end{definition}
    \begin{exercise}
        Доказать, что существует вектор $\omega(s)$ такой, что $v'=\omega v$, $n'=\omega n$, $b'=\omega b$
    \end{exercise}
    \begin{definition}
        Кривая $\gamma$ бирегулярна, если $\forall s \ r''(s)\neq0$.
    \end{definition}
    \begin{subtheorem}
        Пусть кривая $\gamma$ бирегулярна, тогда она плоская тогда и только тогда, когда $\kappa\equiv0$.
    \end{subtheorem}
    \begin{proof}
        \hyperlink{4}{По третьему пункту теоремы 1.3} $b'=\kappa n$. Рассмотрим $(b', \ n)=\kappa$.\\$\underline{\Longrightarrow}$ Так как $b=b_0=const$, то $b'\equiv0$ $\Longrightarrow$ $\kappa\equiv0$.\\
        $\underline{\Longleftarrow}$ Так как $\kappa\equiv0$, то $b'\equiv0$ $\Longrightarrow$ $b=b_0=const$. По определению $(b, \ v)=0$, тогда $0=(b_0, \ r')=(b_0, \ r)'$ $\Longrightarrow$ $(b_0, \ r)=const=c_0$.  Последнее уравнение является уравнением плоскости, значит кривая $\gamma$ плоская.
    \end{proof}
    \begin{theorem}[о восстановлении пространственной кривой по кривизне и кручению]\tab
        \begin{enumerate}
            \item Пусть $k_0(s)$ и $\kappa_0(s)$ --- две бесконечно непрерывно дифференцируемые функции на отрезке $[0, \ l]$, причём $\kappa(s)>0$, тогда существует бирегулярная кривая $\gamma$ такая, что $k=k_0$ и $\kappa=\kappa_0$.
            \item Пусть $\gamma$ и $\tilde{\gamma}$ --- две бирегулярные кривые такие, что $k(s)=\tilde{k}(s)$ и $\kappa(s)=\tilde{\kappa}(s)$, тогда существует движение $f:\R^3\to\R^3$, переводящее $\gamma$ в $\tilde{\gamma}$.
        \end{enumerate}
    \end{theorem}
    \begin{proof}\tab
        \begin{enumerate}
            \item Из доказательства пространственных формул Френе $A=\begin{pmatrix}
                0 & k & 0\\
                -k & 0 & -\kappa\\
                0 & -\kappa & 0
            \end{pmatrix}$, рассмотрим матрицу $Q(s)$ такую, что $\dot{Q}(s)=A(s)\cdot Q(s)$ и $Q(0)=\begin{pmatrix}
                v_0\uparrow & n_0\uparrow & b_0\uparrow
            \end{pmatrix}$. Последнее уравнение является системой линейных дифференциальных уравнений с начальными условиями, тогда по теореме Коши существования и единственности решения существует единственное решение $Q(s)=\begin{pmatrix}
                v(s)\uparrow & n(s)\uparrow & b(s)\uparrow
            \end{pmatrix}$. Рассмотрим $r(s)=\int\limits_0^sv(s_1)ds_1+r_0$ и рассмотрим кривую $\gamma \ : \ z=r(s)$. $r'(s)=v(s)$, значит, эта кривая $\gamma$ является гладкой и регулярной, причём $s$ --- нормальный параметр. Так как $r''=v'=k_0\cdot n$, то $|v'|=k_0$, то есть $k_0$ --- кривизна кривой $\gamma$. \hyperlink{4}{По второй пространственной формуле Френе} $n'=-k_0v-\kappa_0b$ $\Longrightarrow$ $-\kappa_0=(n', \ b)$.
        \end{enumerate}
    \end{proof}
\end{document}