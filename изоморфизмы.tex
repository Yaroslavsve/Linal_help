\documentclass[a4paper, 12pt]{article}

\usepackage{import}

% Корректность отображения всех шрифтов, кодировок и мат. символов
\usepackage[T2A]{fontenc}
\usepackage[utf8]{inputenc}
\usepackage[english, russian]{babel}
\usepackage{amssymb, amsmath, amsthm, mathtools}

% Отображение содержания
\usepackage{tocloft}

% Вставка картинок
\usepackage{graphicx}
\usepackage{tikz}
\usepackage{tkz-euclide}
\usepackage{asymptote}

\usepackage{wrapfig}        % Огибание картинок текстом
\usepackage{cancel}         % Зачёркивания
\usepackage{indentfirst}    % Отступ у первого абзаца
\usepackage{xcolor}         % Цвета
\setlength{\parskip}{.5ex}  % Отступы между абзацами
\usepackage{enumitem}       % Работа со списками
% \usepackage{minted}       % Вставка блоков кода

\usepackage{hyperref}       % гиперссылки
\definecolor{linkcolor}{HTML}{225ae2} % Цвет ссылок
\definecolor{urlcolor}{HTML}{225ae2} % Цвет гиперссылок
\hypersetup{
    pdfstartview=FitH, 
    linkcolor=linkcolor,
    urlcolor=urlcolor,
    colorlinks=true}
\setlength{\arrayrulewidth}{0.5mm} %Толщина линейки в таблицах
\setlength{\tabcolsep}{18pt} %Разделение между столбцами в таблице

% Отступы на странице
\usepackage[left=2cm, right=1.5cm, top=2cm, bottom=2cm]{geometry}

\usepackage{cmap}            % Русский поиск в PDF документе
\usepackage{etoolbox}
\usepackage{soul}            % Разряженный текст \so{} и подчеркивание \ul{}
\usepackage{soulutf8}        % Поддержка UTF8 в soul

\usepackage{titlesec}        % Форматирование заголовков
\titleformat{\section}{\LARGE \bfseries}{\thesection}{1em}{}
\titleformat{\subsection}{\Large\bfseries}{\thesubsection}{1em}{}
\titleformat{\subsubsection}{\large\bfseries}{\thesubsubsection}{1em}{}

\newcommand{\R}{\mathbb R}
\newcommand{\Q}{\mathbb Q}
\newcommand{\Z}{\mathbb Z}
\newcommand{\N}{\mathbb N}
\newcommand{\CC}{\mathbb C}
\newcommand{\F}{\mathbb F}
\newcommand{\aug}{\fboxsep=-\fboxrule\!\!\!\fbox{\strut}\!\!\!}
\newcommand{\sgn}{\operatorname{sgn}}
\newcommand{\id}{\mathrm{id}}
\renewcommand{\phi}{\varphi}
\renewcommand{\epsilon}{\varepsilon}

\newsavebox{\boxedalignbox}
\newenvironment{boxedalign*}
  {\begin{equation*}\begin{lrbox}{\boxedalignbox}$\begin{aligned}}
  {\end{aligned}$\end{lrbox}\fbox{\usebox{\boxedalignbox}}\end{equation*}}

\newcommand\tab[1][.5cm]{\hspace*{#1}}

% Подписи для матриц
\newcommand\undermat[2]{\makebox[0pt][l]{$\smash{\underbrace
{\phantom{\begin{matrix}#2\end{matrix}}}_{\text{$#1$}}}$}#2}
\newcommand\overmat[2]{\makebox[0pt][l]{$\smash{\overbrace
{\phantom{\begin{matrix}#2\end{matrix}}}^{\text{$#1$}}}$}#2}

% Значек "пусть"
\newlength{\tempheight}  
\newcommand{\Let}[0]{  
\mathbin{\text{\settoheight{\tempheight}{\mathstrut}\raisebox{0.5\pgflinewidth}{%
\tikz[baseline,line cap=round,line join=round] \draw (0,0) --++ (0.4em,0) --++ (0,1.5ex) --++ (-0.4em,0);
}}}}


% \newcounter{lemcount}
% \newcounter{thcount}
% \newcounter{offercount}
% \newcounter{concount}
% \newcounter{subthcount}
% \newcounter{defcount}

\theoremstyle{definition}
\newtheorem*{definition}{Определение}
% \newtheorem{definitionnum}[defcount]{Определение}
\newtheorem*{example}{Примеры}
\newtheorem*{example1}{Пример}
\newtheorem*{exercise}{Упражнение}


\theoremstyle{plain}
\newtheorem*{theorem}{Теорема}
% \newtheorem{theoremnum}[thcount]{Теорема}
\newtheorem*{consequense}{Следствие}
\newtheorem*{consequenses}{Следствия}
% \newtheorem{consequensenum}[concount]{Следствие}
\newtheorem*{lemma}{Лемма}
% \newtheorem{lemmanum}[lemcount]{Лемма}
\newtheorem*{subtheorem}{Утверждение}
% \newtheorem{subtheoremnum}[subthcount]{Утверждение}
\newtheorem*{algorithm}{Алгоритм}
\newtheorem*{properties}{Свойства}
\newtheorem*{properties1}{Свойство}


\theoremstyle{remark}
\newtheorem*{remark}{Замечание}
\newtheorem*{offer}{Предложение}
% \newtheorem{offernum}[offercount]{Предложение}
\begin{document}
  \newpage
  \begin{definition}
    Линейное отображение $\phi$ : $V_1 \to V_2$ называется изоморфизмом, если $\phi$ линейно и биективно. $V_1$ и $V_2$ называются изоморфными, если существует изоморфизм $\phi$ : $V_1 \to V_2$. Обозначается: $V_1 \cong V_2$.
  \end{definition}
  \begin{theorem}(Об изоморфизме)
    Конечномерные векторные пространства $V_1$ и $V_2$ изоморфны тогда и только тогда, когда $dimV_1 = dimV_2$. 
  \end{theorem}
  \begin{proof}
    $dimV_1=dimV_2=n$. Пусть $e_1$, $\ldots$, $e_n$ - базис в $V_1$, а $f_1$, $\ldots$, $f_n$ - базис в $V_2$, тогда $\forall v \in V_1$ $v=\sum\limits_{i=1}^nx_ie_i$.\\ Определим отображение $\phi$ : $V_1 \to V_2$ формулой $\phi(v):=\sum\limits_{i=1}^nx_if_i$.
    \begin{enumerate}
      \item (линейность) Пусть $v_1$, $v_2$ $\in V_1$, $v_1=\sum\limits_{i=1}^nx_ie_i$ и $v_2=\sum\limits_{i=1}^ny_ie_i$, тогда\\ $v_1+v_2=\sum\limits_{i=1}^n(x_i+y_i)e_i$ $\Longrightarrow$\\ $\Longrightarrow$ $\phi(v_1+v_2) = \sum\limits_{i=1}^n(x_i+y_i)f_i=\sum\limits_{i=1}^nx_if_i+\sum\limits_{i=1}^ny_if_i=\phi(v_1)+\phi(v_2)$.\\ $\forall\lambda\in\F$ и $\forall v\in V_1$ $\phi(\lambda v)=\sum\limits_{i=1}^n(\lambda x_i)f_i =\lambda\sum\limits_{i=1}^nx_if_i=\lambda\phi(v)$.
      \item (инъективность) $Ker\phi = \{v\in V_1$ | $\phi(v)=0_{V_2}\}$. Пусть $v\in V_1$ и $v\in Ker\phi$, тогда $v=\sum\limits_{i=1}^n\alpha_ie_i$ $\Longrightarrow$\\$\Longrightarrow$ $\phi(v)=\sum\limits_{i=1}^n\alpha_if_i=0$, так как $f_1$, $\ldots$, $f_n$ - линейно независимы $\Longrightarrow$ $\forall i$ $\alpha_i=0$ $\Longrightarrow$ $v=\sum\limits_{i=1}^n\alpha_ie_i=0$ $\Longrightarrow$$Ker\phi=\{0\}$.
      \item (сюръективность) $\forall w\in V_2$ $w=\sum\limits_{j=1}^n\alpha_jf_j$ $\Longrightarrow$ $w=\phi(v)$, $v=\sum\limits_{j=1}^n\alpha_je_j$ $\Longrightarrow$ $\phi(V_1)=V_2$.
    \end{enumerate}
    $\underline{\Longrightarrow}$ Пусть $V_1\cong V_2$, $dimV_1=n$, $\phi$ : $V_1 \to V_2$ - изоморфизм. Выберем базис $e_1$, $\ldots$, $e_n$ в $V_1$ и покажем, что $\phi(e_1)$, $\ldots$, $\phi(e_n)$ - базис в $V_2$.\\
    $\forall w\in V_2$ $\exists v\in V_1$ : $\phi(v)=w$. Пусть $v=\sum\limits_{i=1}^nx_ie_i$, тогда $\phi(v)=w=\sum\limits_{i=1}^nx_i\phi(e_i)$ $\Longrightarrow$\\ $\Longrightarrow$ $V_2=\langle\phi(e_1),$ $\ldots$, $\phi(e_n)\rangle$. Проверим линейную независимость\\
    Предположим, что $\exists\mu_i\in\F$ : $0_{V_2}=\sum\limits_{i=1}^n\mu_i\phi(e_i)=\phi(\sum\limits_{i=1}^n\mu_ie_i)$ $\Longrightarrow$ $\sum\limits_{i=1}^n\mu_ie_i\in Ker\phi=\{0\}$, так как $\phi$ - биекция.\\ Так как $\{e_i\}$ - линейно независимы $\Longrightarrow$ $\mu_i=0$ $\forall i$ $\Longrightarrow$ $\phi(e_1)$, $\ldots$, $\phi(e_n)$ - линейно независимы.
  \end{proof}
\end{document}