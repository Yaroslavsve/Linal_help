\documentclass[a4paper, 12pt]{article}

\usepackage{import}

% Корректность отображения всех шрифтов, кодировок и мат. символов
\usepackage[T2A]{fontenc}
\usepackage[utf8]{inputenc}
\usepackage[english, russian]{babel}
\usepackage{amssymb, amsmath, amsthm, mathtools}

% Отображение содержания
\usepackage{tocloft}

% Вставка картинок
\usepackage{graphicx}
\usepackage{tikz}
\usepackage{tkz-euclide}
\usepackage{asymptote}

\usepackage{wrapfig}        % Огибание картинок текстом
\usepackage{cancel}         % Зачёркивания
\usepackage{indentfirst}    % Отступ у первого абзаца
\usepackage{xcolor}         % Цвета
\setlength{\parskip}{.5ex}  % Отступы между абзацами
\usepackage{enumitem}       % Работа со списками
% \usepackage{minted}       % Вставка блоков кода

\usepackage{hyperref}       % гиперссылки
\definecolor{linkcolor}{HTML}{225ae2} % Цвет ссылок
\definecolor{urlcolor}{HTML}{225ae2} % Цвет гиперссылок
\hypersetup{
    pdfstartview=FitH, 
    linkcolor=linkcolor,
    urlcolor=urlcolor,
    colorlinks=true}
\setlength{\arrayrulewidth}{0.5mm} %Толщина линейки в таблицах
\setlength{\tabcolsep}{18pt} %Разделение между столбцами в таблице

% Отступы на странице
\usepackage[left=2cm, right=1.5cm, top=2cm, bottom=2cm]{geometry}

\usepackage{cmap}            % Русский поиск в PDF документе
\usepackage{etoolbox}
\usepackage{soul}            % Разряженный текст \so{} и подчеркивание \ul{}
\usepackage{soulutf8}        % Поддержка UTF8 в soul

\usepackage{titlesec}        % Форматирование заголовков
\titleformat{\section}{\LARGE \bfseries}{\thesection}{1em}{}
\titleformat{\subsection}{\Large\bfseries}{\thesubsection}{1em}{}
\titleformat{\subsubsection}{\large\bfseries}{\thesubsubsection}{1em}{}

\newcommand{\R}{\mathbb R}
\newcommand{\Q}{\mathbb Q}
\newcommand{\Z}{\mathbb Z}
\newcommand{\N}{\mathbb N}
\newcommand{\CC}{\mathbb C}
\newcommand{\F}{\mathbb F}
\newcommand{\aug}{\fboxsep=-\fboxrule\!\!\!\fbox{\strut}\!\!\!}
\newcommand{\sgn}{\operatorname{sgn}}
\newcommand{\id}{\mathrm{id}}
\renewcommand{\phi}{\varphi}
\renewcommand{\epsilon}{\varepsilon}

\newsavebox{\boxedalignbox}
\newenvironment{boxedalign*}
  {\begin{equation*}\begin{lrbox}{\boxedalignbox}$\begin{aligned}}
  {\end{aligned}$\end{lrbox}\fbox{\usebox{\boxedalignbox}}\end{equation*}}

\newcommand\tab[1][.5cm]{\hspace*{#1}}

% Подписи для матриц
\newcommand\undermat[2]{\makebox[0pt][l]{$\smash{\underbrace
{\phantom{\begin{matrix}#2\end{matrix}}}_{\text{$#1$}}}$}#2}
\newcommand\overmat[2]{\makebox[0pt][l]{$\smash{\overbrace
{\phantom{\begin{matrix}#2\end{matrix}}}^{\text{$#1$}}}$}#2}

% Значек "пусть"
\newlength{\tempheight}  
\newcommand{\Let}[0]{  
\mathbin{\text{\settoheight{\tempheight}{\mathstrut}\raisebox{0.5\pgflinewidth}{%
\tikz[baseline,line cap=round,line join=round] \draw (0,0) --++ (0.4em,0) --++ (0,1.5ex) --++ (-0.4em,0);
}}}}


% \newcounter{lemcount}
% \newcounter{thcount}
% \newcounter{offercount}
% \newcounter{concount}
% \newcounter{subthcount}
% \newcounter{defcount}

\theoremstyle{definition}
\newtheorem*{definition}{Определение}
% \newtheorem{definitionnum}[defcount]{Определение}
\newtheorem*{example}{Примеры}
\newtheorem*{example1}{Пример}
\newtheorem*{exercise}{Упражнение}


\theoremstyle{plain}
\newtheorem*{theorem}{Теорема}
% \newtheorem{theoremnum}[thcount]{Теорема}
\newtheorem*{consequense}{Следствие}
\newtheorem*{consequenses}{Следствия}
% \newtheorem{consequensenum}[concount]{Следствие}
\newtheorem*{lemma}{Лемма}
% \newtheorem{lemmanum}[lemcount]{Лемма}
\newtheorem*{subtheorem}{Утверждение}
% \newtheorem{subtheoremnum}[subthcount]{Утверждение}
\newtheorem*{algorithm}{Алгоритм}
\newtheorem*{properties}{Свойства}
\newtheorem*{properties1}{Свойство}


\theoremstyle{remark}
\newtheorem*{remark}{Замечание}
\newtheorem*{offer}{Предложение}
% \newtheorem{offernum}[offercount]{Предложение}
\begin{document}
  \newpage
  Пусть $U$ инвариантное подпространство $V$ для линейного оператора $\phi: V \to V$.
  \begin{definition}
    Ограничением $\phi$ на подпространство $U$ называется отображение $\phi|\begin{matrix}
      \null \\ U
    \end{matrix}$: $U \to U$ такое, что $\forall u \in U \phi|\begin{matrix}
      \null \\ U
    \end{matrix}$($u$) = $\phi(u)$
  \end{definition}
  Рассмотрим фактор-пространство $\bar{V}$ = $V\begin{matrix}
    \null \\ U
  \end{matrix}$ : $\bar{v}$ = {$v$+$u$|$u \in U$}
  \begin{definition}
    Оператор $\bar{\phi}: \bar{V} \to \bar{V}$ называется фактор-оператором.
  \end{definition}
  $\forall v'$ = $v$+$u$, где $u \in U$, $\phi(v') = \phi(v)+\phi(u)$ $\Longrightarrow$ $\bar{\phi(v')} = \bar{\phi(v)}$ (так как $\phi(u) \in U$) $\Longrightarrow \bar{\phi}: \bar{V} \to \bar{V}$ - линейный оператор. 
  \begin{theorem}
    1. Если $\exists U \neq {0}$, $U$ - подпространство $V$, Im$\phi \subset U$, то в подходящем базисе $A_{\phi}$ = \begin{pmatrix}
      B & \vline & D \\
      \hline
      0 & \vline & C
    \end{pmatrix} (1),
    где $B_{m \times n}$ - матрица линейного оператора $\phi|\begin{matrix}
      \null \\ U
    \end{matrix}$, где $m$ = dim$U$, а $C$ - матрица оператора $\bar{\phi}$.
    2. Если $V$ = $U \oplus W$, где $U$ и $W$ - инвариантные подпространства относительно $\phi$, то в подходящем базисе $A_{\phi}$ = \begin{pmatrix}
      B & \vline & 0 \\
      \hline
      0 & \vline & C
    \end{pmatrix} (2),
    где $B$ = $A_{\phi|\begin{matrix}
      \null \\ U
    \end{matrix}}$, $C$ = $A_{\phi|\begin{matrix}
      \null \\ W
    \end{matrix}}$.
    3. Верны и обратные утверждения: если в некотором базисе $A_{\phi}$ имеет вид (1), то для $\phi$ существует инвариантое подпространство, а если $A_{\phi}$ имеет вид (2), то $V$ - прямая сумма двух инвариантных подпространств.  
  \end{theorem}
  \begin{proof}
    1. Обозначим dim$V$ = n, dim$U$ = m, 0 < m < n. Выберем базис в $U$ $e_1, \ldots, e_m$ и дополним его до базиса в $V$ произвольными векторами $e_{m+1}, \ldots, e_n$.
    Тогда $\forall u\in U$ $u$ = $\sum \limits_{i=1}^m u_i e_i \Longrightarrow \phi(u) = \sum \limits_{i=1}^m u_i \phi(e_i)$
    В частности, столбцы $\phi(e_1) \ldots \phi(e_m)$ имеют вид: \begin{pmatrix}
      a_{1i}\\
      \vdots\\
      a_{mi}\\
      0\\
      \vdots\\
      0
    \end{pmatrix} $\Longrightarrow$ они составляют матрицу \begin{pmatrix}
      B\\
      \hline
      0
    \end{pmatrix}
    Разбивка матрицы, составленной из столбцов образов базисных векторов $e_{m+1}, \ldots, e_n$, 
    Видно, что $B$ = \begin{pmatrix}
      a_{11} & \cdots & a_{1m}\\
      \vdots\\
      a_{m1} & \cdots & a_{mm}
    \end{pmatrix} = $A_{\phi|\begin{matrix}
      \null \\ U
    \end{matrix}}$.
    2. Если $V$ = $U \oplus W$, векторы $e_{m+1}, \ldots, e_n$ надо выбирать в $W$, а остольное аналогично предыдущему пункту.
    3. В обратную сторону для второго случая: если в базисе $e_1, \ldots, e_n$ матрица имеет вид (2), то положим в качестве $U$ = $\langle e_1, \ldots, e_m \rangle$, а $W$ = $\langle_{m+1}, \ldots, e_n \rangle$
    Из определения матрицы $A_{\phi, e}$ следует, что $U$ и $W$ - инварианты относительно $\phi$, $\phi|\begin{matrix}
      \null \\ U
    \end{matrix}$ имеет матрицу $B$, а $\phi$, $\phi|\begin{matrix}
      \null \\ W
    \end{matrix}$ имеет матрицу $C$.
    Для первого случая: $\bar{e_j} = e_j$ + $U$, для $\bar{m+1, n}$, является базисом в фактор-пространстве $\bar{V}$ = $V|\begin{matrix}
      \null \\ U
    \end{matrix} \bar{\phi}(\bar{e_j})$ = $\phi(e_j) = \bar{\sum \limits_{i=1}^m a_{ij} e_i + \sum \limits_{k=m+1}^n a_{kj} e_k} = \sum \limits_{k=m+1}^n e_{kj} \bar{e_k}$ (так как первая сумма $\in U$ )
    $\Longrightarrow C = \begin{pmatrix}
      a_{m+1, m+1} & \cdots & a_{n+1, n}\\
      \vdots\\
      a_{n, m+1} & \cdots & a_{nn}
    \end{pmatrix}$ - матрица оператора $\bar{\phi}$.
  \end{proof}
  \begin{remark}
    В общем случае, если $V = U_1 \oplus \ldots \oplus U_s$, то в некотором базисе, согласно разложению, $A_{phi}$ = \begin{pmatrix}
      B_1\\
      \ddots\\
          B_s
    \end{pmatrix}, где $B_i$ - матрица $\phi|\begin{matrix}
      \null \\ U_i
    \end{matrix} \forall i = \bar{1, s}$ 
  \end{remark}
  \begin{example1}(Естественные примеры инвариантных подпространств (доказательство - упражнение))
    $\phi: V \to V$ - линейный оператор.
    1. Ker$\phi$, Im$\phi$ и любое подпространство $U$ : Im$\phi \subset U$, тогда $U$ является инвариантным подпространством относительно $\phi$.
    2. Если $U_1$ и $U_2$ являются инвариантными подпространствами относительно оператора $\phi$, то $U_1+U_2$ и $U_1 \cap U_2$ также являются инвариантными относительно оператора $\phi$.
  \end{example1}
  \subsection{Действия над линейными отображениями и операторами}
  Пусть $\phi: V_1 \to V_2$ - линейное отображение, тогда:
  1. $\forall \lambda \in \F (\lambda\phi)(x) = \lambda\phi(x)$, $\forall x \in V_1$
  2. Если $\psi: V_1 \to V_2$, то ($\phi+\psi)(x) = \phi(x)+\psi(x)$, $\forall x \in V_1$
  \begin{subtheorem} 1
    Относительно этих операций множество $L(V_1, V_2)$ линейных отображений из $V_1$ в $V_2$ является векторным пространством.
  \end{subtheorem}
  \begin{subtheorem} 2
    Если dim$V_1 = n$, dim$V_2 = m$, то $L(V_1, V_2) \cong M_{m \times n}(\F)$
  \end{subtheorem}
  \begin{proof}
    Зафиксируем базисы в $V_1$ и $V_2$ $e$ и $f$ соответственно, тогда $\forall \phi$ взаимооднозначно соответствует его матрица $A_{\phi, e, f}$ относительно базисов $e$ и $f$.
    $A_{\lambda \phi} = \lambda A_{\phi}$  $\forall \lambda \in \F$
    $(\lambda \phi)(e_j) = \lambda \phi(e_j) \Longrightarrow$ все столбцы $A_{\phi}$ умножаются на $\lambda \Longrightarrow A_{\phi}$ умножается на $\lambda$.
    $\forall j$ = $\bar{1, m} (\phi + \psi)(e_j) = \phi(e_j) + \psi(e_j) \Longrightarrow$ столбцы $A_{\phi + \psi}$ имеют вид $\phi(e_j) + \psi(e_j)$.
  \end{proof}
  Обозначение: $L(V_1, V_2) = \kappa(V_1, V_2) =$ Hom$(V_1, V_2)$.
  $\kappa(V)$ - множество линейных операторов на $V$.
  \begin{definition}
    Произведением линейных операторов $\phi: V_1 \to V_2$ и $\psi: V_1 \to V_2$ называется их композиция $(\phi\circ\psi)(x) = \psi(\phi(x))$, где $x \in V_1$.
  \end{definition}
  \begin{subtheorem} 3
    Композиция линейных отображений является линейным отображением, а композиция линейных операторов - линейным оператором.
  \end{subtheorem}
  \begin{subtheorem} 4
    Пусть $V_1, V_2, V_3$ - конечномерные векторные пространства, а $\psi: V_1 \to V_2$ и $\phi: V_2 \to V_3$ - линейные отображения, тогда, если зафиксировать базисы в этих пространствах, матрица композиции $A_{\psi\circ\phi} = A_{\psi} A_{\phi}$.
  \end{subtheorem}
  \begin{proof}
    Утверждение 3 - упражнение.
    Утверждение 4:
    Пусть $e$ - базис в $V_1$, $f$ - базис в $V_2$, $g$ - базис в $V_3$.
    $A_{\phi} = (\phi(e_1)\uparrow \ldots \phi(e_n)\uparrow)$ в базисе $f$, $A_{\psi} = (\psi(f_1)\uparrow \dots \psi(f_m)\uparrow)$ в базисе $g$.    $\forall x = e X$, обозначим $y = \phi(x)$, $z = \psi(y)$ со столбцами координат $Y$ и $Z$ соответственно.
    Тогда $Y = A_{\phi}X$, $Z = A_{\psi}Y = A_{\psi}(A_{\phi}X) = (A_{\psi}A_{\phi})X = A_{\psi\circ\phi}X$.
  \end{proof}
  \begin{theorem}
    Множество $\kappa(V)$ с операциями $+$, $\cdot\lambda $, $\cdot$ является ассоциативной алгеброй с единицей, равной Id$V$.
    Если dim$V = n$, то $\kappa(V) \cong M_{n}(\F)$.
  \end{theorem}
  \begin{proof}
    Следует из утверждений 1 - 4.
  \end{proof}
  \begin{subtheorem}
    Если $\phi$ - линейный оператор на $V$, то $\forall k \in \N$ подпространства Ker$\phi^k$ и Im$\phi^k$ - инварианты.
    При этом {$0} \equiv Ker\phi \equiv Ker\phi^2 \equiv\ldots$
    $V \supseteq Im\phi \supseteq Im\phi^2\ldots$
  \end{subtheorem}
  \subsection{Собственные векторы и собственные значения оператора}
  Пусть $\phi: V \to V$ - линейный оператор над полем $\F$.
  \begin{definition}
    Вектор $x \in V$ называется собственным вектором оператора $\phi$, если $\exists\lambda\in \F: \phi(x) = \lambda \cdot x$ и $x\neq0$. $\lambda$ называется собственным значением оператора $\phi$, соответствующим вектору $x$.
  \end{definition}
  Пусть dim$V = n$, $e$ - базис в $V$, в нём $\forall x = e\cdot X$, тогда равенство из вышеуказанного определения равносильно $A_{\phi}X = \lambda X \Longleftrightarrow (A_{\phi} - \lambda E)X = 0$ (2) - это СЛУ для нахождения вектора $x$, если известна $\lambda$.
  Система (2) имеет ненулевое решение, только если det$(A_{\phi} - \lambda E) = 0$ (3).
  Равенство (3) называется характеристическим уравненением.
  Собственными значениями могут быть только корни характеристического уравнения.
  \begin{example1}
    Пример 1.
    $V = D^{\infty}(\R)$ - множество бесконечно дифференцируемых функций.
    $\phi \frac{d}{dx} \forall f(x) \phi(f) = f'(x)$.
    $\forall\lambda\in\R (e^{\lambda x})' = \lambda^x$.
    \begin{proof}
      Если $f'(x) = \lambda \cdot f(x)$, то $f(x) = C \cdot e^{\lambda x}$, где $C\neq0$.
      Рассмотрим $(f(x)e^{-\lambda x})' = f'(x)e^{-\lambda x} - \lambda f(x)e^{-\lambda x} = 0 \Longrightarrow f(x)e^{-\lambda x} = C$.
    \end{proof}
    Пример 2.
    $A_{\phi} = \begin{matrix}
      cos\phi & -sin\phi
      sin\phi & cos\phi
    \end{matrix}$.
    \begin{exercise}
      Какие существуют собственные векторы и собственные значения у $\phi$?
    \end{exercise}
  \end{example1}
\end{document}