\documentclass[a4paper, 12pt]{article}

\usepackage{import}

% Корректность отображения всех шрифтов, кодировок и мат. символов
\usepackage[T2A]{fontenc}
\usepackage[utf8]{inputenc}
\usepackage[english, russian]{babel}
\usepackage{amssymb, amsmath, amsthm, mathtools}

% Отображение содержания
\usepackage{tocloft}

% Вставка картинок
\usepackage{graphicx}
\usepackage{tikz}
\usepackage{tkz-euclide}
\usepackage{asymptote}

\usepackage{wrapfig}        % Огибание картинок текстом
\usepackage{cancel}         % Зачёркивания
\usepackage{indentfirst}    % Отступ у первого абзаца
\usepackage{xcolor}         % Цвета
\setlength{\parskip}{.5ex}  % Отступы между абзацами
\usepackage{enumitem}       % Работа со списками
% \usepackage{minted}       % Вставка блоков кода

\usepackage{hyperref}       % гиперссылки
\definecolor{linkcolor}{HTML}{225ae2} % Цвет ссылок
\definecolor{urlcolor}{HTML}{225ae2} % Цвет гиперссылок
\hypersetup{
    pdfstartview=FitH, 
    linkcolor=linkcolor,
    urlcolor=urlcolor,
    colorlinks=true}
\setlength{\arrayrulewidth}{0.5mm} %Толщина линейки в таблицах
\setlength{\tabcolsep}{18pt} %Разделение между столбцами в таблице

% Отступы на странице
\usepackage[left=2cm, right=1.5cm, top=2cm, bottom=2cm]{geometry}

\usepackage{cmap}            % Русский поиск в PDF документе
\usepackage{etoolbox}
\usepackage{soul}            % Разряженный текст \so{} и подчеркивание \ul{}
\usepackage{soulutf8}        % Поддержка UTF8 в soul

\usepackage{titlesec}        % Форматирование заголовков
\titleformat{\section}{\LARGE \bfseries}{\thesection}{1em}{}
\titleformat{\subsection}{\Large\bfseries}{\thesubsection}{1em}{}
\titleformat{\subsubsection}{\large\bfseries}{\thesubsubsection}{1em}{}

\newcommand{\R}{\mathbb R}
\newcommand{\Q}{\mathbb Q}
\newcommand{\Z}{\mathbb Z}
\newcommand{\N}{\mathbb N}
\newcommand{\C}{\mathbb C}
\newcommand{\F}{\mathbb F}
\newcommand{\aug}{\fboxsep=-\fboxrule\!\!\!\fbox{\strut}\!\!\!}
\newcommand{\sgn}{\operatorname{sgn}}
\newcommand{\id}{\mathrm{id}}
\renewcommand{\phi}{\varphi}
\renewcommand{\epsilon}{\varepsilon}

\newsavebox{\boxedalignbox}
\newenvironment{boxedalign*}
  {\begin{equation*}\begin{lrbox}{\boxedalignbox}$\begin{aligned}}
  {\end{aligned}$\end{lrbox}\fbox{\usebox{\boxedalignbox}}\end{equation*}}

\newcommand\tab[1][.5cm]{\hspace*{#1}}

% Подписи для матриц
\newcommand\undermat[2]{\makebox[0pt][l]{$\smash{\underbrace
{\phantom{\begin{matrix}#2\end{matrix}}}_{\text{$#1$}}}$}#2}
\newcommand\overmat[2]{\makebox[0pt][l]{$\smash{\overbrace
{\phantom{\begin{matrix}#2\end{matrix}}}^{\text{$#1$}}}$}#2}

% Значек "пусть"
\newlength{\tempheight}  
\newcommand{\Let}[0]{  
\mathbin{\text{\settoheight{\tempheight}{\mathstrut}\raisebox{0.5\pgflinewidth}{%
\tikz[baseline,line cap=round,line join=round] \draw (0,0) --++ (0.4em,0) --++ (0,1.5ex) --++ (-0.4em,0);
}}}}


% \newcounter{lemcount}
% \newcounter{thcount}
% \newcounter{offercount}
% \newcounter{concount}
% \newcounter{subthcount}
% \newcounter{defcount}

\theoremstyle{definition}
\newtheorem*{definition}{Определение}
% \newtheorem{definitionnum}[defcount]{Определение}
\newtheorem*{example}{Примеры}
\newtheorem*{example1}{Пример}
\newtheorem*{exercise}{Упражнение}


\theoremstyle{plain}
\newtheorem*{theorem}{Теорема}
% \newtheorem{theoremnum}[thcount]{Теорема}
\newtheorem*{consequense}{Следствие}
\newtheorem*{consequenses}{Следствия}
% \newtheorem{consequensenum}[concount]{Следствие}
\newtheorem*{lemma}{Лемма}
% \newtheorem{lemmanum}[lemcount]{Лемма}
\newtheorem*{subtheorem}{Утверждение}
% \newtheorem{subtheoremnum}[subthcount]{Утверждение}
\newtheorem*{algorithm}{Алгоритм}
\newtheorem*{properties}{Свойства}
\newtheorem*{properties1}{Свойство}


\theoremstyle{remark}
\newtheorem*{remark}{Замечание}
\newtheorem*{offer}{Предложение}
% \newtheorem{offernum}[offercount]{Предложение}
\begin{document}
  \newpage
  Существование двумерного инварантного подпространства для линейного оператора над $\R$, отвечающего мнимому корню характеристического многочлена.$\\$
  Пусть $\phi: V \to V$ - линейный оператор, dim$V = n$, тогда в некотором базисе $V$ $\phi$ действует ма матрицей $Y = A_{\phi}X$, где $X\in \R^n$, а $Y$ - столбец образа этого вектора. Пусть $\lambda = \alpha+i\beta$ ($\beta\neq0$) - корень характеристического многочлена.\\
  Рассмотрим линейный оператор над полем $\mathbb{C}$, действующий при той же матрице $A_{\phi} : \forall Z\in \mathbb{C}^n Z \to A_{\phi}Z$, соответствующий оператор будем обозначать той же буквой. Так как $\mathbb{C}$ алгебраически замкнуто, то $\exists$ собственный вектор $Z_0$, отвечающий выбранному $\lambda$. Это значит, что $A_{\phi}Z_{0} = \lambda Z_{0}$, $Z_{0} = X_{0}+iY_{0}$, где $X_0$ и $Y_0 \in \R^n$ $\Longrightarrow$ $A_{\phi}Z_0 = A_{\phi}X_0+iA_{\phi}Y_0 = (\alpha+i\beta)(X_0 + iY_0) = (\alpha X_0 - \beta Y_0)+i(\beta X_0 + \alpha Y_0) \Longrightarrow$ \\
  $\begin{cases}
    A_{\phi}X_0 = \alpha X_0-\beta Y_0\\
    A_{\phi}Y_0 = \beta X_0+\alpha Y_0
  \end{cases}$
  Обозначим $x_0$ и $y_0 \in V$ векторы со столбцами координат $X_0$ и $Y_0$ соответственно, тогда $\\$
  $\begin{cases}
    \phi(x_0) = \alpha x_0-\beta y_0 \\
    \phi(y_0) = \beta x_0+\alpha y_0
  \end{cases}$ $\longrightarrow$ подпространство $U = \langle x_0, y_0\rangle\subset V$ является инвариантным подпространством для $\phi$. \\
  Теперь докажем, что dim$U = 2$. $\\$
  \begin{proof}
    Предположим, что dim$U = 1$, то есть $y_0 = \mu x_0$, где $\mu\in\R$. Тогда $\phi(x_0) = (\alpha-\beta\mu)x_0$ $\Longrightarrow$ если $x_0\neq0$, то $x_0$ - собственный вектор для $\phi$ (для $y_0$ аналогично). Но эти векторы не были собственными для $\phi$. \\
    $A_{\phi\begin{matrix}
        \null \\ U
    \end{matrix}} = \begin{pmatrix}
      \alpha && \beta\\
      -\beta && \alpha
    \end{pmatrix}$ имеет корни $\alpha\pm i\beta\notin\R$ - противоречие.
  \end{proof}
  \begin{theorem}
    Любой линейный оператор в конечномерном вещественном векторным пространстве имеет одномерное или двумерное подпространство.
  \end{theorem}
  \begin{proof}
    Если $\exists\lambda\in\R$ - корень характерического многочлена, ему отвечает собственный вектор $u_i\in V$, $u_i\neq0$, $\Longrightarrow$ $\langle u_i \rangle$ - одномерное инвариантное подпространство. $\\$
    Если $\forall\lambda \in \mathbb{C}\setminus\R$, то $\exists U$ - двумерное инвариантное подпространство.
  \end{proof}
  Вместо диаганализируемости можно использовать следующее утверждение:$\\$
  $A_{\phi} = \begin{pmatrix}
    \lambda_1\\
    \null && \ddots\\
    \null && \null && \lambda_n\\
    \null && \null && \null && \alpha_1 && \beta_1\\
    \null && \null && \null && -\beta_1 && \alpha_1\\
    \null && \null && \null && \null && \null && \ddots\\
    \null && \null && \null && \null && \null && \null && \alpha_m && \beta_m\\
    \null && && \null && \null && \null && \null && -\beta_m && \alpha_m
  \end{pmatrix}$, где $\lambda_i\in\R$, $i = \overline{1,n}$, а $\beta_j\neq0$, $j = \overline{1,m}$
  \subsection{Анулирующие многочлены линейных операторов}
  Пусть $\phi: V\to V$ - линейный оператор над полем $\F$. $\\$
  \begin{definition}
    Линейный оператор $\phi: V\to V$ такой, что $\phi(v) = v$ $\forall v\in V$, называется тождественным оператором и обозначается Id.
  \end{definition}
  \begin{definition}
    Многочлен $f(t) = a_0+a_1t+\ldots+a_mt^m\in\F[t]$, где $a_1\ldots a_m\in\F$, называется анулирующим многочленом оператора $\phi$, если $f(\phi) = a_0Id+a_1\phi+\ldots+a_m\phi^m = 0$, то есть $f(A_{\phi}) = 0 \\
    \Longrightarrow A_{f(\phi)} = f\cdot A_{\phi} = a_0E+a_1A_{\phi}+\ldots+a_mA_{\phi}^m$.
  \end{definition}
  \begin{example}
    $V = \R[t]_n$, $\phi = \frac{d}{dt}. \\
    \phi^n(t^n) = n!$, $\phi^{n+1}\equiv0 \Longrightarrow$ для $\phi = \frac{d}{dt}$ $t^{n+1}$ - анулирующий многочлен.
  \end{example}
  \begin{subtheorem}
    Если dim$V = n$, то $\exists$ многочлен степени $\leq n^2$, анулирующий $\phi$. dim$L(V) = n^2$, $L(V) \cong M_n(\F) \Longrightarrow$ операторы 
  \end{subtheorem}
\end{document} 