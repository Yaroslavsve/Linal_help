\documentclass[a4paper, 12pt]{article}

\usepackage{import}

% Корректность отображения всех шрифтов, кодировок и мат. символов
\usepackage[T2A]{fontenc}
\usepackage[utf8]{inputenc}
\usepackage[english, russian]{babel}
\usepackage{amssymb, amsmath, amsthm, mathtools}

% Отображение содержания
\usepackage{tocloft}

% Вставка картинок
\usepackage{graphicx}
\usepackage{tikz}
\usepackage{tkz-euclide}
\usepackage{asymptote}

\usepackage{wrapfig}        % Огибание картинок текстом
\usepackage{cancel}         % Зачёркивания
\usepackage{indentfirst}    % Отступ у первого абзаца
\usepackage{xcolor}         % Цвета
\setlength{\parskip}{.5ex}  % Отступы между абзацами
\usepackage{enumitem}       % Работа со списками
% \usepackage{minted}       % Вставка блоков кода

\usepackage{hyperref}       % гиперссылки
\definecolor{linkcolor}{HTML}{225ae2} % Цвет ссылок
\definecolor{urlcolor}{HTML}{225ae2} % Цвет гиперссылок
\hypersetup{
    pdfstartview=FitH, 
    linkcolor=linkcolor,
    urlcolor=urlcolor,
    colorlinks=true}
\setlength{\arrayrulewidth}{0.5mm} %Толщина линейки в таблицах
\setlength{\tabcolsep}{18pt} %Разделение между столбцами в таблице

% Отступы на странице
\usepackage[left=2cm, right=1.5cm, top=2cm, bottom=2cm]{geometry}

\usepackage{cmap}            % Русский поиск в PDF документе
\usepackage{etoolbox}
\usepackage{soul}            % Разряженный текст \so{} и подчеркивание \ul{}
\usepackage{soulutf8}        % Поддержка UTF8 в soul

\usepackage{titlesec}        % Форматирование заголовков
\titleformat{\section}{\LARGE \bfseries}{\thesection}{1em}{}
\titleformat{\subsection}{\Large\bfseries}{\thesubsection}{1em}{}
\titleformat{\subsubsection}{\large\bfseries}{\thesubsubsection}{1em}{}

\newcommand{\R}{\mathbb R}
\newcommand{\Q}{\mathbb Q}
\newcommand{\Z}{\mathbb Z}
\newcommand{\N}{\mathbb N}
\newcommand{\CC}{\mathbb C}
\newcommand{\F}{\mathbb F}
\newcommand{\aug}{\fboxsep=-\fboxrule\!\!\!\fbox{\strut}\!\!\!}
\newcommand{\sgn}{\operatorname{sgn}}
\newcommand{\id}{\mathrm{id}}
\renewcommand{\phi}{\varphi}
\renewcommand{\epsilon}{\varepsilon}

\newsavebox{\boxedalignbox}
\newenvironment{boxedalign*}
  {\begin{equation*}\begin{lrbox}{\boxedalignbox}$\begin{aligned}}
  {\end{aligned}$\end{lrbox}\fbox{\usebox{\boxedalignbox}}\end{equation*}}

\newcommand\tab[1][.5cm]{\hspace*{#1}}

% Подписи для матриц
\newcommand\undermat[2]{\makebox[0pt][l]{$\smash{\underbrace
{\phantom{\begin{matrix}#2\end{matrix}}}_{\text{$#1$}}}$}#2}
\newcommand\overmat[2]{\makebox[0pt][l]{$\smash{\overbrace
{\phantom{\begin{matrix}#2\end{matrix}}}^{\text{$#1$}}}$}#2}

% Значек "пусть"
\newlength{\tempheight}  
\newcommand{\Let}[0]{  
\mathbin{\text{\settoheight{\tempheight}{\mathstrut}\raisebox{0.5\pgflinewidth}{%
\tikz[baseline,line cap=round,line join=round] \draw (0,0) --++ (0.4em,0) --++ (0,1.5ex) --++ (-0.4em,0);
}}}}


% \newcounter{lemcount}
% \newcounter{thcount}
% \newcounter{offercount}
% \newcounter{concount}
% \newcounter{subthcount}
% \newcounter{defcount}

\theoremstyle{definition}
\newtheorem*{definition}{Определение}
% \newtheorem{definitionnum}[defcount]{Определение}
\newtheorem*{example}{Примеры}
\newtheorem*{example1}{Пример}
\newtheorem*{exercise}{Упражнение}


\theoremstyle{plain}
\newtheorem*{theorem}{Теорема}
% \newtheorem{theoremnum}[thcount]{Теорема}
\newtheorem*{consequense}{Следствие}
\newtheorem*{consequenses}{Следствия}
% \newtheorem{consequensenum}[concount]{Следствие}
\newtheorem*{lemma}{Лемма}
% \newtheorem{lemmanum}[lemcount]{Лемма}
\newtheorem*{subtheorem}{Утверждение}
% \newtheorem{subtheoremnum}[subthcount]{Утверждение}
\newtheorem*{algorithm}{Алгоритм}
\newtheorem*{properties}{Свойства}
\newtheorem*{properties1}{Свойство}


\theoremstyle{remark}
\newtheorem*{remark}{Замечание}
\newtheorem*{offer}{Предложение}
% \newtheorem{offernum}[offercount]{Предложение}
\begin{document}
  \newpage
  \begin{center}
    Билет 1
  \end{center}
  \begin{definition}
    Функции $f$ и $g$ называются равными, если $D_f=D_g$ и $\forall x\in D_f$ $f(x)=g(x)$.
  \end{definition}
  \begin{definition}
    Упорядоченный набор - функция, которая ставит в соответствие каждому элементому множества $\{1, \ldots, n\}$ элемент из множества $\{a_1,\ldots,a_n\}$ : $1\rightarrow a_1$, $\ldots$, $n\rightarrow a_n$.
  \end{definition}
  Декартовое произведение множеств $A_1\times\ldots\times A_n=(a_1, \ldots, a_n)$ : $a_i\in A_i$.
  \begin{definition}
    Пусть функция $f$ определена на $A_1\times\ldots\times A_n$, тогда $f$ - $n$-местная функция.
  \end{definition}
  \begin{definition}
    Множество $B_n=E_2\times\ldots\times E_n$, где $E_i=\{0,1\}$, называется $n$-мерным булевым кубом.
  \end{definition}
  \begin{definition}
    Функция $f:B_n \to E_2$ называется функцией алгебры логики. Множество всех таких функций обозначим $P_2$.
  \end{definition}
  Представление функции $f(x_1, \ldots, x_n)$ в виде таблицы, имеющей $n+1$ столбец:\\
  $\begin{matrix}
    x_1 & \ldots & x_{n-1} & x_n & f\\
    0 & \ldots & 0 & 0 & 0\\
    0 & \ldots & 0 & 0 & 1\\
    0 & \ldots & 0 & 1 & 0\\
    \vdots & \null & \vdots & \vdots & \vdots\\
    1 & \ldots & 1 & 1 & 1 
  \end{matrix}$\\
  Так как число различных первых $n$ столбцов $2^n$, так как в каждой ячейке одного столбца может быть либо 0, либо 1. $\Longrightarrow$ число функций будет $2^{2^n}$, так как для каждого набора значение функции может быть либо 0, либо 1.
  \begin{definition}
    Переменная $x_i$ называется существенной, если существуют наборы $\alpha_1$, $\ldots$, $\alpha_{i-1}$, $1$, $\alpha_{i+1}$, $\ldots$, $\alpha_n$ и $\alpha_1$, $\ldots$, $\alpha_{i-1}$, $0$, $\alpha_{i+1}$, $\ldots$, $\alpha_n$, на которых функция принимает различные значения. В противном случае переменная $x_i$ называется несущественной (фиктивной).
  \end{definition}
  \begin{definition}
    Пусть $x_i$ - фиктивная переменная, тогда если функция $f(x_1$, $\ldots$, $x_{i-1}$, $x_{i+1}$, $\ldots$, $x_n)=g(x_1$, $\ldots$, $x_{i-1}$, $0$, $x_{i+1}$, $\ldots$, $x_n)$, то функция $g$ называется полученной из $f$ добавлением фиктивной переменной. Функция удаления фиктивной переменной определяется аналогично.
  \end{definition}
  \begin{definition}
    Функция называется симметрической, если при любых перестановках переменных $x_{i_1}, \ldots, x_{i_n}$ значение функции не меняется.
  \end{definition}
  Элементарные функции в алгебре логики:
  \begin{enumerate}
    \item константы 0, 1
    \item тождественный $x$
    \item отрицание $\overline{x}$
    \item конъюнкция $x\wedge y$
    \item дизъюнкция $x\vee y$
    \item имплекация $x\rightarrow y$
    \item штрих Шеффера $x|y$
    \item стрелка Пирса $x\downarrow y$
    \item сложение по модулю 2
    \item эквивалентность
  \end{enumerate}
  \begin{center}
    Билет 2
  \end{center}
  \begin{definition}
    Формула - слово в некотором алфавите $A$.
  \end{definition}
  \begin{definition}
    Алфавит - конечное или бесконечное множество.
  \end{definition}
  \begin{definition}
    Произвольная функция, определённая на начальном отрезке натурального ряда и принимающая на нём значения из $A$.
  \end{definition}
  \begin{definition}
    Пусть $F$ - множество функций алгебры логики, $S$ - множество символов, обозначающих функции из $F$, тогда отображение $\Sigma : S \to F$ - сигнатура для $F$.
  \end{definition}
  \begin{definition}
    Пусть $X=\{x_1, \ldots\}$ - символы переменных.\\
    База индукция: если $x_i$ - символ переменной, то однобуквенное слово, состоящее из $x_i$ - формула в сигнатуре $\Sigma$.\\
    Пусть $s\in S$, $f=\Sigma(s)$ - функция от $n$ переменных, $F_1, \ldots, F_n$ - формулы в сигнатуре $\Sigma$, тогда слово $s(F_1, \ldots, F_n)$ - формула в сигнатуре $\Sigma$.  
  \end{definition}
  \begin{definition}
    Пусть $F$ - формула, $\tilde{x}$ - упорядоченный набор ($x_{i_1}, \ldots, x_{i_n})$, содержащий все формулы $F$, $\tilde{\alpha}=(\alpha_1, \ldots, \alpha_n)$ - двоичный набор.\\
    База индукции: $F$ - однобуквенное слово $x_{i_j}$, тогда $F[\tilde{x}, \tilde{\alpha}]=\alpha_j$ - значение фолмулы на ноборе $\tilde{\alpha}$.\\
    Пусть $F$ - $s(F_1, \ldots, F_n)$, $f=\Sigma(s)$, причём $F_1[\tilde{x}, \tilde{\alpha}]=\beta_1$, $\ldots$, $F_n[\tilde{x}, \tilde{\alpha}]=\beta_n$, тогда $f(\beta_1, \ldots, \beta_n)$ - значение функции на наборе значений переменных.
  \end{definition}
  \begin{definition}
    Формула, определяющая функцию алгебры логики, определённую на $B_n$ такую, что $\forall$ набора $\tilde{\alpha}=(\alpha_1$, $\ldots$, $\alpha_n)\in B_n$ $f(\tilde{\alpha})=F[\tilde{x}, \tilde{\alpha}]$.
  \end{definition}
  \begin{definition}
    Формулы в сигнатуре, представляющие собой переменные, называются невырожденными, остальные - вырожденными. Если функция определяется вырожденной формулой в сигнатуре $\Sigma : S \to F$, то она получена суперпозициями над $F$, где $F$ - множество функций.
  \end{definition}
  \begin{definition}
    (Другое определение суперпозиции) Если одну функцию можно получить с помощью конечного числа применений следующих трёх операций, то данная функция называется функцией, полученной суперпозициями над $F$.\\
    Операции:\\
    \begin{enumerate}
      \item Операция подстановки переменных. Пусть $f(x_1$, $\ldots$, $x_n)\in P_2$, $g(x_1$, $\ldots$, $x_n)$ - функция, определённая на $B_n$ такая, что $g(x_1$, $\ldots$, $x_n)=f(x_{i_1}$, $\ldots$, $x_{i_n})$, где набор ($i_1$, $\ldots$, $i_n$) - набор элементов ($1$, $\ldots$, $n$) (они необязательно различны). Тогда $g$ получена из $f$ операцией подстановки переменных.
      \item Операция подстановки функции в функцию. Пусть $f(x_1$, $\ldots$, $x_n)$, $g(x_1$, $\ldots$, $x_m)$, $h$ определена на $B_{n+m-1}$ и $h(x_1$, $\ldots$, $x_{n+m-1})=f(x_1$, $\ldots$, $x_{n-1}$, $g(x_n$, $\ldots$, $x_{n+m-1}))$, тогда функция $h$ получена из функций $f$ и $g$ операцией подстановки одной функции в другую.
      \item Операция добавления или удаления фиктивных переменных. Пусть $x_i$ - фиктивная переменная, тогда если функция $f(x_1$, $\ldots$, $x_{i-1}$, $x_{i+1}$, $\ldots$, $x_n)=g(x_1$, $\ldots$, $x_{i-1}$, $0$, $x_{i+1}$, $\ldots$, $x_n)$, то функция $g$ называется полученной из $f$ добавлением фиктивной переменной. Функция удаления фиктивной переменной определяется аналогично.
    \end{enumerate}
  \end{definition}
  \begin{center}
    Билет 3
  \end{center}
  \begin{definition}
    Формулы $F_1$ и $F_2$ называются эквивалентными, если они определяют равные функции относительно объединения их переменных. Функции называются равными, если их области определения равны и $\forall x\in D_f(x)$ $f(x)=g(x)$. Слово $F_1=F_2$, если формулы $F_1$ и $F_2$ эквивалентны, называется тождеством. 
  \end{definition}
  Основные тождества:
  \begin{enumerate}
    \item Ассоциативность операций: $\wedge$, $\vee$, $\neg$, $\leftrightarrow$.
    \item Дистрибутивности:
    \begin{enumerate}
      \item $(x\vee y)\wedge z=(x\wedge z)\vee(y\wedge z)$
      \item $(x\wedge y)\vee z=(x\vee z)\wedge(y\vee z)$
      \item $(x+y)\cdot z=x\cdot z+y\cdot z$
    \end{enumerate}
    \item Тождества для отрицания: 
    \begin{enumerate}
      \item $\overline{\overline{x}}=x$
      \item $\overline{x\wedge y}=\overline{x}\vee \overline{y}$
      \item $\overline{x\vee y}=\overline{x}\wedge \overline{y}$
      \item $x\cdot\overline{x}=0$
      \item $x\vee\overline{x}=1$
      \item $\overline{x\rightarrow y}=x\cdot\overline{y}$
    \end{enumerate}
    \item Тождества для эдентичных операндов
    \item Тождества с константным операндом
  \end{enumerate}
  \begin{definition}
    Функции $f$ и $g$ называются двойственными, если $f(x_1$, $\ldots$, $x_n)=g(\overline{x_1}$, $\ldots$, $\overline{x_n})$. Обозначение $g=f*$.
  \end{definition}
  \begin{definition}
    Если функция двойственна самой себе, то она называется самодвойственной.
  \end{definition}
\end{document}