\documentclass[a4paper, 12pt]{article}

\usepackage{import}

% Корректность отображения всех шрифтов, кодировок и мат. символов
\usepackage[T2A]{fontenc}
\usepackage[utf8]{inputenc}
\usepackage[english, russian]{babel}
\usepackage{amssymb, amsmath, amsthm, mathtools}

% Отображение содержания
\usepackage{tocloft}

% Вставка картинок
\usepackage{graphicx}
\usepackage{tikz}
\usepackage{tkz-euclide}
\usepackage{asymptote}

\usepackage{wrapfig}        % Огибание картинок текстом
\usepackage{cancel}         % Зачёркивания
\usepackage{indentfirst}    % Отступ у первого абзаца
\usepackage{xcolor}         % Цвета
\setlength{\parskip}{.5ex}  % Отступы между абзацами
\usepackage{enumitem}       % Работа со списками
% \usepackage{minted}       % Вставка блоков кода

\usepackage{hyperref}       % гиперссылки
\definecolor{linkcolor}{HTML}{225ae2} % Цвет ссылок
\definecolor{urlcolor}{HTML}{225ae2} % Цвет гиперссылок
\hypersetup{
    pdfstartview=FitH, 
    linkcolor=linkcolor,
    urlcolor=urlcolor,
    colorlinks=true}
\setlength{\arrayrulewidth}{0.5mm} %Толщина линейки в таблицах
\setlength{\tabcolsep}{18pt} %Разделение между столбцами в таблице

% Отступы на странице
\usepackage[left=2cm, right=1.5cm, top=2cm, bottom=2cm]{geometry}

\usepackage{cmap}            % Русский поиск в PDF документе
\usepackage{etoolbox}
\usepackage{soul}            % Разряженный текст \so{} и подчеркивание \ul{}
\usepackage{soulutf8}        % Поддержка UTF8 в soul

\usepackage{titlesec}        % Форматирование заголовков
\titleformat{\section}{\LARGE \bfseries}{\thesection}{1em}{}
\titleformat{\subsection}{\Large\bfseries}{\thesubsection}{1em}{}
\titleformat{\subsubsection}{\large\bfseries}{\thesubsubsection}{1em}{}

\newcommand{\R}{\mathbb R}
\newcommand{\Q}{\mathbb Q}
\newcommand{\Z}{\mathbb Z}
\newcommand{\N}{\mathbb N}
\newcommand{\CC}{\mathbb C}
\newcommand{\F}{\mathbb F}
\newcommand{\aug}{\fboxsep=-\fboxrule\!\!\!\fbox{\strut}\!\!\!}
\newcommand{\sgn}{\operatorname{sgn}}
\newcommand{\id}{\mathrm{id}}
\renewcommand{\phi}{\varphi}
\renewcommand{\epsilon}{\varepsilon}

\newsavebox{\boxedalignbox}
\newenvironment{boxedalign*}
  {\begin{equation*}\begin{lrbox}{\boxedalignbox}$\begin{aligned}}
  {\end{aligned}$\end{lrbox}\fbox{\usebox{\boxedalignbox}}\end{equation*}}

\newcommand\tab[1][.5cm]{\hspace*{#1}}

% Подписи для матриц
\newcommand\undermat[2]{\makebox[0pt][l]{$\smash{\underbrace
{\phantom{\begin{matrix}#2\end{matrix}}}_{\text{$#1$}}}$}#2}
\newcommand\overmat[2]{\makebox[0pt][l]{$\smash{\overbrace
{\phantom{\begin{matrix}#2\end{matrix}}}^{\text{$#1$}}}$}#2}

% Значек "пусть"
\newlength{\tempheight}  
\newcommand{\Let}[0]{  
\mathbin{\text{\settoheight{\tempheight}{\mathstrut}\raisebox{0.5\pgflinewidth}{%
\tikz[baseline,line cap=round,line join=round] \draw (0,0) --++ (0.4em,0) --++ (0,1.5ex) --++ (-0.4em,0);
}}}}


% \newcounter{lemcount}
% \newcounter{thcount}
% \newcounter{offercount}
% \newcounter{concount}
% \newcounter{subthcount}
% \newcounter{defcount}

\theoremstyle{definition}
\newtheorem*{definition}{Определение}
% \newtheorem{definitionnum}[defcount]{Определение}
\newtheorem*{example}{Примеры}
\newtheorem*{example1}{Пример}
\newtheorem*{exercise}{Упражнение}


\theoremstyle{plain}
\newtheorem*{theorem}{Теорема}
% \newtheorem{theoremnum}[thcount]{Теорема}
\newtheorem*{consequense}{Следствие}
\newtheorem*{consequenses}{Следствия}
% \newtheorem{consequensenum}[concount]{Следствие}
\newtheorem*{lemma}{Лемма}
% \newtheorem{lemmanum}[lemcount]{Лемма}
\newtheorem*{subtheorem}{Утверждение}
% \newtheorem{subtheoremnum}[subthcount]{Утверждение}
\newtheorem*{algorithm}{Алгоритм}
\newtheorem*{properties}{Свойства}
\newtheorem*{properties1}{Свойство}


\theoremstyle{remark}
\newtheorem*{remark}{Замечание}
\newtheorem*{offer}{Предложение}
% \newtheorem{offernum}[offercount]{Предложение}
\begin{document}
  \tableofcontents
  \fontsize{14pt}{20pt}\selectfont
  \newpage
  \section{Билет 1 (Основные определения и функции алгебры логики)}
  \begin{definition}
    Упорядоченный набор - функция, которая ставит в соответствие каждому элементу множества $\{1, \ldots, n\}$ элемент из множества $\{a_1,\ldots,a_n\}$ : $1\rightarrow a_1$, $\ldots$, $n\rightarrow a_n$.
  \end{definition}
  Декартовое произведение множеств $A_1\times\ldots\times A_n=(a_1, \ldots, a_n)$ : $a_i\in A_i$.
  \begin{definition}
    Пусть функция $f$ определена на $A_1\times\ldots\times A_n$, тогда $f$ - $n$-местная функция.
  \end{definition}
  \begin{definition}
    Множество $B_n=E_2\times\ldots\times E_2$, где $E_i=\{0,1\}$, называется $n$-мерным булевым кубом.
  \end{definition}
  \begin{definition}
    Функция $f:B_n \to E_2$ называется функцией алгебры логики. Множество всех таких функций обозначим $P_2$.
  \end{definition}
  Представление функции $f(x_1, \ldots, x_n)$ в виде таблицы, имеющей $n+1$ столбец:\\
  $\begin{matrix}
    x_1 & \ldots & x_{n-1} & x_n & f\\
    0 & \ldots & 0 & 0 & 0\\
    0 & \ldots & 0 & 0 & 1\\
    0 & \ldots & 0 & 1 & 0\\
    \vdots & \null & \vdots & \vdots & \vdots\\
    1 & \ldots & 1 & 1 & 1 
  \end{matrix}$\\
  Так как число различных первых $n$ столбцов $2^n$, так как в каждой ячейке одного столбца может быть либо 0, либо 1. $\Longrightarrow$ число функций будет $2^{2^n}$, так как для каждого набора значение функции может быть либо 0, либо 1.
  \begin{definition}
    Переменная $x_i$ называется существенной, если существуют наборы $\alpha_1$, $\ldots$, $\alpha_{i-1}$, $1$, $\alpha_{i+1}$, $\ldots$, $\alpha_n$ и $\alpha_1$, $\ldots$, $\alpha_{i-1}$, $0$, $\alpha_{i+1}$, $\ldots$, $\alpha_n$, на которых функция принимает различные значения. В противном случае переменная $x_i$ называется несущественной (фиктивной).
  \end{definition}
  \begin{definition}
    Пусть $x_i$ - фиктивная переменная, тогда, если функция $f(x_1$, $\ldots$, $x_{i-1}$, $x_{i+1}$, $\ldots$, $x_n)=g(x_1$, $\ldots$, $x_{i-1}$, $0$, $x_{i+1}$, $\ldots$, $x_n)$, то функция $g$ называется полученной из $f$ добавлением фиктивной переменной. Функция удаления фиктивной переменной определяется аналогично.
  \end{definition}
  \begin{definition}
    Функция называется симметрической, если при любых перестановках переменных $x_{i_1}, \ldots, x_{i_n}$ значение функции не меняется.
  \end{definition}
  Элементарные функции в алгебре логики:
  \begin{enumerate}
    \item константы 0, 1
    \item тождественный $x$
    \item отрицание $\overline{x}$
    \item конъюнкция $x\wedge y$
    \item дизъюнкция $x\vee y$
    \item имплекация $x\rightarrow y$
    \item штрих Шеффера $x|y$
    \item стрелка Пирса $x\downarrow y$
    \item сложение по модулю 2
    \item эквивалентность
  \end{enumerate}
  \section{Билет 2 (Основные определения алгебры логики, связанные с формулой)}
  \begin{definition}
    Формула - слово в некотором алфавите $A$.
  \end{definition}
  \begin{definition}
    Алфавит - конечное или бесконечное множество.
  \end{definition}
  \begin{definition}
    Слово - произвольная функция, определённая на начальном отрезке натурального ряда и принимающая на нём значения из $A$.
  \end{definition}
  \begin{definition}
    Пусть $F$ - множество функций алгебры логики, $S$ - множество символов, обозначающих функции из $F$, тогда отображение $\Sigma : S \to F$ - сигнатура для $F$.
  \end{definition}
  \begin{definition}
    Пусть $X=\{x_1, \ldots\}$ - символы переменных.\\
    База индукция: если $x_i$ - символ переменной, то однобуквенное слово, состоящее из $x_i$ - формула в сигнатуре $\Sigma$.\\
    Пусть $s\in S$, $f=\Sigma(s)$ - функция от $n$ переменных, Ф$_1$, $\ldots$, Ф$_n$ - формулы в сигнатуре $\Sigma$, тогда слово $s($Ф$_1$, $\ldots$, Ф$_n)$ - формула в сигнатуре $\Sigma$.  
  \end{definition}
  \begin{definition}
    Пусть Ф - формула, $\tilde{x}$ - упорядоченный набор ($x_{i_1}, \ldots, x_{i_n})$, содержащий все переменные формулы Ф, $\tilde{\alpha}=(\alpha_1, \ldots, \alpha_n)$ - двоичный набор.\\
    База индукции: Ф - однобуквенное слово $x_{i_j}$, тогда Ф$[\tilde{x}, \tilde{\alpha}]=\alpha_j$ - значение формулы на наборе $\tilde{\alpha}$.\\
    Пусть Ф - $s($Ф$_1, \ldots, $Ф$_n)$, $f=\Sigma(s)$, причём Ф$_1[\tilde{x}, \tilde{\alpha}]=\beta_1$, $\ldots$, Ф$_n[\tilde{x}, \tilde{\alpha}]=\beta_n$, тогда $f(\beta_1, \ldots, \beta_n)$ - значение формулы Ф на наборе значений переменных.
  \end{definition}
  \begin{definition}
    Формулой, определяющей функцию $f$ алгебры логики, определённой на $B_n$, называется формула Ф такая, что $\forall$ набора $\tilde{\alpha}=(\alpha_1$, $\ldots$, $\alpha_n)\in B_n$ $f(\tilde{\alpha})=$Ф$[\tilde{x}, \tilde{\alpha}]$.
  \end{definition}
  \begin{definition}
    Формулы в сигнатуре, представляющие собой переменные, называются вырожденными, остальные - невырожденными. Если функция определяется невырожденной формулой в сигнатуре $\Sigma:S \to F$, то она получена суперпозициями над $F$, где $F$ - множество функций.
  \end{definition}
  \begin{definition}
    (Другое определение суперпозиции) Если одну функцию можно получить с помощью конечного числа применений следующих трёх операций, то данная функция называется функцией, полученной суперпозициями над $F$.\\
    Операции:\\
    \begin{enumerate}
      \item Операция подстановки переменных. Пусть $f(x_1$, $\ldots$, $x_n)\in P_2$, $g(x_1$, $\ldots$, $x_n)$ - функция, определённая на $B_n$, такая, что $g(x_1$, $\ldots$, $x_n)=f(x_{i_1}$, $\ldots$, $x_{i_n})$, где набор ($i_1$, $\ldots$, $i_n$) - набор элементов ($1$, $\ldots$, $n$) (они необязательно различны). Тогда $g$ получена из $f$ операцией подстановки переменных.
      \item Операция подстановки функции в функцию. Пусть $f(x_1$, $\ldots$, $x_n)$, $g(x_1$, $\ldots$, $x_m)$, $h$ определена на $B_{n+m-1}$ и $h(x_1$, $\ldots$, $x_{n+m-1})=f(x_1$, $\ldots$, $x_{n-1}$, $g(x_n$, $\ldots$, $x_{n+m-1}))$, тогда функция $h$ получена из функций $f$ и $g$ операцией подстановки одной функции в другую.
      \item Операция добавления или удаления фиктивных переменных. Пусть $x_i$ - фиктивная переменная, тогда если функция $f(x_1$, $\ldots$, $x_{i-1}$, $x_{i+1}$, $\ldots$, $x_n)=g(x_1$, $\ldots$, $x_{i-1}$, $0$, $x_{i+1}$, $\ldots$, $x_n)$, то функция $g$ называется полученной из $f$ добавлением фиктивной переменной. Функция удаления фиктивной переменной определяется аналогично.
    \end{enumerate}
  \end{definition}
  \section{Билет 3 (Основные тождества алгебры логики)}
  \begin{definition}
    Формулы $F_1$ и $F_2$ называются эквивалентными, если они определяют равные функции относительно объединения их переменных. Функции называются равными, если их области определения равны и $\forall x\in D_f(x)$ $f(x)=g(x)$. Слово $F_1=F_2$, если формулы $F_1$ и $F_2$ эквивалентны, называется тождеством. 
  \end{definition}
  Основные тождества:
  \begin{enumerate}
    \item Ассоциативность операций: $\wedge$, $\vee$, $\neg$, $\leftrightarrow$.
    \item Дистрибутивности:
    \begin{enumerate}
      \item $(x\vee y)\wedge z=(x\wedge z)\vee(y\wedge z)$
      \item $(x\wedge y)\vee z=(x\vee z)\wedge(y\vee z)$
      \item $(x+y)\cdot z=x\cdot z+y\cdot z$
    \end{enumerate}
    \item Тождества для отрицания: 
    \begin{enumerate}
      \item $\overline{\overline{x}}=x$
      \item $\overline{x\wedge y}=\overline{x}\vee \overline{y}$
      \item $\overline{x\vee y}=\overline{x}\wedge \overline{y}$
      \item $x\cdot\overline{x}=0$
      \item $x\vee\overline{x}=1$
      \item $\overline{x\rightarrow y}=x\cdot\overline{y}$
    \end{enumerate}
    \item Тождества для эдентичных операндов
    \item Тождества с константным операндом
  \end{enumerate}
  \begin{definition}
    Функция $g$ называется двойственной к $f$, если $g(x_1$, $\ldots$, $x_n)=\overline{f}(\overline{x_1}$, $\ldots$, $\overline{x_n})$. Обозначение $g=f^*$.
  \end{definition}
  \begin{definition}
    Если функция двойственна к самой себе, то она называется самодвойственной.
  \end{definition}
  \begin{theorem}(принцип двойственности)
    Если Ф - формула в сигнатуре $\Sigma: S\to F$, определяющая некоторую функцию $g$, то эта формула в сигнатуре $\Sigma^*: S\to F^*$ определяет двойственную функцию $g^*$.
  \end{theorem}
  \begin{proof}
    База индукции: пусть $x_i$ - символ переменной, тогда однобуквенное слово, состоящее из $x_i$ - формула в сигатуре $\Sigma$, определяющая одноместную функцию $g$. Эта формула в сигнатуре $\Sigma^*$ имеет вид $\overline{x_i}$, то есть она определяет функцию, двойственную к $g$.\\
    Пусть $s\in S$, $f=\Sigma(s)$ - формула от $n$ переменных, Ф$_1$, $\ldots$, Ф$_n$ - формулы в сигнатуре $\Sigma$, тогда слово $s$(Ф$_1$, $\ldots$, Ф$_n$) - формула в сигнатуре $\Sigma$. В $\Sigma^*(s)=(\Sigma(s))^*=(\Sigma(s$(Ф$_1$, $\ldots$, Ф$_n$)))$^*=f^*$, то есть данная формула определяет в двойственной сигнатуре двойственнную функцию.
  \end{proof}
  \section{Билет 4 (Совершенные д.н.ф. и к.н.ф)}
  \begin{definition}
    Выражение $f(x_1$, $\ldots$, $x_n)=\bigvee\limits_{(\sigma_1, \ldots,\sigma_n):f(\sigma_1, \ldots,\sigma_n)=1}x_1^{\sigma_1}\cdot\ldots\cdot x_n^{\sigma_n}\cdot f(\sigma_1,\ldots ,\sigma_n)$ называется совершенной дизъюнктивной нормальной формой. $x_i^{\sigma_i}=$
    $\begin{cases}
      x_i,\sigma_i=1\\
      \overline{x}_i,\sigma_i=0
    \end{cases}$.
  \end{definition}
  \begin{theorem}
    Для любой функции $f(x_1$, $\ldots$, $x_n)$ алгебры логики и $\forall m$ верно равенство:\\
    $f(x_1$, $\ldots$, $x_n)=\bigvee\limits_{(\sigma_1, \ldots,\sigma_m)\in B_m}x_1^{\sigma_1}\cdot\ldots\cdot x_m^{\sigma_m}\cdot f(\sigma_1$, $\ldots$, $\sigma_m$, $x_{m+1}$, $\ldots$, $x_n)$.
  \end{theorem}
  \begin{proof}
    Рассмотрим прозвольный набор $(\alpha_1$, $\ldots$, $\alpha_m$), если $(\alpha_1$, $\ldots$, $\alpha_m)\neq(\sigma_1$, $\ldots$, $\sigma_m$), то $\exists \alpha_i\neq\sigma_i$ $\Longrightarrow$ $\alpha_i^{\sigma_i}=0$ $\Longrightarrow$ данное слагаемое будет равно нулю. Тогда единственным не нулевым членом будет $(\alpha_1^{\alpha_1}\cdot\ldots\cdot\alpha_m^{\alpha_m})\cdot f(\alpha_1$, $\ldots$, $\alpha_m$, $x_{m+1}$, $\ldots$, $x_n)=f(x_1$, $\ldots$, $x_n$).
  \end{proof}
  \begin{theorem}
    Любую функцию алгебры логики можно представить с помощью суперпозиций конъюнкции, дизъюнкции и отрицания.
  \end{theorem}
  \begin{proof}
    Так как любая функция алгебры логики, кроме тождественного нуля, реализуется совершенной д.н.ф., значит она представима суперпозициями конъюнкции, дизъюнкции и отрицания. Тождественный ноль можно представить так: $x\wedge\overline{x}=0$.
  \end{proof}
  \begin{theorem}
    Любая функция алгебры логики, кроме тождественной единицы, представима в виде совершенной конъюнктивной нормальной формы.
  \end{theorem}
  \begin{proof}
    Так как любая функция алгебры логики, кроме тождественного нуля, представима в виде совершенной д.н.ф., тогда по принципу двойственности\\
    $f(x_1$, $\ldots$, $x_n)=\bigwedge\limits_{(\sigma_1, \ldots,\sigma_n):f^*(\sigma_1, \ldots,\sigma_n)=1}x_1^{\sigma_1}\vee\ldots\vee x_n^{\sigma_n}$ $\Longrightarrow$\\
    $f(x_1$, $\ldots$, $x_n)=\bigwedge\limits_{(\delta_1, \ldots,\delta_n):f(\delta_1, \ldots,\delta_n)=1}x_1^{\overline{\delta}_1}\vee\ldots\vee x_n^{\overline{\delta}_n}$.
  \end{proof}
  \section{Билет 5 (Полные системы в алгебре логики)}
  \begin{definition}
    Система функций называется полной в $P_2$, если через них выражаются все функции в $P_2$.
  \end{definition}
  \begin{example}
    \begin{enumerate}
      \item $\wedge$ и $\neg$
      \item $\vee$ и $\neg$
      \item $x|y$
      \item $x\downarrow y$
    \end{enumerate}
  \end{example}
  \begin{definition}
    Полиномы по модулю 2 вида: $\sum\limits_{\{i_1,\dots,i_s\}\subseteq{1,\ldots,n}}a_{i_1,\ldots,i_s}\cdot x_{i_1}\cdot\ldots\cdot x_{i_s}$ называются полиномами Жегалкина.
  \end{definition}
  \begin{theorem}(Жегалкина)\\
    Любая функция алгебры логики представима полиномом Жегалкина, причём единственным образом.
  \end{theorem}
  \begin{proof}
    Так как в каждом мономе полинома Жегалкина $n$ перменных, каждая из которых может быть либо 0, либо 1, а коэффициент перед каждым мономом может принимать значение 0 или 1 $\Longrightarrow$ всего есть $2^{2^n}$ различных полиномов Жегалкина.\\
    Пусть два различных полинома Жегалкина задают одну функцию, тогда мы получим ненулевой полином, задающий нулевую константу $\Longrightarrow$ противоречие $\Longrightarrow$  Любая функция алгебры логики представима полиномом Жегалкина, причём единственным образом.
  \end{proof}
  \section{Билет 6 (Замыкание множества и замкнутые классы)}
  \begin{definition}
    Множество функций, которые можно пулучить из данного множества $M$ функций алгебры логики, называется замыканием множества $M$ и обозначается $[M]$.
  \end{definition}
  \begin{example}
    \begin{enumerate}
      \item $P_2=[P_2]$
      \item [{1, $x+y$}] - множество линейных функций
    \end{enumerate}
  \end{example}
  \begin{properties}
    \begin{enumerate}
      \item $M\subseteq[M]$
      \item $[[M]]=[M]$
      \item Если $M_1\subseteq M_2$, то $[M_1]\subseteq[M_2]$
      \item $[M_1]\cup[M_2]\subseteq[M_1\cup M_2]$
    \end{enumerate}
  \end{properties}
  \begin{proof}
    \begin{enumerate}
      \item По определению замыкания.
      \item Из первого следует, что $[M]\subseteq[[M]]$, а $[[M]]\subseteq[M]$, так как в противном случае существовала бы функция, которая не выражается суперпозициями функций из $M$, но выражается суперпозициями функций, которые выражаются суперпозициями функций из $M$, а значит, она выражается суперпозициями из $M$ $\Longrightarrow$ противоречие.
      \item Если функция получается суперепозициями из $M_1$, то её можно получить суперпозициями из $M_2$, так как все функции $M_1$ являются функциями $M_2$.
      \item Пусть функция $f\in[M_1]\cup[M_2]$, тогда она получается суперпозициями из $M_1$ или из $M_2$, пусть для определённости она выражается суперпозициями из $M_1$, но тогда её можно получить суперпозициями из $M_1\cup M_2$, то есть $f\in[M_1\cup M_2]$
    \end{enumerate}
  \end{proof}
  \begin{definition}
    Класс функций $M$ называется замкнутым, если $[M]=M$.
  \end{definition}
  \begin{example}
    \begin{enumerate}
      \item $P_2=[P_2]$
      \item $L=[L]$, $L$ - множество линейных функций.
    \end{enumerate}
  \end{example}
  \section{Билет 7 (Классы $T_0$ и $T_1$)}
  \begin{definition}
    Функция $f$ называется функцией, сохраняющей ноль, если на наборе из нулей она принимает значение 0. 
  \end{definition}
  \begin{definition}
    Функция $f$ называется функцией, сохраняющей единицу, если на наборе из единиц она принимает значение 1. 
  \end{definition}
  Класс функций, сохраняющих ноль, обозначим $T_0$, а класс функций, сохраняющих единицу, обозначим $T_1$.
  \begin{theorem}
    Классы $T_0$ и $T_1$ замкнуты.
  \end{theorem}
  \begin{proof}
    \begin{enumerate}
      \item Операция подстановки переменных:\\
      $g(x_1$, $\ldots$, $x_n)=f(x_{i_1}$, $\ldots$, $x_{i_n})$, если функция $f$ сохраняла ноль, то и функция $g$ будет сохранять ноль, если функция $f$ сохраняла единицу, то и функция $g$ будет сохранять единицу.
      \item Операция подстановки одной функции в другую:\\
      $h(x_1$, $\ldots$, $x_{n+m-1})=f(x_1$, $\ldots$, $x_{n-1}$, $g(x_n$, $\ldots$, $x_{n+m-1}))$, если функции $f$ и $h$ сохраняли ноль, то и функция $g$ будет сохранять ноль, если функции $f$ и $g$ сохраняли единицу, то и функция $h$ будет сохранять единицу.
      \item Операция добавления или удаления фиктивной переменной, не влияет на способность функции сохранять ноль или сохранять единицу.
    \end{enumerate}
    Следовательно суперпозициями мы не сможем получить функцию, не принадлежащую данному классу $\Longrightarrow$ классы $T_0$ и $T_1$ - замкнуты. 
  \end{proof}
  \section{Билет 8 (Класс S и лемма, связанная с ним)}
  Класс самодвойственных функций обозначим $S$.
  \begin{theorem}
    Класс $S$ замкнут.
  \end{theorem}
  \begin{proof}
    \begin{enumerate}
      \item Операция подстановки переменных:\\
      Пусть $f(x_1$, $\ldots$, $x_n)\in S$, $g(x_1$, $\ldots$, $x_n)=f(x_{i_1}$, $\ldots$, $x_{i_n})$, тогда $\overline{g}(\overline{x}_1$, $\ldots$, $\overline{x}_n)=\overline{f}(\overline{x}_{i_1}$, $\ldots$, $\overline{x}_{i_n})=f(x_{i_1}$, $\ldots$, $x_{i_n})=g(x_1$, $\ldots$, $x_n)$ $\Longrightarrow$ $g$ - самодвойственная функция.
      \item Операция подстановки функции в функцию:\\
      Пусть $f(x_1$, $\ldots$, $x_n)\in S$, $g(x_1$, $\ldots$, $x_m)\in S$, $h(x_1$, $\ldots$, $x_n$, $x_{n+1}$, $\ldots$, $x_{n+m-1})=f(x_1$, $\ldots$, $x_{n-1}$, $g(x_n$, $\ldots$, $x_{n+m-1}))$, тогда $\overline{h}(\overline{x}_1$, $\ldots$, $\overline{x}_n$, $\overline{x}_{n+1}$, $\ldots$, $\overline{x}_{m+n-1})=\overline{f}(\overline{x}_1$, $\ldots$, $\overline{x}_{n-1}$, $g(\overline{x}_{n}$, $\ldots$, $\overline{x}_{m+n-1}))=\overline{f}(\overline{x}_1$, $\ldots$, $\overline{x}_{n-1}$, $\overline{g}(x_{n}$, $\ldots$, $x_{m+n-1}))=f(x_1$, $\ldots$, $x_{n-1}$, $g(x_{n}$, $\ldots$, $x_{m+n-1}))=h(x_1$, $\ldots$, $x_n$, $x_{n+1}$, $\ldots$, $x_{m+n-1})$ $\Longrightarrow$ $h$ - самодвойственная функция.
      \item Операция добавления или удаления фиктивных переменных:\\
      Пусть $f(x_1$, $\ldots$, $x_n)\in S$, $g(x_1$, $\ldots$, $x_{i-1}$, $0$, $x_{i+1}$, $\ldots$,  $x_n) = f(x_1$, $\ldots$, $x_n)=g(x_1$, $\ldots$, $x_{i-1}$, $1$, $x_{i+1}$, $\ldots$,  $x_n)$, тогда $\overline{g}(\overline{x}_1$, $\ldots$, $\overline{x}_{i-1}$, $1$, $\overline{x}_{i+1}$, $\ldots$,  $\overline{x}_n) = f(x_1$, $\ldots$, $x_n)=g(x_1$, $\ldots$, $x_{i-1}$, $0$, $x_{i+1}$, $\ldots$,  $x_n)$ $\Longrightarrow$ $g$ - самодвойственная функция. 
    \end{enumerate}
  \end{proof}
  \begin{theorem}
    Если функция $f$ не является самодвойственной, то с помощью неё и функции отрицания можно получить константу.
  \end{theorem}
  \begin{proof}
    Пусть $f(x_1$, $\ldots$, $x_n)\notin S$, тогда существует набор $(\alpha_1$, $\ldots$, $\alpha_n$) :\\ 
    \begin{center}
      $f(\alpha_1$, $\ldots$, $\alpha_n)=f(\overline{\alpha}_1$, $\ldots$, $\overline{\alpha}_n)$.\\
    \end{center} 
    Пусть $\phi_i=x^{\alpha_i}$, $\phi(x)=f(\phi_1(x)$, $\ldots$, $\phi_n(x))$,\\ 
    \begin{center}
      тогда $\phi(0)= f(0^{\alpha_1}, \ldots, 0^{\alpha_n})=f(\overline{\alpha}_1, \ldots, \overline{\alpha}_n)=f(\alpha_1, \ldots, \alpha_n)=f(1^{\alpha_1}, \ldots, 1^{\alpha_n})=\phi(1)\Longrightarrow$\\
    \end{center}
      $\Longrightarrow$ $\phi(x)$ - константа, полученная из несамодвойственной функции и отрицания.
  \end{proof}
  \section{Билет 9 (Класс M и лемма, связанная с ним)}
  \begin{definition}
    Пусть $\tilde{\alpha}=(\alpha_1$, $\ldots$, $\alpha_n)$, $\tilde{\beta}=(\beta_1$, $\ldots$, $\beta_n)$ - двоичные наборы, тогда $\tilde{\alpha}\leqslant\tilde{\beta}$, если $\forall i=\overline{1,n}$ $\alpha_i\leqslant\beta_i$.
  \end{definition}
  \begin{definition}
    Функция алгебры логики называется монотонной, если $\forall$ двоичных наборов $\tilde{\alpha}$ и $\tilde{\beta}$ таких, что $\tilde{\alpha}\leqslant\tilde{\beta}$, $f(\tilde{\alpha})\leqslant f(\tilde{\beta})$.
  \end{definition}
  \begin{theorem}
    Класс $M$ монотонных функций - замкнут.
  \end{theorem}
  \begin{proof}
    \begin{enumerate}
      \item Операция подстановки переменных:\\
      $g(x_1$, $\ldots$, $x_n)=f(x_{i_1}$, $\ldots$, $x_{i_n})$, если функция $f$ монотонна, то\\ $\forall\tilde{\alpha}=(\alpha_1$, $\ldots$, $\alpha_n)$ и $\tilde{\beta}=(\beta_1$, $\ldots$, $\beta_n)$ : $\tilde{\alpha}\leqslant\tilde{\beta}$, $f(\tilde{\alpha})\leqslant f(\tilde{\beta})$ $\Longrightarrow$ $\alpha_1\leqslant\beta_1$, $\ldots$, $\alpha_n\leqslant\beta_n$ $\Longrightarrow$\\
      $\Longrightarrow$ $\alpha_{i_1}\leqslant\beta_{i_1}$, $\ldots$, $\alpha_{i_n}\leqslant\beta_{i_n}$ $\Longrightarrow$ $f(\alpha_{i_1}$, $\ldots$, $\alpha_{i_n})\leqslant f(\beta_{i_1}$, $\ldots$, $\beta_{i_n})$ $\Longrightarrow$\\
      $\Longrightarrow$ $g(\alpha_{1}$, $\ldots$, $\alpha_{n})=f(\alpha_{i_1}$, $\ldots$, $\alpha_{i_n})\leqslant f(\beta_{i_1}$, $\ldots$, $\beta_{i_n})=g(\beta_{i_1}$, $\ldots$, $\beta_{i_n})$ $\Longrightarrow$ $g$ - монотонна.
      \item Операция подстановки одной функции в другую:\\ 
      $f(x_{1}$, $\ldots$, $x_{n})$, $g(x_{1}$, $\ldots$, $x_{m})$ - монотонные функции, $h(x_1$, $\ldots$, $x_{n+m-1})=f(x_1$, $\ldots$, $x_{n-1}$, $g(x_n$, $\ldots$, $x_{n+m-1}))$, так как функции $f$ и $g$ монотонны, $\forall\tilde{\alpha}=(\alpha_1$, $\ldots$, $\alpha_{m+n-1})$ и $\tilde{\beta}=(\beta_1$, $\ldots$, $\beta_{m+n-1})$ : $\tilde{\alpha}\leqslant\tilde{\beta}$, $f(\tilde{\alpha})\leqslant f(\tilde{\beta})$ и $g(\alpha_n$, $\ldots$, $\alpha_{m+n-1})=g(\beta_n$, $\ldots$, $\alpha_{m+n-1})$ $\Longrightarrow$\\
      $(\alpha_1$, $\ldots$, $\alpha_{n-1}$, $g(\alpha_n$, $\ldots$, $\alpha_{m+n-1}))\leqslant (\beta_1$, $\ldots$, $\beta_{n-1}$, $g(\beta_n$, $\ldots$, $\beta_{n+m-1}))$ $\Longrightarrow$ $h(\alpha_1$, $\ldots$, $\alpha_{m+n-1})=f(\alpha_1$, $\ldots$, $\alpha_{n-1}$, $g(\alpha_n$, $\ldots$, $\alpha_{m+n-1}))\leqslant f(\beta_1$, $\ldots$, $\beta_{n-1}$, $g(\beta_n$, $\ldots$, $\beta_{n+m-1}))=h(\beta_1$, $\ldots$, $\beta_{m+n-1})$.
      \item Операция добавления или удаления фиктивных переменных:\\$f(x_1$, $\ldots$, $x_{i-1}$, $x_{i+1}$, $\ldots$, $x_n)=g(x_1$, $\ldots$, $x_{i-1}$, $0$, $x_{i+1}$, $\ldots$, $x_n)$, так как $f$ монотонна $\Longrightarrow$ $\forall\tilde{\alpha}=(\alpha_1$, $\ldots$, $\alpha_{i-1}$, $\alpha_{i+1}$, $\ldots$, $\alpha_n)$ и $\tilde{\beta}=(\beta_1$, $\ldots$, $\beta_{i-1}$, $\beta_{i+1}$, $\ldots$, $\beta_n)$ : $\tilde{\alpha}\leqslant\tilde{\beta}$,\\верно $f(\alpha_1$, $\ldots$, $\alpha_{i-1}$, $\alpha_{i+1}$, $\ldots$, $\alpha_n)\leqslant f(\beta_1$, $\ldots$, $\beta_{i-1}$, $\beta_{i+1}$, $\ldots$, $\beta_n)$.\\ Тогда $\tilde{\alpha}$, с добавленной фиктивной переменной, $\leqslant$ $\tilde{\beta}$, с добавленной фиктивной переменной $\Longrightarrow$ $g(\alpha_1$, $\ldots$, $\alpha_{i-1}$, $0$, $\alpha_{i+1}$, $\ldots$, $\alpha_n)=f(\alpha_1$, $\ldots$, $\alpha_{i-1}$, $\alpha_{i+1}$, $\ldots$, $\alpha_n)\leqslant f(\beta_1$, $\ldots$, $\beta_{i-1}$, $\beta_{i+1}$, $\ldots$, $\beta_n)=g(\beta_1$, $\ldots$, $\beta_{i-1}$, $0$, $\beta_{i+1}$, $\ldots$, $\beta_n)$.
    \end{enumerate}
    Следовательно, суперпозициями мы не сможем получить функцию, не принадлежащую данному классу $\Longrightarrow$ класс $M$ замкнут. 
  \end{proof}
  \begin{theorem}
    Если $f$ - немонотонная функция, то из неё и констант можно получить отрицание.
  \end{theorem}
  \begin{proof}
    Пусть $f(x_1$, $\ldots$, $x_n)$ - немонотонная функция, тогда $\exists\tilde{\alpha}$ и $\tilde{\beta}$ : $\tilde{\alpha}\leqslant\tilde{\beta}$ и $f(\tilde{\alpha})=1$, а $f(\tilde{\beta})=0$. Так как наборы различны, то $\exists\alpha_{i_1}=\ldots=\alpha_{i_k}=0$ и $\beta_{i_1}=\ldots=\beta_{i_k}=1$,\\ а $\forall j\in(1$, $\ldots$, $n)\setminus(i_1$, $\ldots$, $i_k$) $\alpha_j=\beta_j$.\\ Пусть наборы $\tilde{\gamma}_0$, $\ldots$, $\tilde{\gamma}_k$ с номерами $(1$, $\ldots$, $n)\setminus(i_1$, $\ldots$, $i_k$) совпадают с набором $\tilde{\alpha}$, набор $\gamma_j$ на позициях с номерами $i_1$, $\ldots$, $i_j$ принимает значение 1, а на позициях $i_{j+1}$, $\ldots$, $i_k$ принимает значение 0, тогда $\tilde{\gamma}_0=\tilde{\alpha}$, а $\tilde{\gamma}_k=\tilde{\beta}$ $\Longrightarrow$ $f(\tilde{\gamma}_0)=1$, $f(\tilde{\gamma}_k)=0$ $\Longrightarrow$ $\exists\tilde{\gamma}_j$ : $f(\tilde{\gamma}_j)=0$, а $f(\tilde{\gamma}_{j-1})=1$ $\Longrightarrow$\\
    $\Longrightarrow$ $\tilde{\gamma}_{j-1}=(\delta_1$, $\ldots$, $\delta_{i_j-1}$, $0$, $\delta_{i_j+1}$, $\ldots$, $\delta_n)$, $\tilde{\gamma}_j=(\delta_1$, $\ldots$, $\delta_{i_j-1}$, $1$, $\delta_{i_j+1}$, $\ldots$, $\delta_n)$.\\
    Тогда функция $\phi(f(\delta_1$, $\ldots$, $\delta_{i_j-1}$, $x$, $\delta_{i_j+1}$, $\ldots$, $\delta_n))$, при $x=0$ функция равна 1, а при $x=1$, функция равна 0, то есть $\phi=\overline{x}$, а так как она получена с помощью функции $f$ и констант, значит, это искомая функция.
  \end{proof}
  \section{Билет 10 (Класс L и лемма, связанная с ним)}
  \begin{definition}
    Функция $f$ называется линейной, если она представима полиномом Жегалкина степени 1.
  \end{definition}
  \begin{theorem}
    Класс $L$ линейных функций замкнут.
  \end{theorem}
  \begin{proof}
    \begin{enumerate}
      \item Операция подстановки переменных:\\
      $g(x_1$, $\ldots$, $x_n)=f(x_{i_1}$, $\ldots$, $x_{i_n})$, если функция $f$ линейна, то $\forall\tilde{\alpha}=(\alpha_1$, $\ldots$, $\alpha_n)$ $f(\tilde{\alpha})=c_0+c_1\alpha_1+\ldots+c_n\alpha_n$, тогда\\
      $g(\alpha_1$, $\ldots$, $\alpha_n)=c_0+c_1\alpha_{i_1}+\ldots+c_n\alpha_{i_n}$ $\Longrightarrow$ $g$ - линейная функция.
      \item Операция подстановки одной функции в другую:\\ 
      $f(x_{1}$, $\ldots$, $x_{n})$, $g(x_{1}$, $\ldots$, $x_{m})$ - линейные функции, $h(x_1$, $\ldots$, $x_{n+m-1})=f(x_1$, $\ldots$, $x_{n-1}$, $g(x_n$, $\ldots$, $x_{n+m-1}))$, так как функции $f$ и $g$ линейны, $\forall\tilde{\alpha}=(\alpha_1$, $\ldots$, $\alpha_{m+n-1})$ $f(\alpha_1$, $\ldots$, $\alpha_n)=c_0+c_1\alpha_1+\ldots+c_n\alpha_n$, $g(\alpha_1$, $\ldots$, $\alpha_m)=c'_0+c'_1\alpha_1+\ldots+c'_n\alpha_n$ $\Longrightarrow$\\
      $\Longrightarrow$ $h(\alpha_1$, $\ldots$, $\alpha_{n+m-1})=c_0+c_1\alpha_1+\ldots+c_{n-1}\alpha_{n-1}+c_ng(\alpha_n$, $\ldots$, $\alpha_{m+n-1})=\\=c_0+c_1\alpha_1+\ldots+c_{n-1}\alpha_{n-1}+c_n(c'_1\alpha_n+\ldots+c'_m\alpha_{m+n-1})$ $\Longrightarrow$ функция $h$ является линейной.
      \item Операция добавления или удаления фиктивных переменных:\\$f(x_1$, $\ldots$, $x_{i-1}$, $x_{i+1}$, $\ldots$, $x_n)=g(x_1$, $\ldots$, $x_{i-1}$, $0$, $x_{i+1}$, $\ldots$, $x_n)$, так как $f$ линейна $\Longrightarrow$ $\forall\tilde{\alpha}=(\alpha_1$, $\ldots$, $\alpha_{i-1}$, $\alpha_{i+1}$, $\ldots$, $\alpha_n)$ $f(\tilde{\alpha})=c_0+c_1\alpha_1+\ldots+c_{i-1}\alpha_{i-1}+c_{i+1}\alpha_{i+1}+\ldots+c_n\alpha_n$, тогда очевидно, что $g(\alpha_1$, $\ldots$, $\alpha_{i-1}$, $0$, $\alpha_{i+1}$, $\ldots$, $\alpha_n)$ тоже линейная функция. 
    \end{enumerate}
    Следовательно, суперпозициями мы не сможем получить функцию, не принадлежащую данному классу $\Longrightarrow$ класс $L$ замкнут. 
  \end{proof}
  \begin{theorem}
    Если функция $f$ нелинейна, то из неё, констант и отрицания можно получить конъюнкцию.
  \end{theorem}
  \begin{proof}
    Пусть $f(x_1$, $\ldots$, $x_n)$ - нелинейная функция, тогда полином Жегалкина без ограничения общности имеет вид: $x_1x_2f_1(x_3$, $\ldots$, $x_n)+x_1f_2(x_3$, $\ldots$, $x_n)+x_2f_3(x_3$, $\ldots$, $x_n)+f_4(x_3$, $\ldots$, $x_n)$. Так как $f1$ не является тождественно нулевой функцией, существует набор $(\alpha_3$, $\ldots$, $\alpha_n)$ : $f_1(\alpha_3$, $\ldots$, $\alpha_n)=1$, тогда $f=x_1x_2+\alpha x_1+\beta x_2+\gamma$ $\Longrightarrow$\\
    $\Longrightarrow$ $f(x_1+\alpha$, $x_2+\beta)=(x_1+\alpha)(x_2+\beta)+\alpha(x_1+\alpha)+\beta(x_2+\beta)+\gamma=x_1x_2+\alpha\beta\gamma$, если $\alpha\beta\gamma=1$, то возьмём $\overline{f}(x_1+\alpha$, $x_2+\beta)=x_1x_2$,  так как данная функция получена из $f$ с помощью констант и отрицания, значит, это искомая функция.
  \end{proof}
  \section{Билет 11 (Теорема Поста)}
  \begin{theorem}
    Система функций полна тогда и только тогда, когда она не содержится ни в одном из классов $T_0$, $T_1$, $S$, $M$, $L$.
  \end{theorem}
  \begin{proof}
    $\underline{\Longrightarrow}$ Если ситсема $F$ функций алгебры логики полна, то $[F]=P_2$. Предположим, что $F\subseteq K$, где $K$ - один из этих классов, тогда $[F]\subseteq[K]\neq P_2$ - противоречие.\\
    $\underline{\Longleftarrow}$ Пусть $F$ не лежит ни в одном из этих классов, тогда $\exists f_1$, $f_2$, $f_3$, $f_4$, $f_5$ : $f_1\notin T_0$, $f_2\notin T_1$, $f_3\notin S$, $f_4\notin M$, $f_5\notin L$.\\
    Рассмотрим $f_1\notin T_0$, тогда $f_1(0$, $\ldots$, $0)=1$. Есть два случая:
    \begin{enumerate}
      \item Пусть $f_1\notin T_1$, тогда $\phi(x)=f_1(x$, $\ldots$, $x)=\overline{x}$, то есть мы получили из $f_1$ функцию отрицания. Тогда по лемме о несамодвойственной функции из $f_3$ и $\overline{x}$ можно получить константы.
      \item Пусть $f_1\in T_1$, тогда $\phi(x)=f_1(x$, $\ldots$, $x)=1$, то есть $\phi(x)$ - константа 1. Рассмотрим $f_2\notin T_1$, тогда $f_2(f_1(x$, $\ldots$, $x))=0$, то есть мы получили константу 0.
    \end{enumerate}
    Тогда по лемме о немонотонной функции из $f_4$ и констант можно получить $\overline{x}$, а по лемме о нелинейной функции из $f_5$, $\overline{x}$ и констант можно получить $x\wedge y$, то есть мы получим полную систему {$x\wedge y$, $\overline{x}$}.
  \end{proof}
  \section{Билет 12 (Теорема о предполных классах)}
  \begin{definition}
    Класс $K$ функций алгебры логики называется предполным, если $[K]\neq P_2$ и если $f\in P_2\setminus K$, то $[\{f\}\cup K]=P_2$.
  \end{definition}
  \begin{theorem}
    В $P_2$ нет предполных классов, отличных от $T_0$, $T_1$, $S$, $M$, $L$.
  \end{theorem}
  \begin{proof}
    Пусть класс $K$ - предполный класс, отличный от данных пяти классов. Этот класс замкнут, так как в противном случае можно было бы выбрать функцию $f$ : $f\in[K]$ и $f\notin K$, тогда $[\{f\}\cup K]=[K]$, но так как класс $K$ является предполным, то $[K]=P_2$ $\Longrightarrow$ противоречие с тем, что класс $K$ не является полным.\\
    Так как класс $K$ замкнут, то он содержится в одном из классов $T_0$, $T_1$, $S$, $M$, $L$ (обозначим этот класс $Q$), иначе по теореме Поста он был бы полным, а он по условию таким не является. Пусть класс $K$ не совпадает с классом $Q$, тогда $\exists f\in Q\setminus K$ $\Longrightarrow$ $[\{f\}\cup K]\subseteq[Q]\neq P_2$ - противоречие.\\
    Пусть $f\in P_2\setminus Q$, тогда если $[Q\cup\{f\}]=[Q']\neq P_2$, то $Q'$ содержится в одном из оставшихся классов, что невозможно, а значит, класс $Q$ является предполным.
  \end{proof}
  \section{Билет 13 (Теорема о конечной полной подсистеме полной системы в $P_2$)}
  \begin{theorem}
    В любой полной системе алгебры логики можно выделить полную подсистему, состоящую из 4 функций.
  \end{theorem}
  \begin{proof}
    Пусть система $F$ полна, выберем в ней функции $f_1$, $f_2$, $f_3$, $f_4$, $f_5$ : $f_1\notin T_0$, $f_2\notin T_1$, $f_3\notin S$, $f_4\notin M$, $f_5\notin L$, по теореме Поста система из этих функций полна. Если $f_1\in T_1$, тогда $f_1\notin S$, тогда функцию $f_3$ можно выбрать равной $f_1$, а если $f_1(1$, $\ldots$, $1)=0$, то $f_1\notin M$, то есть $f_4$ можно выбрать равной $f_1$ $\Longrightarrow$ в обоих случаях мы получаем полную систему из четырёх функций.
  \end{proof}
 \section{Билет 14 (Базис в $P_k$)}
  \begin{definition}
    Пусть $K$ - замкнутый класс, $F$ - система функций данного класса, тогда $F$ называется полной, если $[F]=K$.
  \end{definition}
  \begin{definition}
    Система функций некоторого класса $K$ называется базисом, если она полна в $K$, но каждая её собствееная подсистема неполна в $K$.
  \end{definition}
  \begin{example}
    \{0, 1, $x_1\cdot x_2$, $x_1\vee x_2$\} - базис в $M$
  \end{example}
  \begin{theorem}
    Каждый замкнутый класс функций алгебры логики имеет конечный базис. (Без доказательства)
  \end{theorem}
  \begin{theorem}
    Число замкнутых классов в $P_2$ счётно. (Без доказательства)
  \end{theorem}
  \section{Билет 15 (Основные определения в $P_k$)}
  \begin{definition}
    Отображение $f:E_k\times\ldots\times E_k\to E_k$ - функция $k$-значной логики.
  \end{definition}
  Элементарные функции:
  \begin{enumerate}
    \item $\overline{x}=x+1(mod$ $k)$
    \item $\sim x=k-1-x$
    \item $J_i(x)=\begin{matrix}
      k-1\textup{, если} x=i\\
      0\textup{, если} x\neq i
    \end{matrix}$
    \item $j_i(x)=\begin{matrix}
      1\textup{, если} x=i\\
      0\textup{, если} x\neq i
    \end{matrix}$
    \item $min(x_1$, $x_2)$
    \item $max(x_1$, $x_2)$
    \item $x_1\cdot x_2(mod$ $k$)
    \item $x_1+x_2(mod$ $k$)
  \end{enumerate}
  \begin{definition}
    Отображение $\Sigma:S\to F$, где $S$ - множество символов, обозначующих функции из $P_k$, а $F$ - множество функций в $P_k$, называется сигнатурой.
  \end{definition}
  \begin{definition}
    База индукции: пусть $x_i$ - символ переменной, тогда однобуквенное слово, состоящее из $x_i$ - формула в сигнатуре.\\
    Пусть $s\in S$, $f=\Sigma(s)$ - функция от $n$ переменных, Ф$_1$, $\ldots$, Ф$_n$ - формулы в сигнатуре $\Sigma$, тогда слово $s($Ф$_1$, $\ldots$, Ф$_n$) - формула в сигнатуре $\Sigma$.
  \end{definition}
  \begin{definition}
    Пусть Ф - формула, $\tilde{x}=(x_{i_1}$, $\ldots$, $x_{i_n})$ - упорядоченный набор, содержащий все переменные формулы Ф, $\tilde{\alpha}=(\alpha_1$, $\ldots$, $\alpha_n)$ - двоичный набор.\\
    База индукции: Ф - однобуквенное слово $x_{i_j}$, тогда Ф$[\tilde{x}$, $\tilde{\alpha}]=\alpha_j$ - значение формулы на наборе.\\
    Пусть $s\in S$, $f=\Sigma(s)$, Ф$_1$, $\ldots$, Ф$_n$ - формулы в сигнатуре. Обозначим Ф$_1[\tilde{x}$, $\tilde{\alpha}]=\beta_1$, $\ldots$, Ф$_n[\tilde{x}$, $\tilde{\alpha}]=\beta_n$, тогда $f(\beta_1$, $\ldots$, $\beta_n)$ - значение формулы на наборе $\tilde{\alpha}$.
  \end{definition}
  \begin{definition}
    Операции:
    \begin{enumerate}
      \item Операция подстановки переменных. Пусть $f(x_1$, $\ldots$, $x_n)\in P_k$, $g(x_1$, $\ldots$, $x_n)$ - функция, определённая на $B_n$, такая, что $g(x_1$, $\ldots$, $x_n)=f(x_{i_1}$, $\ldots$, $x_{i_n})$, где набор ($i_1$, $\ldots$, $i_n$) - набор элементов ($1$, $\ldots$, $n$) (они необязательно различны). Тогда $g$ получена из $f$ операцией подстановки переменных.
      \item Операция подстановки функции в функцию. Пусть $f(x_1$, $\ldots$, $x_n)$, $g(x_1$, $\ldots$, $x_m)$, $h$ определена на $B_{n+m-1}$ и $h(x_1$, $\ldots$, $x_{n+m-1})=f(x_1$, $\ldots$, $x_{n-1}$, $g(x_n$, $\ldots$, $x_{n+m-1}))$, тогда функция $h$ получена из функций $f$ и $g$ операцией подстановки одной функции в другую.
      \item Операция добавления или удаления фиктивных переменных. Пусть $x_i$ - фиктивная переменная, тогда если функция $f(x_1$, $\ldots$, $x_{i-1}$, $x_{i+1}$, $\ldots$, $x_n)=g(x_1$, $\ldots$, $x_{i-1}$, $0$, $x_{i+1}$, $\ldots$, $x_n)$, то функция $g$ называется полученной из $f$ добавлением фиктивной переменной. Функция удаления фиктивной переменной определяется аналогично.
    \end{enumerate}
  \end{definition}
  \section{Билет 16 (Простейшие тождества для функций и аналог совершенной д.н.ф. в $P_k$)}
  Тождества для функций в $P_k$:
  \begin{enumerate}
    \item операции $min(x_1$, $x_2)$, $max(x_1$, $x_2$), $x_1\cdot x_2(mod$ $k)$, $x_1+x_2(mod$ $k)$ ассоциативны и коммутативны
    \item $min(max(x_1$, $x_2)$, $x_3)=max(min(x_1$, $x_3)$, $min(x_2$, $x_3))$
    \item ($x_1+x_2)\cdot x_3=(x_1\cdot x_3)+(x_2\cdot x_3)$
    \item $\sim(\sim x)=x$
    \item $\sim min(x_1$, $x_2)=max(\sim x_1$, $\sim x_2)$
  \end{enumerate}
  \begin{definition}
    Выражение $\bigvee\limits_{(\sigma_1,\ldots,\sigma_n)\in (E_k)^n}min(J_{\sigma_1}(x_1)$, $\ldots$, $J_{\sigma_n}(x_n)$, $f(\sigma_1$, $\ldots$, $\sigma_n))$ - аналог совершенной дизъюнктивной нормальной формы для $P_k$.
  \end{definition}
  \begin{theorem}
    Любая функция, не являющаяся тождественно нулевой, имеет аналог совершенной д.н.ф.
  \end{theorem}
  \begin{proof}
    Рассмотрим произвольный набор $(\alpha_1$, $\ldots$, $\alpha_n)$, так как $J_{\sigma_i}(\alpha_j)=0$ $\forall j\neq i$, а для $j=i$ $J_{\sigma_i}(\alpha_i)=k-1$, значит, все члены, кроме $\alpha_1=\sigma_1$, $\ldots$, $\alpha_n=\sigma_n$, будут равны нулю, а значит, останется только $min(J_{\sigma_1}(\alpha_1)$, $\ldots$, $J_{\sigma_n}(\alpha_n)$, $f(\alpha_1$, $\ldots$, $\alpha_n))=f(\alpha_1$, $\ldots$, $\alpha_n)$.
  \end{proof}
  \section{Билет 17 (Полные системы в $P_k$)}
  \begin{definition}
    Система $F$ функций в $P_k$ называется полной, если любая функция из $P_k$ получается суперпозициями из $F$.
  \end{definition}
  \begin{example}
    \begin{enumerate}
      \item $P_k$
      \item $\{0$, $1$, $\ldots$, $k-1$, $J_0(x)$, $\ldots$, $J_{k-1}(x)$, $min(x_1$, $x_2$), $max(x_1$, $x_2)\}$
      \item $max(x_1$, $x_2)$, $\overline{x}$
      \item $min(x_1$, $x_2)$, $\overline{x}$
      \item $\{0$, $1$, $\ldots$, $k-1$, $j_0(x)$, $\ldots$, $j_{k-1}(x)$, $x_1+x_2$, $x_1\cdot x_2\}$
      \item $V_k(x_1$, $x_2)=max(x_1$, $x_2)+1(mod$ $k$)
    \end{enumerate}
  \end{example}
  Докажем полноту каждой из систем.
  \begin{proof}
    \begin{enumerate}
      \item Так как в системе есть отрицание Поста, то из $\forall x$ можно получить $\{x$, $x+1$, $\ldots$, $x+k-1$\} все эти числа различны по $(mod$ $k$) $\Longrightarrow$ $max(x$, $\ldots$, $x+k-1)=k-1$, тогда из константы $k-1$ можно получить все остальные константы, используя отрицание Поста.\\
      Рассмотрим набор $\{x$, $\ldots$, $x+j-1$, $x+j+1$, $\ldots$, $x+k-1$\}, тогда функция $\phi_j(x)=max(x$, $\ldots$, $x+j-1$, $x+j+1$, $\ldots$, $x+k-1)=\begin{cases}
        k-1, \textup{ при} x+j\neq k-1\\
        k-2, \textup{ при} x+j=k-1
      \end{cases}$. Тогда функция $\psi_j(x)=max(x$, $\ldots$, $x+j-1$, $x+j+1$, $\ldots$, $x+k-1)+1$ (это можно сделать благодаря отрицанию Поста) $\Longrightarrow$ $\psi_j(x)=\begin{cases}
        0, \textup{ при} x+j\neq k-1\\
        k-1, \textup{ при} x+j=k-1
      \end{cases}$. То есть мы получили все константы, $J_i(x)$ $\forall i$, а значит, получили полную систему из примера 2.
      \item Аналогично с предыдущим пунктом, с помощью отрицания Поста можно получить все константы, а значит, можем получить отрицание Лукашевича, а по одному из тождеств, $\sim min(x_1$, $x_2)=max(\sim x_1$, $\sim x_2)$, то есть мы получили полную систему из предыдущего пункта.
      \item Из $V_k(x_1$, $x_2$) получим отрицание Поста: $V_k(x$, $x)=x+1=\overline{x}$ $\Longrightarrow$ можно получить $x+i$ $\forall i$, тогда $max(x_1$, $x_2)=V_k(x_1$, $x_2)+k-1$, то есть мы получили полную систему $\{max(x_1$, $x_2)$, $\overline{x}\}$.
    \end{enumerate}
  \end{proof}
  \section{Билет 18 (Замыкание и замкнутые классы в $P_k$)}
  \begin{definition}
    Замыканием множества $F$ в $P_k$ называется множество всех функций, которые можно получить суперпозициями из $F$.
  \end{definition}
  \begin{definition}
    Если $[F]=F$, то множество $F$ называется замкнутым.
  \end{definition}
  \begin{definition}
    Пусть $Q\subseteq E_k$. Множество функций $T_Q$ : $\forall \alpha_1$, $\ldots$, $\alpha_n\in Q$ $f(\alpha_1$, $\ldots$, $\alpha_n)\in Q$, называется функцией, сохраняющей множество $Q$.
  \end{definition}
  \begin{example}
    \begin{enumerate}
      \item $P_k$
      \item $T_Q$
    \end{enumerate}
  \end{example}
  \begin{theorem}
    Класс $T_Q$ замкнут.
  \end{theorem}
  \begin{proof}
    \begin{enumerate}
      \item Операция подстановки переменных:\\
      Пусть функция $f(x_1$, $\ldots$, $x_n)$ сохраняет множество $Q$, тогда $g(x_1$, $\ldots$, $x_n)=f(x_{i_1}$, $\ldots$, $x_{i_n})$ тоже будет сохранять множество $Q$, так как при перестановке одинаковых переменных ничего не поменяется.
      \item Операция подстановки функции в функцию:\\
      Пусть функции $f(x_1$, $\ldots$, $x_n)$ и $g(x_1$, $\ldots$, $x_m)$ сохраняют множесво $Q$, тогда $h(x_1$, $\ldots$, $x_{m+n-1})=f(x_1$, $\ldots$, $x_{n-1}$, $g(x_n$, $\ldots$, $x_{m+n-1}))$, так как функция $g$ сохраняет множество $Q$ $\Longrightarrow$ все переменные $f$ принимают одно и то же значение, а значит, и функция $h$ будет сохранять множество $Q$.
      \item Операция добавления или удаления фиктивных переменных:\\
      Очевидно.
    \end{enumerate}
  \end{proof}
  \section{Билет 19 (Последовательность Кузнецова и алгоритм, связанный с ней)}
  \begin{definition}
    Определим глубину формулы через индукцию по определению формулы в сигнатуре:\\
    База индукции: пусть $x_i$ - символ переменной, тогда глубина формулы $x_i$ равна 0.\\
    Пусть $s\in S$, $f=\Sigma(s)$, Ф$_1$, $\ldots$, Ф$_n$ - формулы в сигнатуре, причём $m$ - наибольшая из глубин этих формул, тогда глубина формулы $s($Ф$_1$, $\ldots$, Ф$_n)$ равна $m+1$.
  \end{definition}
  \begin{theorem}
    Существует алгоритм, распознающий полноту конечных систем функций в $P_k$. Он заключается в построении последовательности Кузнецова и проверке вхождения в её предел функции Вебба.
  \end{theorem}
  \begin{proof}
    Пусть $F\subseteq P_k$ - конечное множество функций в $P_k$, $\Sigma:S\to F$ - сигнатура. Рассмотрим последовательность $G_1$, $G_2$, $\ldots$ такую, что $G_i$ - множество функций, определяемых невырожденными формулами в сигнатуре $\Sigma$, содержащими только переменные $x_1$, $x_2$ и имеющими глубину, меньшую $i$. Данную последовательность назовём последовательностью Кузнецова. Так как все формулы в соответствующем множестве $G_i$ имеют глубину, меньшую $i$ $\Longrightarrow$ $\varnothing\subseteq G_1\subseteq\ldots$. Так как число функций в $P_k$ от двух переменных равно $k^{k^2}$ $\Longrightarrow$ $|G_i|\leqslant k^{k^2}$ $\Longrightarrow$ последовательность Кузнецова стабилизируется на некотором шаге $G_m=G$, $G$ называется пределом последовательности Кузнецова. Свяжем с каждой функцией из $G_i$ некоторую формулу Ф$'_j$, содержащую только переменные $x_1$, $x_2$ и имеющую глубину, меньшую $i$. Рассмотрим функцию $f\in G_{i+1}\setminus G_i$, она определяется формулой Ф$=s($Ф$_1$, $\ldots$, Ф$_n$), где формулы Ф$_1$, $\ldots$, Ф$_n$ либо являются переменными, либо определяют некоторые функции в $G_i$, но эти функции мы уже определили формулами Ф$'_j$, тогда елси заменить в формуле Ф формулы Ф$_j$ на Ф$'_j$, то мы получим формулу Ф$'$, определяющую ту же самую функцию $f$ $\Longrightarrow$ для получения из $G_i$ $G_{i+1}$ достаточно рассмотреть все формулы Ф'$=s($Ф$'_1$, $\ldots$, Ф$'_n)$. Значит данную последовательность имеет смысл проверять до первого совпадения $G_i$ и $G_{i+1}$.\\
    \begin{lemma}
      Система функций в $P_k$ полна тогда и только тогда, когда в предел последовательности входит функция Вебба.
    \end{lemma}
    \begin{proof}
      $\underline{\Longleftarrow}$ Пусть $V_k(x_1$, $x_2)\in G$, тогда функция Вебба получается суперпозициями из функций данной системы $\Longrightarrow$ эта система полна.\\
      $\underline{\Longrightarrow}$ Пусть система функций $F$ полна, тогда функция Вебба определяется некоторой формулой в сигнатуре $\Sigma$, существенно зависящей от двух переменных и имеющей глубину, меньшую $i$, то есть $V_k\in G_i$, переобозначим переменные так, чтобы существенными стали только переменные $x_1$, $x_2$, а все остальные несущественные переменные заменим на $x_1$, тогда эта формула определяет функцию из $G_{i+1}$ (так как она получена из формул, сопоставленных функциям из $G_i$) $\Longrightarrow$ $V_k\in G_{i+1}$ $\Longrightarrow$ $V_k\in G$.
    \end{proof}
  \end{proof}
  \section{Билет 20 (Теорема о существовании конечной полной подсистемы в полной системе в k-значной логике)}
  \begin{theorem}
    Из любой полной системы функций в $P_k$ можно выделить конечную полную подсистему.
  \end{theorem}
  \begin{proof}
    Пусть $F$ - полная система в $P_k$, тогда суперпозициями из $F$ можно получить функцию Вебба, то есть полную подсистему, а так как она получается суперпозициями из конечного числа функций, значит, подсистема из этих функций конечна и полна.
  \end{proof}
  \section{Билет 21 (Селекторные функции, класс функций, сохраняющих множество $K$, его замкнутость )}
  \begin{definition}
    Функции $g_i^p(x_1$, $\ldots$, $x_p)=x_i$, где $i=\overline{1,p}$,  называются селекторными функциями.
  \end{definition}
  \begin{definition}
    Пусть $K$ - множество функций $h(x_1$, $\ldots$, $x_p)$, зависящих от $p$ переменных и содержащих все селекторные функции от $p$ переменных. Если для любых функций $h_1(x_1$, $\ldots$, $x_p)$, $\ldots$, $h_n(x_1$, $\ldots$, $x_p)$ функция $f(h_1$, $\ldots$, $h_n)\in K$, то скажем, что функция $f$ сохраняет множество $K$.
  \end{definition}
  Рассмотрим класс функций в алгебре логики, сохраняющих множество $K=\{x$, $\overline{x}\}$, то есть в $K$ входят функции $\{x^{\sigma}\}$, где $\sigma=\{0, 1\}$. Тогда функция $f$ сохраняет $K$, если $f(x_1^{\sigma_1}$, $\ldots$, $x_n^{\sigma_n})=x^{\sigma}$, то есть
  $$\begin{cases}
    f(1^{\sigma_1}, \ldots, 1^{\sigma_n})=1^{\sigma}=\sigma=f(\sigma_1, \ldots, \sigma_n)\\
    f(0^{\sigma_1}, \ldots, 0^{\sigma_n})=0^{\sigma}=\overline{\sigma}=f(\overline{\sigma}_1, \ldots, \overline{\sigma}_n)
  \end{cases}$$\\
  $\Longrightarrow$ $f(\sigma_1$, $\ldots$, $\sigma_n)=\overline{f(\overline{\sigma}_1 \ldots \overline{\sigma}_n)}$, то есть мы получили класс $S$ самодвойственных функций.
  \begin{definition}
    Множество всех функций, сохраняющих множество $K$, называется классом сохранения множества $K$. Данный класс обозначим $U(K)$.
  \end{definition}
  \begin{theorem}
    Класс $U(K)$ замкнут.
  \end{theorem}
  \begin{proof}
    \begin{enumerate}
      \item Опреация подстановки переменных:\\
      Пусть функция $f$ сохраняет множество $K$, тогда функция $g(x_1$, $\ldots$, $x_n)=f(x_{i_1}$, $\ldots$, $x_{i_n})$, $f(h_1(x_1$, $\ldots$, $x_p)$, $\ldots$, $h_n(x_1$, $\ldots$, $x_p))\in K$ $\forall h_1$, $\ldots$, $h_n\in K$, а значит, $f(h_{i_1}(x_1$, $\ldots$, $x_p)$, $\ldots$, $h_{i_n}(x_1$, $\ldots$, $x_p))\in K$ $\Longrightarrow$ $g$ сохраняет множество $K$.
      \item Операция подстановки функции в функцию:\\
      Аналогично с предыдущим пунктом.
      \item Операция добавления или удаления фиктивных переменных:\\
      Пусть $f(x_1$, $\ldots$, $x_{i-1}$, $x_{i+1}$, $\ldots$, $x_n)$ сохраняет множество $K$,  $g(x_1$, $\ldots$, $x_{i-1}$, $0$, $x_{i+1}$, $\ldots$, $x_n)$ получена из $f$ добавлением фиктивной переменной, тогда $g$ будет сохранять множество $K$, так как при подстановке функций $h_j(x_1$, $\ldots$, $x_p)$ в функцию $g$ мы получим $g(h_1(x_1$, $\ldots$, $x_p)$, $\ldots$, $h_{i-1}(x_1$, $\ldots$, $x_p)$, $0$, $h_{i+1}(x_1$, $\ldots$, $x_p)$, $\ldots$, $h_n(x_1$, $\ldots$, $x_p))=f(h_1(x_1$, $\ldots$, $x_p)$, $\ldots$, $h_{i-1}(x_1$, $\ldots$, $x_p)$, $h_{i+1}(x_1$, $\ldots$, $x_p)$, $\ldots$, $h_n(x_1$, $\ldots$, $x_p))\in K$.
    \end{enumerate}
    Значит, суперпозициями мы не сможем получить функцию, не сохраняющую множество $K$.
  \end{proof}
  \section{Билет 22 (Лемма о неполноте системы $F$, если $F\subseteq U(K)$ и $V_k\notin U(K)$)}
  \begin{theorem}
    Класс функций $U(K)$ не является полным, если множество $K$ не содержит функцию Вебба.
  \end{theorem}
  \begin{proof}
    Пусть $F$ - множество функций, сохраняющих множество $K$, содержащее все селекторные функции и не содержащее функцию Вебба, $\Sigma$ - сигнатура для $F$, тогда рассмотрим последовательность Кузнецова $G_1$, $G_2$, $\ldots$ и докажем по индукции, что $G\subseteq K$.\\
    База индукции: $\varnothing\subseteq K$.\\
    Пусть $G_i\subseteq K$, докажем для $G_{i+1}$. Рассмотрим функцию $h\in G_{i+1}\setminus G_i$, она задаётся формулой $s(A_1$, $\ldots$, $A_n)$, где $\Sigma(s)\in F$, $A_j$ либо является функцией из $G_i$, глубина которой меньше $i$, либо является переменной $x_1$, либо является переменной $x_2$. В первом случае $A_j$ задаёт некоторую функцию $h_j(x_1$, $x_2)\in G_i$, во втором случае $h_j(x_1$, $x_2)=g_1^2(x_1$, $x_2)$, в третьем случае $h_j(x_1$, $x_2)=g_2^2(x_1$, $x_2)$. Так как $G_i\subseteq K$, значит, $\forall j$ $A_j\in K$, а значит, $G_{i+1}\subseteq K$. А так как $K$ не содержит функцию Вебба, по критерию $K$ неполно.
  \end{proof}
  \section{Билет 23 (Существование для неполной системы $F$ множества $K$ такого, что $V_k\notin K$ и $F\subseteq U(K)$)}
  \begin{theorem}
    Если система функций $F$ в $k$-значной логике не является полной, то в $P_k$ существует множество $K$ функций от двух переменных, содержащее обе селекторные функции и не содержащее функцию Вебба, такое, что $F\subseteq  U(K)$.
  \end{theorem}
  \begin{proof}
    Пусть $F$ - система функций в $k$-значной логике, $\Sigma$ - сигнатура для $F$. Рассмотрим последовательность Кузнецова $G_1$, $G_2$, $\ldots$. Пусть $G_m=G_{m+1}$, так как $F$ неполна $\Longrightarrow$ $V_k\notin F$. Пусть $K=G_m\cup \{g_1^2$, $g_2^2\}$. Так как $V_k$ не является селекторной функцией и $V_k\notin G_m$ $\Longrightarrow$ $V_k\notin K$. Пусть $f(x_1$, $\ldots$, $x_n)\in F$, $h_1$, $\ldots$, $h_n\in K$. Рассмотрим $f(h_1(x_1$, $x_2)$, $\ldots$, $h_n(x_1$, $x_2))$. Пусть $s$ - символ для $f$ в сигнатуре $\Sigma$. Если $h_j(x_1$, $x_2)\in G_m$, то эта функция определяется в сигнатуре $\Sigma$ формулой $A_j$, глубина которой меньше $m$, если $h_j(x_1$, $x_2)=g_1^2(x_1$, $x_2)$, то возьмём в качестве $A_j$ $x_1$, если $h_j(x_1$, $x_2)=g_2^2(x_1$, $x_2)$, то возьмём в качестве $A_j$ $x_2$. Тогда функция $f(h_1(x_1$, $x_2)$, $\ldots$, $h_n(x_1$, $x_2))$ определяется формулой $s(A_1$, $\ldots$, $A_n)$, глубина которой меньше $m+1$, значит, функция реализующая эту формулу, $h(x_1$, $x_2)\in G_{m+1}=G_m\subseteq K$ $\Longrightarrow$ $h\in K$ $\Longrightarrow$ $F$ сохраняет множество $K$ и $F\subseteq U(K)$.
  \end{proof}
  \section{Билет 24 (Теорема Кузнецова о предполных классах в k-значной логике )}
  \begin{theorem}
    В $P_k$ можно построить замкнутые классы $M_1$, $\ldots$, $M_s$ такие, что ни один из них не содержится в других и произвольная система $F\subseteq P_k$ полна тогда и только тогда, когда $F$ не содержится ни в одном из этих классов.
  \end{theorem}
  \begin{proof}
    Рассмотрим все классы $N_1$, $\ldots$, $N_q$ вида $U(K)$, где $K$ - множество функций от двух переменных, содержащее обе селекторные функции и не содержащее функцию Вебба. По лемме 1 они замкнуты. Пусть $F\subseteq P_k$ неполна, тогда по лемме 3 существует класс $N_i$  такой, что $F\subseteq N_i$, тогда по лемме 2, множество $F$ неполно $\Longrightarrow$ полнота системы эквивалентна невключению её ни в один из классов $N_1$, $\ldots$, $N_q$, удалив из них те, которые содержатся в других, получим искомую систему классов $M_1$, $\ldots$, $M_s$.
  \end{proof}
  \section{Билет 25 (Лемма о трёх наборах)}
  \begin{definition}
    Функция $f\in P_k$ называется существенной, если она имеет больше одной существенной переменной.
  \end{definition}
  \begin{theorem}
    Пусть $f(x_1$, $\ldots$, $x_n)$ - существенная функция, принимающая $l$ значений, где $l\geqslant3$, и пусть $x_1$ - её существенная переменная, тогда существуют наборы $(\alpha$, $\alpha_2$, $\ldots$, $\alpha_n)$, $(\beta$, $\alpha_2$, $\ldots$, $\alpha_n)$, $(\alpha$, $\gamma_2$, $\ldots$, $\gamma_n)$, на которых она принимает три различных значения.
  \end{theorem}
  \begin{proof}
    Так как переменная $x_1$ является существенной, существуют значения $\alpha_2$, $\ldots$, $\alpha_n$ такие, что в следующем списке $S$: $f(0$, $\alpha_2$, $\ldots$, $\alpha_n)$, $f(1$, $\alpha_2$, $\ldots$, $\alpha_n)$, $\ldots$, $f(k-1$, $\alpha_2$, $\ldots$, $\alpha_n)$ - есть более одного значения. Рассмотрим два случая:
    \begin{enumerate}
      \item В $S$ меньше чем $l$ значений, тогда найдём набор, на котором функция $f$ принимает значение, не встречающееся в $S$, обозначим этот набор ($\alpha$, $\gamma_2$, $\ldots$ $\gamma_n)$. $f(\alpha$, $\gamma_2$, $\ldots$ $\gamma_n)\neq f(\alpha$, $\alpha_2$, $\ldots$ $\alpha_n)$, так как $f(\alpha$, $\gamma_2$, $\ldots$ $\gamma_n)\notin S$, $f(\alpha$, $\gamma_2$, $\ldots$ $\gamma_n)\neq f(\beta$, $\alpha_2$, $\ldots$ $\alpha_n)$, где набор ($\beta$, $\alpha_2$, $\ldots$ $\alpha_n)\neq f(\alpha$, $\alpha_2$, $\ldots$ $\alpha_n)$.
      \item В $S$ ровно $l$ значений, тогда существует такое $\alpha$, что $f(\alpha$, $x_2$, $\ldots$ $x_n)$ - не константа $\Longrightarrow$ $\exists (\alpha$, $\alpha_2$, $\ldots$ $\alpha_n)\neq(\alpha$, $\gamma_2$, $\ldots$ $\gamma_n)$. Так как $l\geqslant3$ $\exists\beta$ такое, что ($\beta$, $\alpha_2$, $\ldots$ $\alpha_n)\neq(\alpha$, $\alpha_2$, $\ldots$ $\alpha_n)\neq(\alpha$, $\gamma_2$, $\ldots$ $\gamma_n)$.
    \end{enumerate}
  \end{proof}
  \section{Билет 26 (Лемма о подмножестве $G_1\times\ldots\times G_n$, на котором функция принимает хотя бы $l$ значений)}
  \begin{theorem}
    Если $f(x_1$, $\ldots$, $x_n)$ - существенная функция в $P_k$, принимающая хотя бы $l$ значений, где $l\geqslant3$, тогда существуют подмножества $G_1$, $\ldots$, $G_n$ множества $E_k$ такие, что $1\leqslant|G_1|\leqslant l-1$, $\ldots$, $1\leqslant|G_n|\leqslant l-1$, причём на множестве $G_1\times\ldots\times G_n$ функция принимает хотя бы $l$ значений.
  \end{theorem}
  \begin{proof}
    Без ограничения общности будем считать, что $x_1$ - существенная переменная. По лемме о трёх наборах существуют наборы ($\alpha$, $\alpha_2$, $\ldots$ $\alpha_n)$, ($\beta$, $\alpha_2$, $\ldots$ $\alpha_n)$, ($\alpha$, $\gamma_2$, $\ldots$ $\gamma_n)$, на которых функция принимает три различных значения. Пусть остальные $l-3$ значения функция принимает на наборах $\delta_i=(\delta_{i1}$, $\ldots$ $\delta_{in})$, тогда в качестве $G_1$ выберем набор ($\alpha$, $\beta$, $\delta_{11}$, $\ldots$ $\delta_{l-3,1})$, в качестве $G_2$, $\ldots$, $G_n$ выберем наборы ($\alpha_2$, $\gamma_2$, $\delta_{12}$, $\ldots$ $\delta_{l-3,2})$, $\ldots$, ($\alpha_n$, $\gamma_n$, $\delta_{1n}$, $\ldots$ $\delta_{l-3,n})$. Каждое из $G_j$ непусто и в каждом не больше $l-1$ элемента, а значит, мы получили искомые множества.
  \end{proof}
  \section{Билет 27 (Лемма о квадрате)}
  \begin{definition}
    Четвёрка наборов $\{(\alpha_1$, $\ldots$, $\alpha_{i-1}$, $x$, $\alpha_{i+1}$, $\ldots$, $\alpha_{j-1}$, $y$, $\alpha_{j+1}$, $\ldots$, $\alpha_n)|x\in \{p_1$, $p_2\}$, $y\in\{q_1$, $q_2\}\}$ называется квадратом в $P_k$.
  \end{definition}
  \begin{theorem}
    Пусть $f(x_1$, $\ldots$, $x_n)$ - существенная функция в $P_k$, принимающая $l$ значений, причём $l\geqslant3$. Тогда сущетвует квадрат, на котором $f$ принимает некоторое своё значение ровно в одной точке.
  \end{theorem}
  \begin{proof}
    Без ограничения общности будем считать, что $x_1$ - существенная переменная, тогда по лемме о трёх наборах существуют наборы $(\alpha$, $\alpha_2$, $\ldots$, $\alpha_n$), ($\beta$, $\alpha_2$, $\ldots$, $\alpha_n$), ($\alpha$, $\gamma_2$, $\ldots$, $\gamma_n)$, на которых функция принимает три различных значения. Рассмотрим последовательность пар:\\
    $\begin{matrix}
      P_1=\{(\alpha, \alpha_2, \ldots, \alpha_n), (\beta, \alpha_2, \ldots, \alpha_n)\}\\
      P_2=\{(\alpha, \gamma_2, \alpha_3, \ldots, \alpha_n), (\beta, \gamma_2, \alpha_3, \ldots, \alpha_n)\}\\
      \vdots\\
      P_i=\{(\alpha, \gamma_2, \ldots, \gamma_i, \alpha_{i+1}, \ldots, \alpha_n), (\beta, \gamma_2, \ldots, \gamma_i, \alpha_{i+1}, \ldots, \alpha_n)\}\\
      \vdots\\
      P_n=\{(\alpha, \gamma_2, \ldots, \gamma_n), (\beta, \gamma_2, \ldots, \gamma_n)\}
    \end{matrix}$\\
    На наборах пары $P_1$ функция принимает значения $a$ и $b$, на первом наборе пары $P_n$ функция принимает значение, отличное от $a$ и $b$, а на втором наборе она может принимать одно из значений либо $a$, либо $b$, но не оба. Значит, существуют пары $P_i$ и $P_{i+1}$ такие, что на наборах пары $P_i$ функция приниает оба значения $a$ и $b$, а на наборах пары $P_{i+1}$ функция не принимает одно из этих значений. Заметим, что наборы из пар $P_i$ и $P_{i+1}$ образуют квадрат в $P_k$, причём одно из значений $a$ и $b$, которое функция не принимает на наборах пары $P_{i+1}$, и будет искомым значением, которое функция принимает ровно в одной точкче.
  \end{proof}
  \section{Билет 28 (Теорема Слупецкого, теорема Яблонского, теорема Мартина)}
  \begin{theorem}
    Пусть $F$ - система функций в $P_k$, где $k\geqslant3$, содержащая все функции одной переменной. Тогда для полноты $F$ необходимо и достаточно, чтобы она содержала существенную функцию, принимающую все $k$ значений.
  \end{theorem}
  \begin{proof}
    $\underline{\Longrightarrow}$ Пусть $F$ не содержит функцию, принимающую все $k$ значений, тогда проверим, сможем ли мы получить суперпозициями из $F$ функцию, принимающую все $k$ значений:\\
    \begin{enumerate}
      \item Операция подстановки переменных:\\
      Пусть $f\in F$ и $g(x_1$, $\ldots$, $x_n)=f(x_{i_1}$, $\ldots$, $x_{i_n}$), тогда, если $f$ является существенной и не принимает все $k$ значений, то и $g$ будет существенной функцией, не принимающей все $k$ значений. Если $f$ не является существенной, то есть $f$ имеет единственную существенную переменную $x_j$, но тогда и функция $g$ будет иметь единственную существенную переменную $x_{i_j}$.
      \item Операция подстановки функции в функцию:\\
      Пусть $f$, $g\in F$ и $h(x_1$, $\ldots$, $x_{m+n-1})=f(x_1$, $\ldots$, $x_{n-1}$, $g(x_n$, $\ldots$, $x_{m+n-1}))$, тогда если функции $f$ и $g$ являются существенными и не принимают все $k$ значений, то и $h$ их не принимает. Если функция $f$ принимает все $k$ значений и не является существенной, то она имеет одну существенную переменную $x_j$. Если $j\neq n$, то она будет единственной сущесвенной переменной у функции $h$. Пусть $j=n$, тогда если функция $g$ не принимает все $k$ значений, то и функция не будет принимать все $k$ значений. Если $g$ принимает все $k$ значений, то она не является существенной, а значит, содержит ровно одну существенную переменную $\Longrightarrow$ $h$ тоже содержит ровно одну существенную перемнную, то есть функция $h$ не является сущетвенной.
      \item Операция добавления или удаления несущественных переменных:\\
      Если функция не была существенной, то при добавлении несущественной переменной, она не может стать существенной. Если функция была существенной и не принимала все $k$ значений, то при добавлении фиктивной переменной, количество принимаемых ею значений не изменится, так как она от этой переменной существенно не зависит.
    \end{enumerate}
    Значит, суперпозициями мы не сможем получить существенную функцию, принимающую все $k$ значений, то есть система $F$ неполна.\\
    $\underline{\Longleftarrow}$ Пусть $F$ имеет существенную функцию $f(x_1$, $\ldots$, $x_n)$, принимающую все $k$ значений, тогда по лемме о квадрате существует квадрат $\{(\alpha_1$, $\ldots$, $\alpha_{i-1}$, $x$, $\alpha_{i+1}$, $\ldots$, $\alpha_{j-1}$, $y$, $\alpha_{j+1}$, $\ldots$, $\alpha_n)|x\in \{p_1$, $p_2\}$, $y\in \{q_1$, $q_2\}\}$, на котором функция $f$ принимает некоторое своё значение $a$ ровно в одной точке.
    $\phi_0(x)=\begin{cases}
      0, \textup{ при } x=a\\
      1, \textup{ при } x\neq a
    \end{cases}$
    Так как $\phi_0$ зависит от одной переменной, то $\phi_0\in F$. Пусть функция $g$ имеет вид: $g(x_1$, $x_2)=\phi_0(f(\{(\alpha_1$, $\ldots$, $\alpha_{i-1}$, $x_1$, $\alpha_{i+1}$, $\ldots$, $\alpha_{j-1}$, $x_2$, $\alpha_{j+1}$, $\ldots$, $\alpha_n))$. Так как функция $g$ на квадрате $\{(p_1$, $q_1$), $(p_1$, $q_2$), $(p_2$, $q_1$), $(p_2$, $q_2$)\} принимает значения 0 и 1, причём 0 она принимает ровно в одной точке. Без ограничения общности будем считать, что $g(p_1$, $q_1)=0$, а в остальных точках квадрата функция принимает значение 1.\\
    База индукции: пусть $\phi_1(x)=\begin{cases}
      p_1, \textup{ при } x=0\\
      p_2, \textup{ при } x\neq 0
    \end{cases}$, а $\phi_2(x)=\begin{cases}
      q_1, \textup{ при } x=0\\
      q_2, \textup{ при } x\neq 0
    \end{cases}$, обе этих функции зависят от одной переменной. Пусть $g'(x_1$, $x_2)=g(\phi_1(x_1)$, $\phi_2(x_2))=\begin{cases}
      0, \textup{ при } x_1=x_2=0\\
      1, \textup{ иначе }
    \end{cases}$, тогда эта функция совпадает с дизъюнкцией, если её аргументы ограничить множеством $\{0$, $1\}$, обозначим эту функцию $x_1\vee_{01}x_2$. Так как функция $j_i(x)=\begin{cases}
      1, \textup{ при } x=i\\
      0, \textup{ при } x\neq i
    \end{cases}$ зависит от одной переменной $\Longrightarrow$ $j_i(x)\in F$, причём $j_0(x)=\overline{x}$, если ограничить $x$ на множество $\{0$, $1\}$. Пусть $g''(x_1$, $x_2)=j_0(j_0(x_1)\vee_{01}j_0(x_2))$, она совпадает с конъюнкцией, если её ограничить на множество $\{0$, $1\}\times\{0$, $1\}$. Данную функцию обозначим $x_1\wedge_{01}x_2$.\\
    Пусть $h(x_1$, $\ldots$, $x_n)\in P_k$, которая принимает только значения 0 и 1. Тогда благодаря совершенной д.н.ф. имеем: $$h(x_1, \ldots,x_n)=\bigvee\limits_{\sigma_1,\ldots, \sigma_n}j_{\sigma_1}(x_1)\wedge_{01}\ldots\wedge_{01}j_{\sigma_n}(x_n)\wedge_{01}h(\sigma_1, \ldots,\sigma_n).$$ Так как константы $h(\sigma_1, \ldots,\sigma_n)$ принадлежат $F$, то данная функция получена суперпозициями над $F$. Если функция $h(x_1$, $\ldots$, $x_n)$ - функция из $P_k$, принимающая только какие-то два значения $a$ и $b$, то рассмотрим функцию $h'(x_1$, $\ldots$, $x_n)=\begin{cases}
      0, \textup{ если }h(x_1, \ldots,x_n)=a\\
      1, \textup{ иначе }
    \end{cases}$ и функцию $\psi(x)=\begin{cases}
      a, \textup { если }x=0\\
      b, \textup { иначе}
    \end{cases}$, тогда $h(x_1$, $\ldots$, $x_n)=\psi(h'(x_1$, $\ldots$, $x_n))$ $\Longrightarrow$ любая функция из $P_k$, принимающая не более двух значений, получается суперпозициями над $F$.\\
    Шаг индукции: пусть все функции $k$-значной логики, принимающие не более чем $l-1$ значение, получаются суперпозициями над $F$, докажем, что любая функция, принимающая $l$ значений, тоже будет получаться суперпозициями над $F$. Рассмотрим существенную функцию $f(x_1$, $\ldots$, $x_n)$, принимающую все $k$ значений, тогда по лемме 2 существуют подмножества $G_1$, $\ldots$, $G_n$ множества $E_k$ такие, что $1\geqslant|G_1|\geqslant l-1$, $\ldots$, $1\geqslant|G_n|\geqslant l-1$, причём на множестве $G_1\times\ldots\times G_n$ функция принимает хотя бы $l$ значений. Обозначим эти $l$ значений $a_1$, $\ldots$, $a_n$ и рассмотрим наборы из $G_1\times\ldots\times G_n$, на которых $f$ принимает эти $l$ значений:
    $$\begin{matrix}
      a_1=f(a_{11},...,a_{1n})\\
      \vdots\\
      a_l=f(a_{l1},...,a_{ln})
    \end{matrix}$$\\
    Пусть функция $h(x_1$, $\ldots$, $x_m)$ принимает только значения $a_1$, $\ldots$, $a_l$, тогда рассмотрим произвольный набор $(\alpha_1$, $\ldots$, $\alpha_m)$ значений переменных $x_1$, $\ldots$, $x_m$. На этом наборе функция $h$ принимает некоторое значение $a_i$. Зададим функции $\psi_1(\alpha_1$, $\ldots$, $\alpha_m$), $\ldots$, $\psi_n(\alpha_1$, $\ldots$, $\alpha_m$), равные $a_{i1}$, $\ldots$, $a_{in}$. Тогда $f(\psi_1(\alpha_1$, $\ldots$, $\alpha_m$), $\ldots$, $\psi_n(\alpha_1$, $\ldots$, $\alpha_m))=f(a_{i1}$, $\ldots$, $a_{in})=a_i$ $\Longrightarrow$ значения функций $\psi_1$, $\ldots$, $\psi_n$ определены для всех значений их аргументов, при этом $h(x_1$, $\ldots$, $x_m)\equiv f(\psi_1(x_1$, $\ldots$, $x_m$), $\ldots$, $\psi_n(x_1$, $\ldots$, $x_m$)) по построению. А так как все функции $\psi_i$ принимают только значения из множества $\{a_{1i}$, $\ldots$, $a_{li}\}$, которые принадлежат множеству $G_i$ $\Longrightarrow$ данные функции принимают не более чем $l-1$ значение, а по предположению индукции эти функции получаются суперпозициями над $F$, а значит, и функция $h$ получается суперпозициями над $F$. То есть мы получили, что любая функция из $P_k$, принимающая только значения $a_1$, $\ldots$, $a_l$, получается суперпозициями над $F$. Если $l=k$, то все функции из $P_k$ получаются суперпозициями над $F$.\\
    Пусть $l<k$, тогда рассмотрим произвольную функцию $h\in P_k$, принимающую не более чем $l$ значений. Пусть эти значения пренадлежат списку $b_1$, $\ldots$, $b_l$. Рассмотрим $h'=
    \begin{cases}
      a_i, \textup{ на некотором набооре}\\
      b_i, \textup{ на остальных наборах}
    \end{cases}$ и функцию\\ $$\psi=\begin{cases}
      b_i, \textup{ если значение функции } h'=a_i\\
      0, \textup{ на остальных наборах}
    \end{cases}$$ $\Longrightarrow$ $h(x_1$, $\ldots$, $x_m)=\psi(h'(x_1$, $\ldots$, $x_m))$, а так как $h'$ и $\psi$ получаются суперпозициями над $F$, значит, функция $h$ тоже. Значит, если $l=k$, то система $F$ полна.
  \end{proof}
  \begin{theorem}
    Пусть система $F$ $k$-значной логики, где $k\geqslant3$ содержит все функции однлй переменной, принимающие не более $k-1$ значения, тогда для её полноты необходимо и достаточно, чтобы она содержала существенную функцию, принимающую все $k$ значений.
  \end{theorem}
  \begin{proof}
    Данная теорема следует из доказательства теоремы Слупецкого, так как в этом доказательстве для докакзательства случая $l=k$ использовались не все одноместные функции, а только те функции, которые принимают не более чем $k-1$ значение.
  \end{proof}
  \begin{theorem}
    Функция $f(x_1$, $\ldots$, $x_n)\in P_k$ образует полную систему (является шефферовой) тогда и только тогда, когда она содержит все функции одной переменной, принимающие не более чем $k-1$ значение.
  \end{theorem}
  \begin{proof}
    $\underline{\Longrightarrow}$ Следует из теоремы Яблонского.\\
    $\underline{\Longleftarrow}$ Пусть $f$ порождает все функции одной переменной, принимающие не более чем $k-1$ значение, в частности она порождает все константы $\Longrightarrow$ $f$ принимает все $k$ значений. Предположим, что $f$ не является существенной, тогда она имеет ровно одну существенную переменную. Класс данных функций обозначим $M(k)$\\
    \begin{enumerate}
      \item Операция подстановки переменных:\\
      Пусть $g(x_1$, $\ldots$, $x_n)=f(x_{i_1}$, $\ldots$, $x_{i_n})$, тогда если $f$ имела единственную существенную переменную $x_j$, то $g$ также будет иметь единственную существенную переменную $x_{i_j}$. Если $f$ принимала все $k$ значений, то, варьируя значение переменной, получим все $k$ значений, но тогда $g\in M(k)$.
      \item Операция подстановки функции в функцию:\\
      Пусть $h(x_1$, $\ldots$, $x_{n+m-1})=f(x_1$, $\ldots$, $x_{n-1}$, $g(x_n$, $\ldots$, $x_{m+n-1}))$. Если $f$ имеют единственную существенную переменную $x_i$, где $i<n$, то и $h$ имеет единственную существенную переменную $x_i$ и, варьируя значение переменной $x_i$ получим все $k$ значений. Если единственной существенной переменной $f$ является переменная $x_n$, то рассморим единственную существенную переменную функции $g$ $\Longrightarrow$ она является единственной существенной переменной и для $h$, аналогично, варьируя значения данной переменной, получим все $k$ значений функции $g$, а значит, и функции $h$.
      \item Операция добавления или удаления фиктивных переменных:\\
      Добавление или удаление фиктивных переменных не повлияет на количество существенных переменных и число значений, которые принимает функция.
    \end{enumerate}
    $\Longrightarrow$ из функции $f$ можно получить суперпозициями только одноместные функции, принимающие все $k$ значений, то есть нельзя получить константу $\Longrightarrow$ $f$ является существенной, тогда по теореме Яблонского она образует полную систему.
  \end{proof}
  \section{Билет 29 (Теорема Янова)}
  \begin{theorem}
    В $\forall k\geqslant3$ в $P_k$ существует замкнутый класс, не имеющий базиса.
  \end{theorem}
  \begin{proof}
    Рассмотрим последовательность функций $$f_0=0, \ldots, f_i(x_1, \ldots, x_i)=\begin{cases}
      1, \textup{ если } x_1=x_2=\ldots=x_i=2\\
      0, \textup { иначе}
    \end{cases}$$. Пусть $M$ - замыкание множества $\{f_0$, $\ldots$\}, тогда рассмотрим операции суперпозиции на множестве $M$.\\
    \begin{enumerate}
      \item Операция подстановки переменных:\\
      Пусть $g(x_1$, $\ldots$, $x_n)=f(x_{i_1}$, $\ldots$, $x_{i_n})$, тогда, если $f$ получалась из некоторой функции $f_j$ добавлением фиктивной переменной, то и $g$ получалась из некоторой функции $f_m$, где $n\leqslant i$, добавлением фиктивной переменной.
      \item Операция добавления функции в функцию:\\
      Пусть функции $f$ и $g$ получаются из некоторых функций $f_i$ добавлением фиктивных переменных и $h(x_1$, $\ldots$, $x_{n+m-1})=f(x_1$, $\ldots$, $x_{n-1}$, $g(x_n$, $\ldots$, $x_{m+n-1}))$, тогда функция $h$ тождественно равна 0, так как $g$ не принимает значение 2, то есть $h$ получается добавлением фиктивных переменных к ыункции $f_0$.
      \item Операция добавления или удаления фиктивных переменных:\\
      Очевидно.
    \end{enumerate}
    Предположим, что данный класс имеет базис, тогда:
    \begin{enumerate}
      \item Если в базисе $B$ есть хотябы две различные функции, которые получаются из $f_i$ и $f_j$ соотвественно, тогда одна из этих функций получается из другой сначала отождествлением некоторых переменных, а затем добавлением фиктивных переменных $\Longrightarrow$ это не базис.
      \item Если в базисе имеется единственная функция, получаемая из некоторой функции $f_i$ добавлением фиктивных переменных, тогда из неё суперпозициями можно получить только функции $f_j$, где $j\leqslant i$, а значит, нельзя получить функцию $f_{i+1}$. 
    \end{enumerate}
  \end{proof}
  \section{Билет 30 (Теорема Мучника)}
  \begin{theorem}
    $\forall k\geqslant3$, в $P_k$ существует замкнутый класс, имеющий счётный базис.
  \end{theorem}
  \begin{proof}
    Рассмотрим систему функий $f_i(x_1, \ldots, x_i)=$ $$=\begin{cases}
      1, \textup{ если в наборе есть ровно одна единица, а все остальные элементы равны 2}\\
      0, \textup{ иначе}
    \end{cases}$$ $i=2$, $3$, $\ldots$, обозначим $M$ множество $\{f_2$, $f_3$, $\ldots\}$.\\
    Предположим, что какая-то функция $f_m$ выражается через остальные функции и рассмотрим сигнатуру $\Sigma$ в $\{f_2$, $f_3$, $\ldots$, $f_{m-1}$, $f_{m+1}$, $\ldots$\}, тогда существует невырожденная формула Ф, определяющая относительно переменных $x_1$, $\ldots$, $x_m$ функцию $f_m$. Все фиктивные переменные формулы Ф заменим на $x_1$, что не изменит функции, реализуемые формулой. По определению Ф имеет вид $s(B_1$, $\ldots$, $B_r$), где $s$ - символ сигнатуры, а $B_i$ - либо переменная, либо невырожденная формула в сигнатуре. Рассмотрим произвольный набор $(\alpha_1$, $\ldots$, $\alpha_m)$ значений переменных $(x_1$, $\ldots$, $x_m$). Обозначим $\beta_i$ значение формулы $B_i$ на этом наборе, тогда $f_m(\alpha_1$, $\ldots$, $\alpha_m)=f_r(\beta_1$, $\ldots$, $\beta_r)$. Рассмотрим 3 случая:
    \begin{enumerate}
      \item Среди формул $B_i$ есть не менее двух невырожденных, тогда не менее двух значений $\beta_i$, равных 0 или 1. Значит, функция $f_r$ обращается в ноль на любом наборе $\Longrightarrow$ функция $f_m$ является тождественным нулём - противоречие.
      \item Среди формул $B_i$ есть ровно одна невырожденная, её обозначим $B_j$, тогда есть функция $B_{j'}$, являющаяся переменной, обозначим её $x_q$, так как $r\geqslant2$. Пусть $\alpha_q=1$, $\alpha_1=\ldots=\alpha_{q-1}=\alpha_{q+1}=\ldots=\alpha_m=2$, тогда $f_m(\alpha_1$, $\ldots$, $\alpha_m)=1$, но так как $\beta_{j'}=1$ и $\beta_j$ равно либо 0, либо 1, то $f_r(\beta_1$, $\ldots$, $\beta_r)=0$, а значит, и $f_m=0$ - противоречие.
      \item Все формулы являются переменными, тогда $r>m$ так как все переменные функции $f_m$ являются существенными $\Longrightarrow$ $\exists i$ и $j$ : $i\neq j$ и $B_i$ и $B_j$ - одна и та же переменная, обозначим её $x_q$. Пусть $\alpha_q=1$, $\alpha_1=\ldots=\alpha_{q-1}=\alpha_{q+1}=\ldots=\alpha_m=2$, тогда $f_m(\alpha_1$, $\ldots$, $\alpha_m)=1$, но $\beta_i=\beta_j=1$, а значит, $f_r(\beta_1$, $\ldots$, $\beta_r)=0=f_m$ - противоречие.
    \end{enumerate}
    Таким образом мы получили, что $\forall f_m$ не выражается через остальные функции, а так как система $\{f_2$, $f_3$, $\ldots\}$ полна в $M$, имеем, что эти функции образуют базис в $M$.
  \end{proof}
  \section{Билет 31 (Теорема о представлении функций k-значной логики полиномами)}
  \begin{theorem}
    Система полиномов по модулю $k$ полна в $P_k$ тогда и только тогда, когда $k$ - простое число.
  \end{theorem}
  \begin{proof}
    Рассотрим функцию $f(x_1$, $\ldots$, $x_n)=\sum\limits_{\sigma_1,\ldots,\sigma_n}j_{\sigma_1}(x_1)\cdot\ldots\cdot j_{\sigma_n}(x_n)\cdot f(\sigma_1$, $\ldots$, $\sigma_n)(mod$ $k)$ (аналог совершенной д.н.ф.).\\
    $j_{\sigma}(x)=j_0(x-\sigma)$ $\Longrightarrow$ функцию $j_{\sigma}(x)$ представима полиномом по модулю $k$ тогда и только тогда, когда функция $j_0$ представима полтномом по модулю $k$. Тогда рассмотрим два случая:
    \begin{enumerate}
      \item $k$ - простое число, тогда по малой теореме Ферма $x^{k-1}\equiv1(mod$ $k)$ $\forall x=\overline{1,k-1}$ $\Longrightarrow$ $j_0(x)=1-x^{k-1}(mod$ $k)$, то есть $j_0(x)$ представима полиномом по модулю $k$, а значит, система полиномов полна.
      \item $k$ - составное число, тогда $k=k_1\cdot k_2$, где $k_1\geqslant k_2>1$ - натуральные числа. Предположим, что $j_0(x)=b_0+b_1+\ldots+b_sx^s(mod$ $k)$, тогда $j_0(0)=b_0(mod$ $k)=1$, $j_0(k_1)=1+b_1k_1+\ldots+b_sk_1^s(mod$ $k)=k_1k_2n$ $\Longrightarrow$ $1=k_1(k_2n-b_1-b_2k_1-\ldots-b_sk_1^{s-1})$ $\Longrightarrow$ $k_1=1$ - противоречие.
    \end{enumerate}
  \end{proof}
  \section{Билет 32 (Основные определения, связанные с конечным автоматом)}
  \begin{definition}
    Конечный абстрактный детерминированный автомат - объект $V=(A,Q,B,\phi,\psi)$, где $A$, $Q$, $B$ - конечные множества; $\phi:Q\times A\rightarrow$; $\psi:Q\times A\rightarrow B$. Множества $A$, $Q$, $B$ называются входным алфавитом, алфавитом состояний и выходным алфавитом соответственно.
  \end{definition}
  Способы задания:\\
  \begin{enumerate}
    \item С помощью таблицы со строками $a_1$, $\ldots$, $a_m$ и столбцами $q_1$, $\ldots$, $q_n$, где на пересечении $i$-ой строки и $j$-ого столбца находится элемент $(\phi(q_j, a_i), \psi(q_j, a_i))$.
    \item С помощью диграммы мура: (см. рисунок 8 лекции на странице 3)
    \item С помощью системы уравнений $k$-значной логики.
  \end{enumerate}
  \begin{definition}
    Слово в алфавите $A$ - входное слово, в алфавите $B$ - выходное слово, в алфавите $Q$ - слово состояний.
  \end{definition}
  Обозначения: $\Lambda$ - пустое слово, $|\alpha|$ - длина слова $\alpha$, $X^*$ - множество слов в алфавите $X$, $\alpha]_l$ - начало слова $\alpha$, имеющее длину $l$.
  \begin{definition}
    База индукции: $\phi(q, \Lambda)=q$\\
    Пусть $\phi(q,\alpha)$ - состояние автомата, появляющееся при подаче на вход слова $\alpha$, $a$ - буква алфавита, тогда $\phi(q, \alpha a)=\phi(\phi(q,\alpha),a)$ - состояние автомата при подаче на вход слова $\alpha a$.\\
    Так как для функции $\psi$ нужен, чтобы был хотя бы какой-нибудь символ на вход $\Longrightarrow$ сразу перейти к шагу индукции.
    Пусть $\psi(q,\alpha)$ - значение выходного символа при подаче на вход слова $\alpha$, $a$ - буква алфавита, тогда $\psi(q,\alpha a)=\psi(\psi(q,\alpha),a)$ - выходной символ при подаче на вход слова $\alpha a$.
  \end{definition}
  \begin{properties}
    \begin{enumerate}
      \item $\phi(q,\alpha_1\alpha_2)=\phi(\phi(q,\alpha_1),\alpha_2)$
      \item $\psi(q,\alpha_1\alpha_2)=\psi(\phi(q,\alpha_1),\alpha_2)$
      \item $\overline{\psi}(q,\alpha_1\alpha_2)=\overline{\psi}(q,\alpha_1)\overline{\psi}(\phi(q,\alpha_1),\alpha_2)$
    \end{enumerate}
  \end{properties}
  \section{Билет 33 (Инициальный конечный автомат и функции, реализуемые им)}
  \begin{definition}
    Инициальный конечный автомат - объект $V_{q_1}=(A$, $Q$, $B$, $\phi$, $\psi$, $q_1)$, где $A$, $Q$, $B$ - конечные множества, $\phi :Q\times A\rightarrow Q$, $\psi : Q\times A \rightarrow B$, $q_1\in Q$.
  \end{definition}
  \begin{definition}
    Конечно-автоматная функция - функция, заданная инициальным конечным автоматом. Она задана на $A^*$ и определена равенством $f(\alpha)=\overline{\psi}(q_1,\alpha)$.
  \end{definition}
  $i$-ую букву слова $\alpha$ обозначим $\alpha(i)$, так как слово является функцией, определённой на отрезке натурального ряда.
  \begin{definition}
    Канонические уравнения - это соотношения вида:
    $$\begin{matrix}
      \kappa(1)=q_1\\
      \kappa(t+1)=\phi(\kappa(t), \alpha(t))\\
      \beta(t)=\psi(\kappa(t), \alpha(t))
    \end{matrix}$$
    где слово $\kappa=\overline{\phi}(q_1, \alpha)$, слово $\beta=\overline{\psi}(q_1,\alpha)$, $\kappa=\kappa(1)\ldots\kappa(s+1)$, $\beta=\beta(1)\ldots\beta(s)$, $\alpha=\alpha(1)\ldots\alpha(s)$.
  \end{definition}
  \section{Билет 34 (Преобразование бесконечных и периодических последовательностей)}
  \begin{definition}
    Пусть $V_q(A$, $Q$,$B$, $\phi$, $\psi$, $q_1$) - инициальный конечный автомат, $\alpha=\alpha(1)\alpha(2)\ldots$ - бесконечная последовательность символов алфавита $A$. Множество таких последовательностей обозначим $A^{\infty}$. Скажем, что инициальный конечный автомат $V_q$ преобразует входную последовательность $\alpha$ в выходную последовательность $\beta$, где $\beta=\overline{\psi}(q,\alpha)$, а $i$-ый элемент последовательности $\beta$ имеет вид: $\beta(i)=\psi(q$, $\alpha(1)$, $\ldots$, $\alpha(i))$.
  \end{definition}
  \begin{definition}
    Последовательность $\alpha=\alpha(1)\alpha(2)\ldots\in A^{\infty}$ называется периодической, если $\exists\tau$ и $\tau'$ : $\forall i\geqslant\tau'+1$ $\Longrightarrow$ $\alpha(i)=\alpha(i+\tau)$. $\tau'$ - длина предпериода, $\tau$ - длина периода.
  \end{definition}
  \begin{theorem}
    Конечный инициальный автомат $V_{q_1}(A$, $Q$, $B$, $\phi$, $\psi$, $q_1)$, имеющий $n$ состояний, прелбразует любую периодическую последовательность $\alpha\in A^{\infty}$, имеющую наименьшую длину периода $\tau$, в периодическую последовательность с наименьшей длиной периода вида $\theta\cdot m$, где $\theta|\tau$, $m\in \{1$, $\ldots$, $n\}$. Если $|A|\geqslant3$ и $|B|\geqslant2$, то каждое такое значение вида $\theta\cdot m$ достигается при некотором автомате $V_q$ и соответствующей последовательности $\alpha$.
  \end{theorem}
  \begin{proof}
    Пусть $V_{q_1}(A$, $Q$, $B$, $\phi$, $\psi$, $q_1)$, $|Q|=n$, $\alpha$ - периодическая последовательность с предпериодом $\tau'$ и периодом $\tau$, тогда $\forall i\geqslant\tau'+1$ $\Longrightarrow$ $\alpha(i)=\alpha(i+\tau)$. Обозначим $\alpha_1=\alpha(1)\ldots\alpha(\tau')$, $\alpha_2=\alpha(\tau'+1)\ldots\alpha(\tau'+\tau)$, тогда $\alpha$ имеет вид: $\alpha_1\alpha_2\alpha_2\ldots$. Так как число состояний автомата равно $n$, то в последовательности $\phi(q_1, \alpha_1\alpha_2)$, $\phi(q_1, \alpha_1\alpha_2\alpha_2)$, $\ldots$, $\phi(q_1,\alpha_1\alpha_2^{n+1})$ есть два одинаковых значения, то есть $\exists i_1$, $i_2$ : $1\leqslant i_1\leq i_2\leqslant n+1$ и $\phi(q_1,\alpha_1\alpha_2^{i_1})=\phi(q_1,\alpha_1\alpha_2^{i_2})$. Пусть $q=\phi(q_1,\alpha_1\alpha_2^{i_1})$, $\alpha_3=\alpha_2^{i_2-i_1}$, тогда из предыдущего уравнения и первого свойства функции $\phi$ имеем $q=\phi(q,\alpha_3)$ $\Longrightarrow$ автомат будет возвращаться в состояние $q$, а выходная последовательность, с некоторого момента, будет образована повторяющимися словами $\overline{\psi}(q,\alpha_3)$ $\Longrightarrow$ $\overline{\psi}(q_1,\alpha_3)$ - периодическая последовательность с периодом, равным длине слова $\alpha_3=(i_2-i_1)\tau$ $\Longrightarrow$ минимальная длина периода является делителем числа $(i_2-i_1)\tau$ и имеет вид: $\theta\cdot m$, где $m|(i_2-i_1)$, а $\theta|\tau$, а так как $i_2-i_1\in\{1,\ldots,n\}$.\\
    Пусть $A=\{a_1,\ldots,a_k\}$, $k\geqslant3$ и $B=\{b_1,\ldots,b_p\}$ $p\geqslant2$. $\tau$ - произвольное натуральное число, $\theta$ - его делитель, $m\in \{1,\ldots,n\}$. Определим инициальный автомат $V_{q_1}$ диаграммой Мура (см. рисунок 8 лекции на странице 7).\\
    Если первая входная буква отлична от $a_1$, то до состояния $q_{n-m}$ будем получать $b_1$, а в состоянии $q_{n-m}$ получим либо $b_1$, либо $b_2$ в зависимости от последней буквы входного алфавита.\\
    Пусть первой входной буквой является буква $a_1$, тогда если следующая входная буква отлична от $a_{k-1}$ и $a_k$, то автомат не изменяет своего состояния, иначе - сдвигается по циклу и выходной буквой будет $b_1$, если автомат не переходит в состояние $q_{n-m+1}$, где выходным словом будет $b_2$.\\
    Рассмотрим слова $\alpha_1=a_1^{\theta-1}a_{k-1}$, $\alpha_2=a_1^{\theta-1}a_k$, $\alpha_3=\alpha_1^{\frac{\tau}{\theta}-1}\alpha_2$. Пусть $\theta>1$, возьмём периодичесую последовательность $\alpha=\alpha_3\alpha_3\ldots$, тогда в каждом слове $\alpha_3$ буква $a_k$ встретиться ровно один раз $\Longrightarrow$ наименьшая длина периода будет равна длине слова $\alpha_3$, что равно $\theta\cdot(\frac{\tau}{\theta}-1+1)=\tau$. Так как $\theta>1$, то слово $\alpha_3$ начинается с $a_1$ $\Longrightarrow$ пока на вход поступает $a_1$ состояние меняться не будет, это будет происходить $\theta-1$ раз, а при подаче последней буквы слова $\alpha_1$ или $\alpha_2$ автомат перейдёт в новое состояние. До перехода в состояние $q_{n-m+1}$ на выходе будем получать $b_1$, а при переходе в это состояние на выходе получим $b_2$. Однократное прохождение цикла происходит за время, равное $\theta\cdot m$, где $m$ - число состояний цикла $\Longrightarrow$ выходная последовательность будет образована периодически повторяющимися фрагментами $b_1^{\theta\cdot m-1}b_2$, а её период равен $\theta\cdot m$.\\
    Пусть $\theta=1$ $\Longrightarrow$ число состояний не меньше длины наименьшего периода, тогда изменем вышеуказанную диаграмму Мура так: если первая буква входного слово равна $a_{k-1}$ или $a_k$, то из состояния $q_1$ переходим в состояние $q_{n-m+1}$, иначе - в состояние $q_2$, остальное аналогично.
  \end{proof}
  \section{Билет 35 (Неотличимость автоматов и изоморфизм между ними)}
  \begin{definition}
    Пусть $V=(A$, $Q$, $B$, $\phi$, $\psi$) и $V'=(A$, $Q'$, $B$, $\phi'$, $\psi'$) - конечные автоматы. Если $\forall\alpha\in A^*$ $\overline{\psi}(q, \alpha)=\overline{\psi'}(q', \alpha)$, где $q\in Q$ и $q'\in Q'$, то состояния $q$ и $q'$ назыаются неотличимыми, иначе - отличимыми.
  \end{definition}
  \begin{definition}
    Если любые два состояния автомата отличимы друг от друга, то такой автомат называется автоматом приведённого вида.
  \end{definition}
  \begin{definition}
    Если для любого состояния $q$ автомата $V$ существует неотличимое от него состояние $q'$ автомата $V'$, и это верно, если поменять местами авоматы $V$ и $V'$, то такие автоматы назовём неотличимыми.
  \end{definition}
  \begin{definition}
    Пусть $V=(A$, $Q$, $B$, $\phi$, $\psi$) и $V'=(A$, $Q'$, $B$, $\psi'$, $\phi'$) - конечные автоматы. Если существует взаимно-однозначное отображение $\xi:Q\rightarrow Q'$ такое, что
    \begin{enumerate}
      \item $\xi(\phi(q,a))=\phi'(\xi(q),a)$\\
      \item $\psi(q,a)=\psi'(\xi(q),a)$ $(\forall q\in Q$, $a\in A)$
    \end{enumerate}
    то такие автоматы назовём изоморфными.
  \end{definition}
  \begin{theorem}
    Для любого конечного автомата $V$ существует единственный с точностью до изоморфизма конечный автомат приведнного вида, неотличимый от $V$.
  \end{theorem}
  \begin{proof}
    Пусть $V=(A$, $Q$, $B$, $\phi$, $\psi$) - конечный автомат. Рассмотрим разиение множества $Q$ на классы $Q_1$, $\ldots$, $Q_n$ попарно отличимых состояний, $q$, $q'\in Q_i$, $i\in\{1$, $\ldots$, $n\}$, $a\in A$. Если $\phi(q,a)\in Q_j$, $\phi(q',a)\in Q_{j'}$, $j\neq j'$, то $\exists \alpha$, $a\in A^*$ : $\overline{\psi}(\phi(q,a),\alpha)\neq\overline{\psi}(\phi(q',a)\alpha)$ $\Longrightarrow$ $\overline{\psi}(q,a\alpha)=\psi(q,a)\overline{\psi}(\phi(q,a),\alpha)\neq\psi(q',a)\overline{\psi}(\phi(q',a),\alpha)=\overline{\psi}(q',a\alpha)$ $\Longrightarrow$ состояния $q$ и $q'$ отличимы - противоречие $\Longrightarrow$ $Q_{j}=\phi'(Q_i,a)$. Значит, $\forall q$ и $q'\in Q_i$ $\psi(q,a)=\psi(q',a)=b$ $\Longrightarrow$ $b=\psi'(Q_i,a)$. В результате имеем функции $\phi':\{Q_1$, $\ldots$, $Q_n\}\times A\rightarrow \{Q_1,\ldots, Q_n\}$ и $\psi':\{Q_1,\ldots, Q_n\}\times A\rightarrow B$. Рассмотрим автомат $V'=(A$, $\{Q_1$, $\ldots$, $Q_n\}$, $B$, $\phi'$, $\psi'$), из определения функций $\phi'$ и $\psi'$ имеем $\forall q\in Q_i$ и $\alpha\in A^*$ $\phi(q,\alpha)\in \phi'(Q_i, \alpha)$, $\overline{\psi}(q,\alpha)=\overline{\psi'}(Q_i,\alpha)$ $\Longrightarrow$ автоматы $V$ и $V'$ неотличимы. Пусть $i\neq j$, $i$, $j\in\{1,\ldots, n\}$, $q\in Q_i$, $q'\in Q_j$, так как $Q_i$ и $Q_j$ отличимы, существует слово $\alpha\in A^*$ : $\overline{\psi}(q,\alpha)\neq\overline{\psi}(q',\alpha)$, тогда $\overline{\psi'}(Q_i, \alpha)\neq\overline{\psi'}(Q_j,\alpha)$ $\Longrightarrow$ автомат $V'$ является конечным автоматом приведённого вида.\\
    Пусть $V''=(A$, $Q''$, $B$, $\phi''$, $\psi'')$ - автомат приведённого вида, $Q''=\{q_1'',\ldots, q_m''\}$. Каждое $q_i''$ неотличимо от некоторого состояния автомата $V$, а значит, и от некоторого состояния автомата $V'$, а так как $q_1''$, $\ldots$, $q_m''$ попарно отличимы, то попарно отличимых $Q_1$, $\ldots$, $Q_n$ не меньше $m$ $\Longrightarrow$ $n\geqslant m$. Пусть $q\in Q_i$, тогда $q$ неотличимо от некоторого состояния автомата $V$, а значит, и некоторого состояния автомата $V''$, а так как состояния $Q_1$, $\ldots$, $Q_n$ попарно отличимы, то число отличимых состояний в $Q''$ не меньше $n$ $\Longrightarrow$ $n=m$. Рассмотрим отображение $\xi$ состояния $Q_i$ автомата $V'$ в соответствующее ему неотличимое состояние автомата $V''$. Это отображение взаимно-однозначно, так как каждому состоянию $Q_i$ автомата $V'$ соответствует единственное неотличимое состояние автомата $V''$. Так как $Q_i$ и $\xi(Q_i)$ неотличимы, то $\forall a\in A$ состояния $\phi'(Q_i,a)$ и $\phi''(\xi(Q_i),a)$ неотличимы $\Longrightarrow$ $\xi(\phi'(Q_i,a))=\phi''(\xi(Q_i),a)$. Так как состояния $Q_i$ и $\xi(Q_i)$ неотличимы $\Longrightarrow$ $\forall a\in A$ $\overline{\psi'}(\{Q_1,\ldots,Q_n\},a)=\overline{\psi''}(\xi(Q''),a)$ $\Longrightarrow$ $\psi'(Q_i,a)=\psi''(\xi(Q_i),a)$ $\forall Q_i$ и $a\in A$. Значит, по определению автоматы $V'$ и $V''$ изоморфны.
  \end{proof}
  \section{Билет 36 (Теорема Мура для одного конечного автомата)}
  \begin{theorem}
    Если два состояния автомата $V=(A$, $Q$, $B$, $\phi$, $\psi$) отличимы, то существует различающие их слово длины $|Q|-1$, причём эта оценка не улучшаема.
  \end{theorem}
  \begin{proof}
    Пусть $V=(A$, $Q$, $B$, $\phi$, $\psi$) - конечный автомат, состояния $q_1$ и $q_2$ отличимы. Рассмотрим отношение $\rho_k$ неотличимости состояний на множестве $Q$: $q\rho_kq'$ $\Leftrightarrow$ $\forall\alpha\in A^k$ $\overline{\psi}(q,\alpha)=\overline{\psi}(q',\alpha)$. Отношение $\rho_k$ - отношение эквивалентности, так как оно рефлексивно, симметрично и транзитивно. Множество классов эквивалентности обозначим $R_k$. Так как состояния $q_1$, $q_2$ отличимы, то существует непустое различающее их слово $\alpha$ наименьшей длины, оно представимо в виде $\alpha'a$, где $\alpha'\in A^*$, $a\in A$. Так как $\alpha$ - слово наименьшей длины, различающее состояния $q_1$ и $q_2$ $\Longrightarrow$ эти состояния отличаются последней буквой $\psi(\phi(q_1,\alpha'),a)\neq\psi(\phi(q_2,\alpha'),a)$ $\Longrightarrow$ $\phi(q_1,\alpha')\neq\phi(q_2,\alpha')$ $\Longrightarrow$ состояния $q_1$ и $q_2$ пренадлежат различным классам отношения $\rho_1$ $\Longrightarrow$ $|R_1|\geqslant2$.\\
    $|R_k|\leqslant|R_{k+1}|$ так как все классы эквивалентности, содержащиеся в $R_k$, содержатся в $R_{k+1}$. Рассмотрим разбиение $R_{\infty}$ множества $Q$ на классы попарно неотличимых состояний. Пусть $R_k=R_{k+1}$ и $R_k\neq R_{\infty}$, тогда $\exists q$ и $q'\in M$, $M\in R_k$, которые отличимы. Выберем $q$, $q'$, $M$ так, чтобы различающее эти состояния слово $\alpha$ имело наименьшую длину $\Longrightarrow$ длина слова $\alpha$ не меньше 2, так как $M\in R_k$ и $k\geqslant1$, тогда $\alpha$ имеет вид: $a\alpha'$, где $\alpha'\in A^*$, $a\in A$. Рассмотрим сосояния $\tilde{q}=\phi(q,a)$, $\widehat{q}=\phi(q',a)$. Слово $\alpha'$ различает состояния $\tilde{q}$ и $\widehat{q}$, так как $\psi(\phi(q,a),\alpha')=\psi(q,a\alpha')\neq\psi(q',a\alpha')=\psi(\phi(q',a),\alpha')$. Длина слова $\alpha'$ меньше длины слова $\alpha$, а так как $\alpha$ - слово наименьшей длины, то состояния $\tilde{q}$ и $\widehat{q}$ пренадлежат различным классам в $R_k$. Пусть $\tilde{q}\in M_1$, $\widehat{q}\in M_2$, $M_1$, $M_2\in R_k$, тогда рассмотрим слово $\alpha''$, имеющее длину $k$ и различающее состояния $\tilde{q}$ и $\widehat{q}$ $\Leftrightarrow$ $\overline{\psi}(\tilde{q},\alpha'')\neq\overline{\psi}(\widehat{q},\alpha'')$. Рассмотрим слово $\alpha'''=a\alpha''$, тогда $\overline{\psi}(q,\alpha''')=\psi(q,a)\overline{\psi}(\phi(q,a),\alpha'')=\psi(q,a)\overline{\psi}(\tilde{q},\alpha'')\neq\psi(q',a)\overline{\psi}(\widehat{q},\alpha'')=\psi(q',a)\overline{\psi}(\phi(q',a),\alpha'')=\overline{\psi}(q',\alpha''')$ $\Longrightarrow$ слово $\alpha'''$ различает состояния $q$ и $q'$ $\Longrightarrow$ эти состояния лежат в разных классах $R_{k+1}$ - противоречие.\\
    Рассмотрим последовательность $|R_1|$, $|R_2$, $\ldots$, она монотонно неубывает, и каждый её член не превосходит $|Q|$ $\Longrightarrow$ $\exists R_k$ : $|R_k|=|R_{k+1}|=R_{\infty}$ $\Longrightarrow$ $2\leqslant|R_1|\leq\ldots\leq|R_k|\leqslant|Q|$.\\
    База индукции: $|R_1|\geqslant1+1$.\\
    Пусть верно для $i=j-1$, то есть $|R_{j-1}|\geqslant j-1+1=j$. Так как $|R_{j-1}|\leq|R_j|$, то $|R_j|\geqslant|R_{j-1}|+1\geqslant j+1$.\\
    $\Longrightarrow$ $|Q|\geqslant|R_k|\geqslant k+1$ $\Longrightarrow$ $k\leqslant|Q|-1$ $\Longrightarrow$ любые два отличимые состояния автомата различимы словом длины $|Q|-1$.\\
    Докажем, что есть автомат, для которого данную оценку нельзя улучшить. Рассмотрим автомат $V=(A$, $Q$, $B$, $\phi$, $\psi$), $A=B=\{0,1\}$, $Q=\{q_1$, $\ldots$, $q_n\}$, который из любого состояния $q_i$, кроме $q_1$ и $q_n$, при подаче на вход 0 переходит в состояние $q_{i+1}$, при подаче на вход 1, переходит в состояние $q_{i-1}$, а выходной буквой будет 0. Из состояния $q_1$ при любом входном символе автомат переходит в состояние $q_2$, а выходным символом является 1. Из состояния $q_n$ при подаче на вход 0 автомат остаётся в  этом состоянии, а при подаче на вход 1 автомат переходит в состояние $q_{n-1}$, выходным символом в обоих случаях является 0.\\
    Чтобы отличить состояния $q_{n-1}$ и $q_n$ переведём состояние $q_{n-1}$ в состояние $q_1$, что можно сделать, если подать на вход $n-2$ единицы, причём длина такого слова минимальна, тогда отличия этих состояний остаётся подать на вход любой символ и получить на выходе 1 для $q_{n-1}$ и 0 для $q_n$. Длина данного слова равна $n-1$ и она минимальна.
  \end{proof}
  \section{Билет 37 (Теорема Мура для двух конечных автоматов)}
  \begin{theorem}
    Пусть $V=(A$, $Q$, $B$, $\phi$, $\psi$) и $V'=(A$, $Q'$, $B$, $\phi'$, $\psi'$) - конечвные автоматы, состояние $q_1$ автомата $V$ отличимо от состояния $q_2$ автомата $V'$, тогда существует различающее эти два состояния слово, имеющее длину $|Q|+|Q'|-1$, причём, вообще говоря, эта оценка неулучшаема.
  \end{theorem}
  \begin{proof}
    Пусть $V=(A$, $Q$, $B$, $\phi$, $\psi$) и $V'=(A$, $Q'$, $B$, $\phi'$, $\psi'$) - конечные автоматы, состояние $q_1$ автомата $V$ отличимо от состояния $q_2$ автомата $V'$. Если $Q$ и $Q'$пересекаются, то переобозначим состояния второго автомата так, чтобы они не пересекались, так что без ограничения общности будем считать, что $Q\cup Q'=\varnothing$. Рассмотрим автомат $V''=(A$, $Q\cup Q'$, $B$, $\phi''$, $\psi''$), $\phi''(q,a)=\phi(q,a)$, $\psi''(q,a)=\psi(q,a)$ $\forall q\in Q$, $\phi''(q,a)=\phi'(q,a)$, $\psi''(q,a)=\psi'(q,a)$ $\forall q\in Q'$. Состояния $q_1$ и $q_2$ являются отличимыми состояниями автомата $V''$, тогда по теоерме Мура для одного автомата существует различающее их слово, длина которого равна $|Q\cup Q'|-1=|Q|+|Q'|-1$.\\
    Докажем, что сущестуют автоматы, для которых эту оценку нельзя улучшить. Рассмотрим автомат $V=(A$, $Q$, $B$, $\phi$, $\psi$), $A=B=\{0,1\}$, $Q=\{q_1$, $\ldots$, $q_n\}$, который из любого состояния $q_i$, кроме $q_1$ и $q_n$, при подаче на вход 0 переходит в состояние $q_{i+1}$, при подаче на вход 1 переходит в состояние $q_{i-1}$, а выходной буквой будет 0. Из состояния $q_1$ при ллюбом входном символе автомат переходит в состояние $q_2$, а выходным символом является 1. Из состояния $q_n$ при подаче на вход 0 автомат остаётся в  этом состоянии, а при подаче на вход 1 автомат переходит в состояние $q_{n-1}$, выходным символом в обоих случаях является 0.\\
    Рассмотрим автомат $V'=(A$, $Q'$, $B$, $\phi'$, $\psi'$), $A=B=\{0,1\}$, $Q'=\{q'_1$, $\ldots$, $q'_m\}$ и $n\leqslant m$, который из любого состояния $q_i$, $1<i<n+1$, при подаче на вход 0 переходит в состояние $q_{i+1}$, при подаче на вход 1, переходит в состояние $q_{i-1}$, а выходной буквой будет 0. Для любого состояния $q_i$, где $n+1\leqslant i<m$, при подаче на вход 0 автомат переходит в состояние $q_{i+1}$, а при подаче на вход 1 автомат переходит в состояние $q_{n-1}$, в обоих случаях выходной буквой является 0. Из состояния $q_1$ при любом входном символе автомат переходит в состояние $q_2$, а выходным символом является 1. Из состояния $q_m$ при подаче на вход любого символа автомат остаётся в этом состоянии, выходным символом в обоих случаях является 0.\\
    Каждому состоянию $q'_i$ автомата $V'$, где $i\leqslant n$ сопоставим состояние $q_i$ автомата $V$, а для каждого состояния $q'_i$, где $m\geq i\geq n$, сопоставим состояние $q_n$ автомата $V$. Если состояние автомата $V$ сопоставлену состоянию автомата $V'$, то такие состояния будем называть соответствующими. Так как при подаче некоторого входного слова соответствующие состояния переходят в соответствующие состояния, за исключением того момента, когда второй автомат переходит в состояние $q'_m$ $\Longrightarrow$ если состояния $q_1$ и $q'_1$ отличимы, то существует слово $\alpha$, отличающее эти два состояния $\Longrightarrow$ некоторая начальная часть $\alpha'$ слова $\alpha'$ переводит второй автомат в состояние $q'_m$, а минимальная длина такого слова равна $m-1$. Данное слово переведёт первый автомат в состояние $q_n$, а значит, для того чтобы получить на выходе первого автомата 1, так как значение на выходе второго автомата равно нулю и меняться не будет, достаточно перевести первый автомат в состояние $q_1$, что можно сделать с помощью слова, наименьшая длина которого равна $n-1$, и остаётся лишь подать на вход любой символ, чтобы перевести первый автомат из состояния $q_1$ в состояние $q_2$ и получить на выходе единицу. Следовательно, наименьшая длина отличающего слова равна $m-1+n-1+1=n+m-1$.
  \end{proof}
  \section{Билет 38 (События в конечном алфавите)}
  \begin{definition}
    Пусть $V_q=(A$, $Q$, $B$, $\phi$, $\psi$, $q$) - инициальный конечный автомат, $B'\subseteq B$, множество $M=\{\alpha|\alpha\in A^*\setminus\{\Lambda\}$, $\psi(q,\alpha)\in B'\}$ называется представимым в инициальном конечном автомате $V_q$ с помощью подмножества $B'$ выходных символов.
  \end{definition}
  \begin{definition}
    Подмножества множества $A^*\setminus\{\Lambda\}$ называются событиями в алфавите $A$.
  \end{definition}
  \begin{definition}
    Если существует инициальный конечный автомат $V_q$, представляющий событие $M$ посредством некоторого подмножества $B'$, то событие $M$ называется представимым.
  \end{definition}
  \begin{definition}
    Произведение событий $M_1\cdot M_2$ - множество всех слов вида: $\alpha_1\alpha2$, где $\alpha_1\in M_1$, $\alpha_2\in M_2$.
  \end{definition}
  \begin{definition}
    Итерация события $M$ ($<M>$) - множество слов вида: $\alpha_1\ldots\alpha_k$, где $\alpha_1$, $\ldots$, $\alpha_k\in M$, $k\geqslant1$.
  \end{definition}
  \begin{properties}
    \begin{enumerate}
      \item $\varnothing\cdot M=M\cdot\varnothing=\varnothing$
      \item $<\varnothing>=\varnothing$
      \item $<M>=M\cdot<M>\cup M$
      \item $M\cdot<M>=<M>\cdot M$
    \end{enumerate}
  \end{properties}
  \begin{definition}
    Событие $M\subseteq A^*$ называется регулярным, если его можно получить из событий $\varnothing$, $\{a\}\in A$ с помощью конечного числа применения операции объединения событий, произведения событий и итерации события.
  \end{definition}
\end{document}