\documentclass[a4paper, 12pt]{article}

\usepackage{import}

% Корректность отображения всех шрифтов, кодировок и мат. символов
\usepackage[T2A]{fontenc}
\usepackage[utf8]{inputenc}
\usepackage[english, russian]{babel}
\usepackage{amssymb, amsmath, amsthm, mathtools}

% Отображение содержания
\usepackage{tocloft}

% Вставка картинок
\usepackage{graphicx}
\usepackage{tikz}
\usepackage{tkz-euclide}
\usepackage{asymptote}

\usepackage{wrapfig}        % Огибание картинок текстом
\usepackage{cancel}         % Зачёркивания
\usepackage{indentfirst}    % Отступ у первого абзаца
\usepackage{xcolor}         % Цвета
\setlength{\parskip}{.5ex}  % Отступы между абзацами
\usepackage{enumitem}       % Работа со списками
% \usepackage{minted}       % Вставка блоков кода

\usepackage{hyperref}       % гиперссылки
\definecolor{linkcolor}{HTML}{225ae2} % Цвет ссылок
\definecolor{urlcolor}{HTML}{225ae2} % Цвет гиперссылок
\hypersetup{
    pdfstartview=FitH, 
    linkcolor=linkcolor,
    urlcolor=urlcolor,
    colorlinks=true}
\setlength{\arrayrulewidth}{0.5mm} %Толщина линейки в таблицах
\setlength{\tabcolsep}{18pt} %Разделение между столбцами в таблице

% Отступы на странице
\usepackage[left=2cm, right=1.5cm, top=2cm, bottom=2cm]{geometry}

\usepackage{cmap}            % Русский поиск в PDF документе
\usepackage{etoolbox}
\usepackage{soul}            % Разряженный текст \so{} и подчеркивание \ul{}
\usepackage{soulutf8}        % Поддержка UTF8 в soul

\usepackage{titlesec}        % Форматирование заголовков
\titleformat{\section}{\LARGE \bfseries}{\thesection}{1em}{}
\titleformat{\subsection}{\Large\bfseries}{\thesubsection}{1em}{}
\titleformat{\subsubsection}{\large\bfseries}{\thesubsubsection}{1em}{}

\newcommand{\R}{\mathbb R}
\newcommand{\Q}{\mathbb Q}
\newcommand{\Z}{\mathbb Z}
\newcommand{\N}{\mathbb N}
\newcommand{\CC}{\mathbb C}
\newcommand{\F}{\mathbb F}
\newcommand{\aug}{\fboxsep=-\fboxrule\!\!\!\fbox{\strut}\!\!\!}
\newcommand{\sgn}{\operatorname{sgn}}
\newcommand{\id}{\mathrm{id}}
\renewcommand{\phi}{\varphi}
\renewcommand{\epsilon}{\varepsilon}

\newsavebox{\boxedalignbox}
\newenvironment{boxedalign*}
  {\begin{equation*}\begin{lrbox}{\boxedalignbox}$\begin{aligned}}
  {\end{aligned}$\end{lrbox}\fbox{\usebox{\boxedalignbox}}\end{equation*}}

\newcommand\tab[1][.5cm]{\hspace*{#1}}

% Подписи для матриц
\newcommand\undermat[2]{\makebox[0pt][l]{$\smash{\underbrace
{\phantom{\begin{matrix}#2\end{matrix}}}_{\text{$#1$}}}$}#2}
\newcommand\overmat[2]{\makebox[0pt][l]{$\smash{\overbrace
{\phantom{\begin{matrix}#2\end{matrix}}}^{\text{$#1$}}}$}#2}

% Значек "пусть"
\newlength{\tempheight}  
\newcommand{\Let}[0]{  
\mathbin{\text{\settoheight{\tempheight}{\mathstrut}\raisebox{0.5\pgflinewidth}{%
\tikz[baseline,line cap=round,line join=round] \draw (0,0) --++ (0.4em,0) --++ (0,1.5ex) --++ (-0.4em,0);
}}}}


% \newcounter{lemcount}
% \newcounter{thcount}
% \newcounter{offercount}
% \newcounter{concount}
% \newcounter{subthcount}
% \newcounter{defcount}

\theoremstyle{definition}
\newtheorem*{definition}{Определение}
% \newtheorem{definitionnum}[defcount]{Определение}
\newtheorem*{example}{Примеры}
\newtheorem*{example1}{Пример}
\newtheorem*{exercise}{Упражнение}


\theoremstyle{plain}
\newtheorem*{theorem}{Теорема}
% \newtheorem{theoremnum}[thcount]{Теорема}
\newtheorem*{consequense}{Следствие}
\newtheorem*{consequenses}{Следствия}
% \newtheorem{consequensenum}[concount]{Следствие}
\newtheorem*{lemma}{Лемма}
% \newtheorem{lemmanum}[lemcount]{Лемма}
\newtheorem*{subtheorem}{Утверждение}
% \newtheorem{subtheoremnum}[subthcount]{Утверждение}
\newtheorem*{algorithm}{Алгоритм}
\newtheorem*{properties}{Свойства}
\newtheorem*{properties1}{Свойство}


\theoremstyle{remark}
\newtheorem*{remark}{Замечание}
\newtheorem*{offer}{Предложение}
% \newtheorem{offernum}[offercount]{Предложение}
\begin{document}
  \newpage
  \begin{center}
    Билет 1
  \end{center}
  \begin{definition}
    Упорядоченный набор - функция, которая ставит в соответствие каждому элементому множества $\{1, \ldots, n\}$ элемент из множества $\{a_1,\ldots,a_n\}$ : $1\rightarrow a_1$, $\ldots$, $n\rightarrow a_n$.
  \end{definition}
  Декартовое произведение множеств $A_1\times\ldots\times A_n=(a_1, \ldots, a_n)$ : $a_i\in A_i$.
  \begin{definition}
    Пусть функция $f$ определена на $A_1\times\ldots\times A_n$, тогда $f$ - $n$-местная функция.
  \end{definition}
  \begin{definition}
    Множество $B_n=E_2\times\ldots\times E_n$, где $E_i=\{0,1\}$, называется $n$-мерным булевым кубом.
  \end{definition}
  \begin{definition}
    Функция $f:B_n \to E_2$ называется функцией алгебры логики. Множество всех таких функций обозначим $P_2$.
  \end{definition}
  Представление функции $f(x_1, \ldots, x_n)$ в виде таблицы, имеющей $n+1$ столбец:\\
  $\begin{matrix}
    x_1 & \ldots & x_{n-1} & x_n & f\\
    0 & \ldots & 0 & 0 & 0\\
    0 & \ldots & 0 & 0 & 1\\
    0 & \ldots & 0 & 1 & 0\\
    \vdots & \null & \vdots & \vdots & \vdots\\
    1 & \ldots & 1 & 1 & 1 
  \end{matrix}$\\
  Так как число различных первых $n$ столбцов $2^n$, так как в каждой ячейке одного столбца может быть либо 0, либо 1. $\Longrightarrow$ число функций будет $2^{2^n}$, так как для каждого набора значение функции может быть либо 0, либо 1.
  \begin{definition}
    Переменная $x_i$ называется существенной, если существуют наборы $\alpha_1$, $\ldots$, $\alpha_{i-1}$, $1$, $\alpha_{i+1}$, $\ldots$, $\alpha_n$ и $\alpha_1$, $\ldots$, $\alpha_{i-1}$, $0$, $\alpha_{i+1}$, $\ldots$, $\alpha_n$, на которых функция принимает различные значения. В противном случае переменная $x_i$ называется несущественной (фиктивной).
  \end{definition}
  \begin{definition}
    Пусть $x_i$ - фиктивная переменная, тогда если функция $f(x_1$, $\ldots$, $x_{i-1}$, $x_{i+1}$, $\ldots$, $x_n)=g(x_1$, $\ldots$, $x_{i-1}$, $0$, $x_{i+1}$, $\ldots$, $x_n)$, то функция $g$ называется полученной из $f$ добавлением фиктивной переменной. Функция удаления фиктивной переменной определяется аналогично.
  \end{definition}
  \begin{definition}
    Функция называется симметрической, если при любых перестановках переменных $x_{i_1}, \ldots, x_{i_n}$ значение функции не меняется.
  \end{definition}
  Элементарные функции в алгебре логики:
  \begin{enumerate}
    \item константы 0, 1
    \item тождественный $x$
    \item отрицание $\overline{x}$
    \item конъюнкция $x\wedge y$
    \item дизъюнкция $x\vee y$
    \item имплекация $x\rightarrow y$
    \item штрих Шеффера $x|y$
    \item стрелка Пирса $x\downarrow y$
    \item сложение по модулю 2
    \item эквивалентность
  \end{enumerate}
  \begin{center}
    Билет 2
  \end{center}
  \begin{definition}
    Формула - слово в некотором алфавите $A$.
  \end{definition}
  \begin{definition}
    Алфавит - конечное или бесконечное множество.
  \end{definition}
  \begin{definition}
    Слово - произвольная функция, определённая на начальном отрезке натурального ряда и принимающая на нём значения из $A$.
  \end{definition}
  \begin{definition}
    Пусть $F$ - множество функций алгебры логики, $S$ - множество символов, обозначающих функции из $F$, тогда отображение $\Sigma : S \to F$ - сигнатура для $F$.
  \end{definition}
  \begin{definition}
    Пусть $X=\{x_1, \ldots\}$ - символы переменных.\\
    База индукция: если $x_i$ - символ переменной, то однобуквенное слово, состоящее из $x_i$ - формула в сигнатуре $\Sigma$.\\
    Пусть $s\in S$, $f=\Sigma(s)$ - функция от $n$ переменных, Ф$_1$, $\ldots$, Ф$_n$ - формулы в сигнатуре $\Sigma$, тогда слово $s($Ф$_1$, $\ldots$, Ф$_n)$ - формула в сигнатуре $\Sigma$.  
  \end{definition}
  \begin{definition}
    Пусть Ф - формула, $\tilde{x}$ - упорядоченный набор ($x_{i_1}, \ldots, x_{i_n})$, содержащий все переменные формулы Ф, $\tilde{\alpha}=(\alpha_1, \ldots, \alpha_n)$ - двоичный набор.\\
    База индукции: Ф - однобуквенное слово $x_{i_j}$, тогда Ф$[\tilde{x}, \tilde{\alpha}]=\alpha_j$ - значение формулы на наборе $\tilde{\alpha}$.\\
    Пусть $F$ - $s($Ф$_1, \ldots, $Ф$_n)$, $f=\Sigma(s)$, причём Ф$_1[\tilde{x}, \tilde{\alpha}]=\beta_1$, $\ldots$, Ф$_n[\tilde{x}, \tilde{\alpha}]=\beta_n$, тогда $f(\beta_1, \ldots, \beta_n)$ - значение функции на наборе значений переменных.
  \end{definition}
  \begin{definition}
    Формула, определяющая функцию алгебры логики, определённую на $B_n$ такую, что $\forall$ набора $\tilde{\alpha}=(\alpha_1$, $\ldots$, $\alpha_n)\in B_n$ $f(\tilde{\alpha})=F[\tilde{x}, \tilde{\alpha}]$.
  \end{definition}
  \begin{definition}
    Формулы в сигнатуре, представляющие собой переменные, называются вырожденными, остальные - невырожденными. Если функция определяется невырожденной формулой в сигнатуре $\Sigma:S \to F$, то она получена суперпозициями над $F$, где $F$ - множество функций.
  \end{definition}
  \begin{definition}
    (Другое определение суперпозиции) Если одну функцию можно получить с помощью конечного числа применений следующих трёх операций, то данная функция называется функцией, полученной суперпозициями над $F$.\\
    Операции:\\
    \begin{enumerate}
      \item Операция подстановки переменных. Пусть $f(x_1$, $\ldots$, $x_n)\in P_2$, $g(x_1$, $\ldots$, $x_n)$ - функция, определённая на $B_n$ такая, что $g(x_1$, $\ldots$, $x_n)=f(x_{i_1}$, $\ldots$, $x_{i_n})$, где набор ($i_1$, $\ldots$, $i_n$) - набор элементов ($1$, $\ldots$, $n$) (они необязательно различны). Тогда $g$ получена из $f$ операцией подстановки переменных.
      \item Операция подстановки функции в функцию. Пусть $f(x_1$, $\ldots$, $x_n)$, $g(x_1$, $\ldots$, $x_m)$, $h$ определена на $B_{n+m-1}$ и $h(x_1$, $\ldots$, $x_{n+m-1})=f(x_1$, $\ldots$, $x_{n-1}$, $g(x_n$, $\ldots$, $x_{n+m-1}))$, тогда функция $h$ получена из функций $f$ и $g$ операцией подстановки одной функции в другую.
      \item Операция добавления или удаления фиктивных переменных. Пусть $x_i$ - фиктивная переменная, тогда если функция $f(x_1$, $\ldots$, $x_{i-1}$, $x_{i+1}$, $\ldots$, $x_n)=g(x_1$, $\ldots$, $x_{i-1}$, $0$, $x_{i+1}$, $\ldots$, $x_n)$, то функция $g$ называется полученной из $f$ добавлением фиктивной переменной. Функция удаления фиктивной переменной определяется аналогично.
    \end{enumerate}
  \end{definition}
  \begin{center}
    Билет 3
  \end{center}
  \begin{definition}
    Формулы $F_1$ и $F_2$ называются эквивалентными, если они определяют равные функции относительно объединения их переменных. Функции называются равными, если их области определения равны и $\forall x\in D_f(x)$ $f(x)=g(x)$. Слово $F_1=F_2$, если формулы $F_1$ и $F_2$ эквивалентны, называется тождеством. 
  \end{definition}
  Основные тождества:
  \begin{enumerate}
    \item Ассоциативность операций: $\wedge$, $\vee$, $\neg$, $\leftrightarrow$.
    \item Дистрибутивности:
    \begin{enumerate}
      \item $(x\vee y)\wedge z=(x\wedge z)\vee(y\wedge z)$
      \item $(x\wedge y)\vee z=(x\vee z)\wedge(y\vee z)$
      \item $(x+y)\cdot z=x\cdot z+y\cdot z$
    \end{enumerate}
    \item Тождества для отрицания: 
    \begin{enumerate}
      \item $\overline{\overline{x}}=x$
      \item $\overline{x\wedge y}=\overline{x}\vee \overline{y}$
      \item $\overline{x\vee y}=\overline{x}\wedge \overline{y}$
      \item $x\cdot\overline{x}=0$
      \item $x\vee\overline{x}=1$
      \item $\overline{x\rightarrow y}=x\cdot\overline{y}$
    \end{enumerate}
    \item Тождества для эдентичных операндов
    \item Тождества с константным операндом
  \end{enumerate}
  \begin{definition}
    Функция $g$ называется двойственной к $f$, если $g(x_1$, $\ldots$, $x_n)=\overline{f}(\overline{x_1}$, $\ldots$, $\overline{x_n})$. Обозначение $g=f^*$.
  \end{definition}
  \begin{definition}
    Если функция двойственна к самой себе, то она называется самодвойственной.
  \end{definition}
  \begin{theorem}(принцип двойственности)
    Если Ф - формула в сигнатуре $\Sigma: S\to F$, определяющая некоторую функцию $g$, то эта формула в сигнатуре $\Sigma^*: S\to F^*$ определяет двойственную функцию $g^*$.
  \end{theorem}
  \begin{proof}
    База индукции: пусть $x_i$ - символ переменной, тогда однобуквенное слово, состоящее из $x_i$ - формула в сигатуре $\Sigma$, определяющая одноместную функцию $g$. Эта формула в сигнатуре $\Sigma^*$ имеет вид $\overline{x_i}$, то есть она определяет функцию, двойственную к $g$.\\
    Пусть $s\in S$, $f=\Sigma(s)$ - формула от $n$ переменных, Ф$_1$, $\ldots$, Ф$_n$ - формулы в сигнатуре $\Sigma$, тогда слово $s$(Ф$_1$, $\ldots$, Ф$_n$) - формула в сигнатуре $\Sigma$. В $\Sigma^*(s)=(\Sigma(s))^*=(\Sigma(s$(Ф$_1$, $\ldots$, Ф$_n$)))$^*=f^*$, то есть данная формула определяет в двойственной сигнатуре двойственнную функцию.
  \end{proof}
  \begin{center}
    Билет 4
  \end{center}
  \begin{definition}
    Выражение $f(x_1$, $\ldots$, $x_n)=\bigvee\limits_{(\sigma_1, \ldots,\sigma_n):f(\sigma_1, \ldots,\sigma_n)=1}x_1^{\sigma_1}\cdot\ldots\cdot x_n^{\sigma_n}$ называется совершенной дизъюнктивной нормальной формой. $x_i^{\sigma_i}=$
    $\begin{cases}
      x_i,\sigma_i=1\\
      \overline{x_i},\sigma_i=0
    \end{cases}$.
  \end{definition}
  \begin{theorem}
    Для любой функции $f(x_1$, $\ldots$, $x_n)$ алгебры логики верно равенство:\\
    $f(x_1$, $\ldots$, $x_n)=\bigvee\limits_{(\sigma_1, \ldots,\sigma_m)\in B_m}x_1^{\sigma_1}\cdot\ldots\cdot x_m^{\sigma_m}\cdot f(\sigma_1$, $\ldots$, $\sigma_m$, $\sigma_{m+1}$, $\ldots$, $\sigma_n)$.
  \end{theorem}
  \begin{proof}
    Рассмотрим прозвольный набор $(\alpha_1$, $\ldots$, $\alpha_m$), если $(\alpha_1$, $\ldots$, $\alpha_m)\neq(\sigma_1$, $\ldots$, $\sigma_m$), то $\exists \alpha_i\neq\sigma_i$ $\Longrightarrow$ $\alpha_i^{\sigma_i}=0$ $\Longrightarrow$ данное слагаемое будет равно нулю. Тогда единственным не нулевым членом будет $(\alpha_1^{\alpha_1}\cdot\ldots\cdot\alpha_m^{\alpha_m})\cdot f(\alpha_1$, $\ldots$, $\alpha_m$, $\alpha_{m+1}$, $\ldots$, $\alpha_n)=f(\alpha_1$, $\ldots$, $\alpha_n$).
  \end{proof}
  \begin{theorem}
    Любую функцию алгебры логики можно представить с помощью суперпозиций конъюнкции, дизъюнкции и отрицания.
  \end{theorem}
  \begin{proof}
    Так как любая функция алгебры логики, кроме тождественного нуля, реализуется совершенной д.н.ф., значит она представима суперпозициями конъюнкции, дизъюнкции и отрицания. Тождественный ноль можно представить так: $x\wedge\overline{x}=0$.
  \end{proof}
  \begin{theorem}
    Любая функция алгебры логики, кроме тождественной единицы, представима в виде совершенной конъюнктивной нормальной формы.
  \end{theorem}
  \begin{proof}
    Так как любая функция алгебры логики, кроме тождественного нуля, представима в виде совершенной д.н.ф., тогда по принципу двойственности\\
    $f(x_1$, $\ldots$, $x_n)=\bigwedge\limits_{(\sigma_1, \ldots,\sigma_n):f^*(\sigma_1, \ldots,\sigma_n)=1}x_1^{\sigma_1}\vee\ldots\vee x_n^{\sigma_n}$ $\Longrightarrow$\\
    $f(x_1$, $\ldots$, $x_n)=\bigwedge\limits_{(\delta_1, \ldots,\delta_n):f(\delta_1, \ldots,\delta_n)=1}x_1^{\overline{\delta}_1}\vee\ldots\vee x_n^{\overline{\delta}_n}$.
  \end{proof}
  \begin{center}
    Билет 5
  \end{center}
  \begin{definition}
    Система функций называется полной в $P_2$, если через них выражаются все функции в $P_2$.
  \end{definition}
  \begin{example}
    \begin{enumerate}
      \item $\wedge$ и $\neg$
      \item $\vee$ и $\neg$
      \item $x|y$
      \item $x\downarrow y$
    \end{enumerate}
  \end{example}
  \begin{definition}
    Полиномы по модулю 2 вида: $\sum\limits_{\{i_1,\dots,i_s\}\subseteq{1,\ldots,n}}a_{i_1,\ldots,i_s}\cdot x_{i_1}\cdot\ldots\cdot x_{i_s}$ называются полиномами Жегалкина.
  \end{definition}
  \begin{theorem}(Жегалкина)\\
    Любая функция алгебры логики представима полиномом Жегалкина, причём единственным образом.
  \end{theorem}
  \begin{proof}
    Так как в каждом мономе полинома Жегалкина $n$ перменных, каждая из которых может быть либо 0, либо 1, а коэффициент перед каждым мономом может принимать значение 0 или 1 $\Longrightarrow$ всего есть $2^{2^n}$ различных полиномов Жегалкина.\\
    Пусть два различных полинома Жегалкина задают одну функцию, тогда мы получим ненулевой полином, задающий нулевую константу $\Longrightarrow$ противоречие $\Longrightarrow$  Любая функция алгебры логики представима полиномом Жегалкина, причём единственным образом.
  \end{proof}
  \begin{center}
    Билет 6
  \end{center}
  \begin{definition}
    Множество функций, кторые можно пулучить из данного множества $M$ функций алгебры логики, называется замыканием множества $M$ и обозначается $[M]$.
  \end{definition}
  \begin{example}
    \begin{enumerate}
      \item $P_2=[P_2]$
      \item [{1, $x+y$}] - множество линейных функций
    \end{enumerate}
  \end{example}
  \begin{properties}
    \begin{enumerate}
      \item $M\subseteq[M]$
      \item $[[M]]=[M]$
      \item Если $M_1\subseteq M_2$, то $[M_1]\subseteq[M_2]$
      \item $[M_1]\cup[M_2]\subseteq[M_1\cup M_2]$
    \end{enumerate}
  \end{properties}
  \begin{proof}
    \begin{enumerate}
      \item По определению замыкания.
      \item Из первого следует, что $[M]\subseteq[[M]]$, а $[[M]]\subseteq[M]$, так как в противном случае существовала бы функция, которая не выражается суперпозициями функций из $M$, но выражается суперпозициями функций, которые выражаются суперпозициями функций из $M$, а значит она выражается суерпозициями из $M$ $\Longrightarrow$ противоречие.
      \item Если функция получается суперепозициями из $M_1$, то её можно получить суперпозициями из $M_2$, так как все функции $M_1$ являются функциями $M_2$.
      \item Пусть функция $f\in[M_1]\cap[M_2]$, тогда она получается суперпозициями из $M_1$ или из $M_2$, пусть для определённости она выражается суперпозициями из $M_1$, но тогда её можно получить суперпозициями из $M_1\cap M_2$, то есть $f\in[M_1\cap M_2]$
    \end{enumerate}
  \end{proof}
  \begin{definition}
    Класс функций $M$ называется замкнутым, если $[M]=M$.
  \end{definition}
  \begin{example}
    \begin{enumerate}
      \item $P_2=[P_2]$
      \item $L=[L]$, $L$ - множество линейных функций.
    \end{enumerate}
  \end{example}
  \begin{center}
    Билет 7
  \end{center}
  \begin{definition}
    Функция $f$ называется функцией, сохраняющей ноль, если на наборе из нулей она принимает значение 0. 
  \end{definition}
  \begin{definition}
    Функция $f$ называется функцией, сохраняющей единицу, если на наборе из единиц она принимает значение 1. 
  \end{definition}
  Класс функций, сохраняющих ноль, обозначим $T_0$, а класс функций, сохраняющих единицу, обозначим $T_1$.
  \begin{theorem}
    Классы $T_0$ и $T_1$ замкнуты.
  \end{theorem}
  \begin{proof}
    \begin{enumerate}
      \item Операция подстановки переменных:\\
      $g(x_1$, $\ldots$, $x_n)=f(x_{i_1}$, $\ldots$, $x_{i_n})$, если функция $f$ сохраняла ноль, то и функция $g$ будет сохранять ноль, если функция $f$ сохраняла единицу, то и функция $g$ будет сохранять единицу.
      \item Операция подстановки одной функции в другую:\\
      $h(x_1$, $\ldots$, $x_{n+m-1})=f(x_1$, $\ldots$, $x_{n-1}$, $g(x_n$, $\ldots$, $x_{n+m-1}))$, если функции $f$ и $h$ сохраняли ноль, то и функция $g$ будет сохранять ноль, если функции $f$ и $g$ сохраняли единицу, то и функция $h$ будет сохранять единицу.
      \item Операция добавления или удаления фиктивной переменной, не влияют на способность функции сохранять ноль или сохранять единицу.
    \end{enumerate}
    Следовательно суперпозициями мы не сможем получить функцию, не пренадлежащую данному классу $\longrightarrow$ классы $T_0$ и $T_1$ - замкнуты. 
  \end{proof}
  \begin{center}
    Билет 8
  \end{center}
  Класс самодвойственных функций обозначим $S$.
  \begin{theorem}
    Класс $S$ замкнут.
  \end{theorem}
  \begin{proof}
    \begin{enumerate}
      \item Операция подстановки переменных:\\
      Пусть $f(x_1$, $\ldots$, $x_n)\in S$, $g(x_1$, $\ldots$, $x_n)=f(x_{i_1}$, $\ldots$, $x_{i_n})$, тогда $\overline{g}(\overline{x}_1$, $\ldots$, $\overline{x}_n)=\overline{f}(\overline{x}_{i_1}$, $\ldots$, $\overline{x}_{i_n})=f(x_{i_1}$, $\ldots$, $x_{i_n})=g(x_1$, $\ldots$, $x_n)$ $\Longrightarrow$ $g$ - самодвойственная функция.
      \item Операция подстановки функции в функцию:\\
      Пусть $f(x_1$, $\ldots$, $x_n)\in S$, $g(x_1$, $\ldots$, $x_m)\in S$, $h(x_1$, $\ldots$, $x_n$, $x_{n+1}$, $\ldots$, $x_{n+m-1})=f(x_1$, $\ldots$, $x_{n-1}$, $g(x_n$, $\ldots$, $x_{n+m-1}))$, тогда $\overline{h}(\overline{x}_1$, $\ldots$, $\overline{x}_n$, $\overline{x}_{n+1}$, $\ldots$, $\overline{x}_{m+n-1})=\overline{f}(\overline{x}_1$, $\ldots$, $\overline{x}_{n-1}$, $g(\overline{x}_{n}$, $\ldots$, $\overline{x}_{m+n-1}))=\overline{f}(\overline{x}_1$, $\ldots$, $\overline{x}_{n-1}$, $\overline{g}(x_{n}$, $\ldots$, $x_{m+n-1}))=f(x_1$, $\ldots$, $x_{n-1}$, $g(x_{n}$, $\ldots$, $x_{m+n-1}))=h(x_1$, $\ldots$, $x_n$, $x_{n+1}$, $\ldots$, $x_{m+n-1})$ $\Longrightarrow$ $h$ - самодвойственная функция.
      \item Операция добавления или удаления фиктивных переменных:\\
      Пусть $f(x_1$, $\ldots$, $x_n)\in S$, $g(x_1$, $\ldots$, $x_{i-1}$, $0$, $x_{i+1}$, $\ldots$,  $x_n) = f(x_1$, $\ldots$, $x_n)=g(x_1$, $\ldots$, $x_{i-1}$, $1$, $x_{i+1}$, $\ldots$,  $x_n)$, тогда $\overline{g}(\overline{x}_1$, $\ldots$, $\overline{x}_{i-1}$, $1$, $\overline{x}_{i+1}$, $\ldots$,  $\overline{x}_n) = f(x_1$, $\ldots$, $x_n)=g(x_1$, $\ldots$, $x_{i-1}$, $0$, $x_{i+1}$, $\ldots$,  $x_n)$ $\Longrightarrow$ $g$ - самодвойственная функция. 
    \end{enumerate}
  \end{proof}
  \begin{theorem}
    Если функция $f$ не является самодвойственной, то с помощью неё и функции отрицания можно получить константу.
  \end{theorem}
  \begin{proof}
    Пусть $f(x_1$, $\ldots$, $x_n)\notin S$, тогда существует набор $(\alpha_1$, $\ldots$, $\alpha_n$) :\\ 
    \begin{center}
      $f(\alpha_1$, $\ldots$, $\alpha_n)=f(\overline{\alpha}_1$, $\ldots$, $\overline{\alpha}_n)$.\\
    \end{center} 
    Пусть $\phi_i=x^{\alpha_i}$, $\phi(x)=f(\phi_1(x)$, $\ldots$, $\phi_n(x))$,\\ 
    \begin{center}
      тогда $\phi(0)= f(0^{\alpha_1}, \ldots, 0^{\alpha_n})=f(\overline{\alpha}_1, \ldots, \overline{\alpha}_n)=f(\alpha_1, \ldots, \alpha_n)=f(1^{\alpha_1}, \ldots, 1^{\alpha_n})=\phi(1)\Longrightarrow$\\
    \end{center}
      $\Longrightarrow$ $\phi(x)$ - константа, полученная из несамодвойственной функции и отрицания.
  \end{proof}
  \begin{center}
    Билет 9
  \end{center}
  \begin{definition}
    Пусть $\tilde{\alpha}=(\alpha_1$, $\ldots$, $\alpha_n)$, $\tilde{\beta}=(\beta_1$, $\ldots$, $\beta_n)$ - двоичные наборы, тогда $\tilde{\alpha}\leqslant\tilde{\beta}$, если $\forall i=\overline{1,n}$ $\alpha_i\leqslant\beta_i$.
  \end{definition}
  \begin{definition}
    Функция алгебры логики называется монотонной, если $\forall$ двоичных наборов $\tilde{\alpha}$ и $\tilde{\beta}$ таких, что $\tilde{\alpha}\leqslant\tilde{\beta}$, $f(\tilde{\alpha})\leqslant f(\tilde{\beta})$.
  \end{definition}
  \begin{theorem}
    Класс $M$ монотонных функций - замкнут.
  \end{theorem}
  \begin{proof}
    \begin{enumerate}
      \item Операция подстановки переменных:\\
      $g(x_1$, $\ldots$, $x_n)=f(x_{i_1}$, $\ldots$, $x_{i_n})$, если функция $f$ монотонна, то\\ $\forall\tilde{\alpha}=(\alpha_1$, $\ldots$, $\alpha_n)$ и $\tilde{\beta}=(\beta_1$, $\ldots$, $\beta_n)$ : $\tilde{\alpha}\leqslant\tilde{\beta}$, $f(\tilde{\alpha})\leqslant f(\tilde{\beta})$ $\Longrightarrow$ $\alpha_1\leqslant\beta_1$, $\ldots$, $\alpha_n\leqslant\beta_n$ $\Longrightarrow$\\
      $\Longrightarrow$ $\alpha_{i_1}\leqslant\beta_{i_1}$, $\ldots$, $\alpha_{i_n}\leqslant\beta_{i_n}$ $\Longrightarrow$ $f(\alpha_{i_1}$, $\ldots$, $\alpha_{i_n})\leqslant f(\beta_{i_1}$, $\ldots$, $\beta_{i_n})$ $\Longrightarrow$\\
      $\Longrightarrow$ $g(\alpha_{1}$, $\ldots$, $\alpha_{n})=f(\alpha_{i_1}$, $\ldots$, $\alpha_{i_n})\leqslant f(\beta_{i_1}$, $\ldots$, $\beta_{i_n})=g(\beta_{i_1}$, $\ldots$, $\beta_{i_n})$ $\Longrightarrow$ $g$ - монотонна.
      \item Операция подстановки одной функции в другую:\\ 
      $f(x_{1}$, $\ldots$, $x_{n})$, $g(x_{1}$, $\ldots$, $x_{m})$ - монотонные функции, $h(x_1$, $\ldots$, $x_{n+m-1})=f(x_1$, $\ldots$, $x_{n-1}$, $g(x_n$, $\ldots$, $x_{n+m-1}))$, так как функции $f$ и $g$ монотонны, $\forall\tilde{\alpha}=(\alpha_1$, $\ldots$, $\alpha_{m+n-1})$ и $\tilde{\beta}=(\beta_1$, $\ldots$, $\beta_{m+n-1})$ : $\tilde{\alpha}\leqslant\tilde{\beta}$, $f(\tilde{\alpha})\leqslant f(\tilde{\beta})$ и $g(\alpha_n$, $\ldots$, $\alpha_{m+n-1})=g(\beta_n$, $\ldots$, $\alpha_{m+n-1})$ $\Longrightarrow$\\
      $(\alpha_1$, $\ldots$, $\alpha_{n-1}$, $g(\alpha_n$, $\ldots$, $\alpha_{m+n-1}))\leqslant (\beta_1$, $\ldots$, $\beta_{n-1}$, $g(\beta_n$, $\ldots$, $\beta_{n+m-1}))$ $\Longrightarrow$ $h(\alpha_1$, $\ldots$, $\alpha_{m+n-1})=f(\alpha_1$, $\ldots$, $\alpha_{n-1}$, $g(\alpha_n$, $\ldots$, $\alpha_{m+n-1}))\leqslant f(\beta_1$, $\ldots$, $\beta_{n-1}$, $g(\beta_n$, $\ldots$, $\beta_{n+m-1}))=h(\beta_1$, $\ldots$, $\beta_{m+n-1})$.
      \item Операция добавления или удаления фиктивных переменных: 
    \end{enumerate}
    Следовательно суперпозициями мы не сможем получить функцию, не пренадлежащую данному классу $\longrightarrow$ класс $M$ - замкнут. 
  \end{proof}
\end{document}