\documentclass[a4paper, 12pt]{article}

\usepackage{import}

% Корректность отображения всех шрифтов, кодировок и мат. символов
\usepackage[T2A]{fontenc}
\usepackage[utf8]{inputenc}
\usepackage[english, russian]{babel}
\usepackage{amssymb, amsmath, amsthm, mathtools}

% Отображение содержания
\usepackage{tocloft}

% Вставка картинок
\usepackage{graphicx}
\usepackage{tikz}
\usepackage{tkz-euclide}
\usepackage{asymptote}

\usepackage{wrapfig}        % Огибание картинок текстом
\usepackage{cancel}         % Зачёркивания
\usepackage{indentfirst}    % Отступ у первого абзаца
\usepackage{xcolor}         % Цвета
\setlength{\parskip}{.5ex}  % Отступы между абзацами
\usepackage{enumitem}       % Работа со списками
% \usepackage{minted}       % Вставка блоков кода

\usepackage{hyperref}       % гиперссылки
\definecolor{linkcolor}{HTML}{225ae2} % Цвет ссылок
\definecolor{urlcolor}{HTML}{225ae2} % Цвет гиперссылок
\hypersetup{
    pdfstartview=FitH, 
    linkcolor=linkcolor,
    urlcolor=urlcolor,
    colorlinks=true}
\setlength{\arrayrulewidth}{0.5mm} %Толщина линейки в таблицах
\setlength{\tabcolsep}{18pt} %Разделение между столбцами в таблице

% Отступы на странице
\usepackage[left=2cm, right=1.5cm, top=2cm, bottom=2cm]{geometry}

\usepackage{cmap}            % Русский поиск в PDF документе
\usepackage{etoolbox}
\usepackage{soul}            % Разряженный текст \so{} и подчеркивание \ul{}
\usepackage{soulutf8}        % Поддержка UTF8 в soul

\usepackage{titlesec}        % Форматирование заголовков
\titleformat{\section}{\LARGE \bfseries}{\thesection}{1em}{}
\titleformat{\subsection}{\Large\bfseries}{\thesubsection}{1em}{}
\titleformat{\subsubsection}{\large\bfseries}{\thesubsubsection}{1em}{}

\newcommand{\R}{\mathbb R}
\newcommand{\Q}{\mathbb Q}
\newcommand{\Z}{\mathbb Z}
\newcommand{\N}{\mathbb N}
\newcommand{\CC}{\mathbb C}
\newcommand{\F}{\mathbb F}
\newcommand{\aug}{\fboxsep=-\fboxrule\!\!\!\fbox{\strut}\!\!\!}
\newcommand{\sgn}{\operatorname{sgn}}
\newcommand{\id}{\mathrm{id}}
\renewcommand{\phi}{\varphi}
\renewcommand{\epsilon}{\varepsilon}

\newsavebox{\boxedalignbox}
\newenvironment{boxedalign*}
  {\begin{equation*}\begin{lrbox}{\boxedalignbox}$\begin{aligned}}
  {\end{aligned}$\end{lrbox}\fbox{\usebox{\boxedalignbox}}\end{equation*}}

\newcommand\tab[1][.5cm]{\hspace*{#1}}

% Подписи для матриц
\newcommand\undermat[2]{\makebox[0pt][l]{$\smash{\underbrace
{\phantom{\begin{matrix}#2\end{matrix}}}_{\text{$#1$}}}$}#2}
\newcommand\overmat[2]{\makebox[0pt][l]{$\smash{\overbrace
{\phantom{\begin{matrix}#2\end{matrix}}}^{\text{$#1$}}}$}#2}

% Значек "пусть"
\newlength{\tempheight}  
\newcommand{\Let}[0]{  
\mathbin{\text{\settoheight{\tempheight}{\mathstrut}\raisebox{0.5\pgflinewidth}{%
\tikz[baseline,line cap=round,line join=round] \draw (0,0) --++ (0.4em,0) --++ (0,1.5ex) --++ (-0.4em,0);
}}}}


% \newcounter{lemcount}
% \newcounter{thcount}
% \newcounter{offercount}
% \newcounter{concount}
% \newcounter{subthcount}
% \newcounter{defcount}

\theoremstyle{definition}
\newtheorem*{definition}{Определение}
% \newtheorem{definitionnum}[defcount]{Определение}
\newtheorem*{example}{Примеры}
\newtheorem*{example1}{Пример}
\newtheorem*{exercise}{Упражнение}


\theoremstyle{plain}
\newtheorem*{theorem}{Теорема}
% \newtheorem{theoremnum}[thcount]{Теорема}
\newtheorem*{consequense}{Следствие}
\newtheorem*{consequenses}{Следствия}
% \newtheorem{consequensenum}[concount]{Следствие}
\newtheorem*{lemma}{Лемма}
% \newtheorem{lemmanum}[lemcount]{Лемма}
\newtheorem*{subtheorem}{Утверждение}
% \newtheorem{subtheoremnum}[subthcount]{Утверждение}
\newtheorem*{algorithm}{Алгоритм}
\newtheorem*{properties}{Свойства}
\newtheorem*{properties1}{Свойство}


\theoremstyle{remark}
\newtheorem*{remark}{Замечание}
\newtheorem*{offer}{Предложение}
% \newtheorem{offernum}[offercount]{Предложение}
\begin{document}
  \newpage
  \begin{center}
    Билет 1
  \end{center}
  \begin{definition}
    Упорядоченный набор - функция, которая ставит в соответствие каждому элементу множества $\{1, \ldots, n\}$ элемент из множества $\{a_1,\ldots,a_n\}$ : $1\rightarrow a_1$, $\ldots$, $n\rightarrow a_n$.
  \end{definition}
  Декартовое произведение множеств $A_1\times\ldots\times A_n=(a_1, \ldots, a_n)$ : $a_i\in A_i$.
  \begin{definition}
    Пусть функция $f$ определена на $A_1\times\ldots\times A_n$, тогда $f$ - $n$-местная функция.
  \end{definition}
  \begin{definition}
    Множество $B_n=E_2\times\ldots\times E_n$, где $E_i=\{0,1\}$, называется $n$-мерным булевым кубом.
  \end{definition}
  \begin{definition}
    Функция $f:B_n \to E_2$ называется функцией алгебры логики. Множество всех таких функций обозначим $P_2$.
  \end{definition}
  Представление функции $f(x_1, \ldots, x_n)$ в виде таблицы, имеющей $n+1$ столбец:\\
  $\begin{matrix}
    x_1 & \ldots & x_{n-1} & x_n & f\\
    0 & \ldots & 0 & 0 & 0\\
    0 & \ldots & 0 & 0 & 1\\
    0 & \ldots & 0 & 1 & 0\\
    \vdots & \null & \vdots & \vdots & \vdots\\
    1 & \ldots & 1 & 1 & 1 
  \end{matrix}$\\
  Так как число различных первых $n$ столбцов $2^n$, так как в каждой ячейке одного столбца может быть либо 0, либо 1. $\Longrightarrow$ число функций будет $2^{2^n}$, так как для каждого набора значение функции может быть либо 0, либо 1.
  \begin{definition}
    Переменная $x_i$ называется существенной, если существуют наборы $\alpha_1$, $\ldots$, $\alpha_{i-1}$, $1$, $\alpha_{i+1}$, $\ldots$, $\alpha_n$ и $\alpha_1$, $\ldots$, $\alpha_{i-1}$, $0$, $\alpha_{i+1}$, $\ldots$, $\alpha_n$, на которых функция принимает различные значения. В противном случае переменная $x_i$ называется несущественной (фиктивной).
  \end{definition}
  \begin{definition}
    Пусть $x_i$ - фиктивная переменная, тогда если функция $f(x_1$, $\ldots$, $x_{i-1}$, $x_{i+1}$, $\ldots$, $x_n)=g(x_1$, $\ldots$, $x_{i-1}$, $0$, $x_{i+1}$, $\ldots$, $x_n)$, то функция $g$ называется полученной из $f$ добавлением фиктивной переменной. Функция удаления фиктивной переменной определяется аналогично.
  \end{definition}
  \begin{definition}
    Функция называется симметрической, если при любых перестановках переменных $x_{i_1}, \ldots, x_{i_n}$ значение функции не меняется.
  \end{definition}
  Элементарные функции в алгебре логики:
  \begin{enumerate}
    \item константы 0, 1
    \item тождественный $x$
    \item отрицание $\overline{x}$
    \item конъюнкция $x\wedge y$
    \item дизъюнкция $x\vee y$
    \item имплекация $x\rightarrow y$
    \item штрих Шеффера $x|y$
    \item стрелка Пирса $x\downarrow y$
    \item сложение по модулю 2
    \item эквивалентность
  \end{enumerate}
  \begin{center}
    Билет 2
  \end{center}
  \begin{definition}
    Формула - слово в некотором алфавите $A$.
  \end{definition}
  \begin{definition}
    Алфавит - конечное или бесконечное множество.
  \end{definition}
  \begin{definition}
    Слово - произвольная функция, определённая на начальном отрезке натурального ряда и принимающая на нём значения из $A$.
  \end{definition}
  \begin{definition}
    Пусть $F$ - множество функций алгебры логики, $S$ - множество символов, обозначающих функции из $F$, тогда отображение $\Sigma : S \to F$ - сигнатура для $F$.
  \end{definition}
  \begin{definition}
    Пусть $X=\{x_1, \ldots\}$ - символы переменных.\\
    База индукция: если $x_i$ - символ переменной, то однобуквенное слово, состоящее из $x_i$ - формула в сигнатуре $\Sigma$.\\
    Пусть $s\in S$, $f=\Sigma(s)$ - функция от $n$ переменных, Ф$_1$, $\ldots$, Ф$_n$ - формулы в сигнатуре $\Sigma$, тогда слово $s($Ф$_1$, $\ldots$, Ф$_n)$ - формула в сигнатуре $\Sigma$.  
  \end{definition}
  \begin{definition}
    Пусть Ф - формула, $\tilde{x}$ - упорядоченный набор ($x_{i_1}, \ldots, x_{i_n})$, содержащий все переменные формулы Ф, $\tilde{\alpha}=(\alpha_1, \ldots, \alpha_n)$ - двоичный набор.\\
    База индукции: Ф - однобуквенное слово $x_{i_j}$, тогда Ф$[\tilde{x}, \tilde{\alpha}]=\alpha_j$ - значение формулы на наборе $\tilde{\alpha}$.\\
    Пусть $F$ - $s($Ф$_1, \ldots, $Ф$_n)$, $f=\Sigma(s)$, причём Ф$_1[\tilde{x}, \tilde{\alpha}]=\beta_1$, $\ldots$, Ф$_n[\tilde{x}, \tilde{\alpha}]=\beta_n$, тогда $f(\beta_1, \ldots, \beta_n)$ - значение формулы на наборе значений переменных.
  \end{definition}
  \begin{definition}
    Формулой, определяющей функцию $f$ алгебры логики, определённой на $B_n$, называется формула Ф такая, что $\forall$ набора $\tilde{\alpha}=(\alpha_1$, $\ldots$, $\alpha_n)\in B_n$ $f(\tilde{\alpha})=$Ф$[\tilde{x}, \tilde{\alpha}]$.
  \end{definition}
  \begin{definition}
    Формулы в сигнатуре, представляющие собой переменные, называются вырожденными, остальные - невырожденными. Если функция определяется невырожденной формулой в сигнатуре $\Sigma:S \to F$, то она получена суперпозициями над $F$, где $F$ - множество функций.
  \end{definition}
  \begin{definition}
    (Другое определение суперпозиции) Если одну функцию можно получить с помощью конечного числа применений следующих трёх операций, то данная функция называется функцией, полученной суперпозициями над $F$.\\
    Операции:\\
    \begin{enumerate}
      \item Операция подстановки переменных. Пусть $f(x_1$, $\ldots$, $x_n)\in P_2$, $g(x_1$, $\ldots$, $x_n)$ - функция, определённая на $B_n$, такая, что $g(x_1$, $\ldots$, $x_n)=f(x_{i_1}$, $\ldots$, $x_{i_n})$, где набор ($i_1$, $\ldots$, $i_n$) - набор элементов ($1$, $\ldots$, $n$) (они необязательно различны). Тогда $g$ получена из $f$ операцией подстановки переменных.
      \item Операция подстановки функции в функцию. Пусть $f(x_1$, $\ldots$, $x_n)$, $g(x_1$, $\ldots$, $x_m)$, $h$ определена на $B_{n+m-1}$ и $h(x_1$, $\ldots$, $x_{n+m-1})=f(x_1$, $\ldots$, $x_{n-1}$, $g(x_n$, $\ldots$, $x_{n+m-1}))$, тогда функция $h$ получена из функций $f$ и $g$ операцией подстановки одной функции в другую.
      \item Операция добавления или удаления фиктивных переменных. Пусть $x_i$ - фиктивная переменная, тогда если функция $f(x_1$, $\ldots$, $x_{i-1}$, $x_{i+1}$, $\ldots$, $x_n)=g(x_1$, $\ldots$, $x_{i-1}$, $0$, $x_{i+1}$, $\ldots$, $x_n)$, то функция $g$ называется полученной из $f$ добавлением фиктивной переменной. Функция удаления фиктивной переменной определяется аналогично.
    \end{enumerate}
  \end{definition}
  \begin{center}
    Билет 3
  \end{center}
  \begin{definition}
    Формулы $F_1$ и $F_2$ называются эквивалентными, если они определяют равные функции относительно объединения их переменных. Функции называются равными, если их области определения равны и $\forall x\in D_f(x)$ $f(x)=g(x)$. Слово $F_1=F_2$, если формулы $F_1$ и $F_2$ эквивалентны, называется тождеством. 
  \end{definition}
  Основные тождества:
  \begin{enumerate}
    \item Ассоциативность операций: $\wedge$, $\vee$, $\neg$, $\leftrightarrow$.
    \item Дистрибутивности:
    \begin{enumerate}
      \item $(x\vee y)\wedge z=(x\wedge z)\vee(y\wedge z)$
      \item $(x\wedge y)\vee z=(x\vee z)\wedge(y\vee z)$
      \item $(x+y)\cdot z=x\cdot z+y\cdot z$
    \end{enumerate}
    \item Тождества для отрицания: 
    \begin{enumerate}
      \item $\overline{\overline{x}}=x$
      \item $\overline{x\wedge y}=\overline{x}\vee \overline{y}$
      \item $\overline{x\vee y}=\overline{x}\wedge \overline{y}$
      \item $x\cdot\overline{x}=0$
      \item $x\vee\overline{x}=1$
      \item $\overline{x\rightarrow y}=x\cdot\overline{y}$
    \end{enumerate}
    \item Тождества для эдентичных операндов
    \item Тождества с константным операндом
  \end{enumerate}
  \begin{definition}
    Функция $g$ называется двойственной к $f$, если $g(x_1$, $\ldots$, $x_n)=\overline{f}(\overline{x_1}$, $\ldots$, $\overline{x_n})$. Обозначение $g=f^*$.
  \end{definition}
  \begin{definition}
    Если функция двойственна к самой себе, то она называется самодвойственной.
  \end{definition}
  \begin{theorem}(принцип двойственности)
    Если Ф - формула в сигнатуре $\Sigma: S\to F$, определяющая некоторую функцию $g$, то эта формула в сигнатуре $\Sigma^*: S\to F^*$ определяет двойственную функцию $g^*$.
  \end{theorem}
  \begin{proof}
    База индукции: пусть $x_i$ - символ переменной, тогда однобуквенное слово, состоящее из $x_i$ - формула в сигатуре $\Sigma$, определяющая одноместную функцию $g$. Эта формула в сигнатуре $\Sigma^*$ имеет вид $\overline{x_i}$, то есть она определяет функцию, двойственную к $g$.\\
    Пусть $s\in S$, $f=\Sigma(s)$ - формула от $n$ переменных, Ф$_1$, $\ldots$, Ф$_n$ - формулы в сигнатуре $\Sigma$, тогда слово $s$(Ф$_1$, $\ldots$, Ф$_n$) - формула в сигнатуре $\Sigma$. В $\Sigma^*(s)=(\Sigma(s))^*=(\Sigma(s$(Ф$_1$, $\ldots$, Ф$_n$)))$^*=f^*$, то есть данная формула определяет в двойственной сигнатуре двойственнную функцию.
  \end{proof}
  \begin{center}
    Билет 4
  \end{center}
  \begin{definition}
    Выражение $f(x_1$, $\ldots$, $x_n)=\bigvee\limits_{(\sigma_1, \ldots,\sigma_n):f(\sigma_1, \ldots,\sigma_n)=1}x_1^{\sigma_1}\cdot\ldots\cdot x_n^{\sigma_n}$ называется совершенной дизъюнктивной нормальной формой. $x_i^{\sigma_i}=$
    $\begin{cases}
      x_i,\sigma_i=1\\
      \overline{x_i},\sigma_i=0
    \end{cases}$.
  \end{definition}
  \begin{theorem}
    Для любой функции $f(x_1$, $\ldots$, $x_n)$ алгебры логики верно равенство:\\
    $f(x_1$, $\ldots$, $x_n)=\bigvee\limits_{(\sigma_1, \ldots,\sigma_m)\in B_m}x_1^{\sigma_1}\cdot\ldots\cdot x_m^{\sigma_m}\cdot f(\sigma_1$, $\ldots$, $\sigma_m$, $\sigma_{m+1}$, $\ldots$, $\sigma_n)$.
  \end{theorem}
  \begin{proof}
    Рассмотрим прозвольный набор $(\alpha_1$, $\ldots$, $\alpha_m$), если $(\alpha_1$, $\ldots$, $\alpha_m)\neq(\sigma_1$, $\ldots$, $\sigma_m$), то $\exists \alpha_i\neq\sigma_i$ $\Longrightarrow$ $\alpha_i^{\sigma_i}=0$ $\Longrightarrow$ данное слагаемое будет равно нулю. Тогда единственным не нулевым членом будет $(\alpha_1^{\alpha_1}\cdot\ldots\cdot\alpha_m^{\alpha_m})\cdot f(\alpha_1$, $\ldots$, $\alpha_m$, $\alpha_{m+1}$, $\ldots$, $\alpha_n)=f(\alpha_1$, $\ldots$, $\alpha_n$).
  \end{proof}
  \begin{theorem}
    Любую функцию алгебры логики можно представить с помощью суперпозиций конъюнкции, дизъюнкции и отрицания.
  \end{theorem}
  \begin{proof}
    Так как любая функция алгебры логики, кроме тождественного нуля, реализуется совершенной д.н.ф., значит она представима суперпозициями конъюнкции, дизъюнкции и отрицания. Тождественный ноль можно представить так: $x\wedge\overline{x}=0$.
  \end{proof}
  \begin{theorem}
    Любая функция алгебры логики, кроме тождественной единицы, представима в виде совершенной конъюнктивной нормальной формы.
  \end{theorem}
  \begin{proof}
    Так как любая функция алгебры логики, кроме тождественного нуля, представима в виде совершенной д.н.ф., тогда по принципу двойственности\\
    $f(x_1$, $\ldots$, $x_n)=\bigwedge\limits_{(\sigma_1, \ldots,\sigma_n):f^*(\sigma_1, \ldots,\sigma_n)=1}x_1^{\sigma_1}\vee\ldots\vee x_n^{\sigma_n}$ $\Longrightarrow$\\
    $f(x_1$, $\ldots$, $x_n)=\bigwedge\limits_{(\delta_1, \ldots,\delta_n):f(\delta_1, \ldots,\delta_n)=1}x_1^{\overline{\delta}_1}\vee\ldots\vee x_n^{\overline{\delta}_n}$.
  \end{proof}
  \begin{center}
    Билет 5
  \end{center}
  \begin{definition}
    Система функций называется полной в $P_2$, если через них выражаются все функции в $P_2$.
  \end{definition}
  \begin{example}
    \begin{enumerate}
      \item $\wedge$ и $\neg$
      \item $\vee$ и $\neg$
      \item $x|y$
      \item $x\downarrow y$
    \end{enumerate}
  \end{example}
  \begin{definition}
    Полиномы по модулю 2 вида: $\sum\limits_{\{i_1,\dots,i_s\}\subseteq{1,\ldots,n}}a_{i_1,\ldots,i_s}\cdot x_{i_1}\cdot\ldots\cdot x_{i_s}$ называются полиномами Жегалкина.
  \end{definition}
  \begin{theorem}(Жегалкина)\\
    Любая функция алгебры логики представима полиномом Жегалкина, причём единственным образом.
  \end{theorem}
  \begin{proof}
    Так как в каждом мономе полинома Жегалкина $n$ перменных, каждая из которых может быть либо 0, либо 1, а коэффициент перед каждым мономом может принимать значение 0 или 1 $\Longrightarrow$ всего есть $2^{2^n}$ различных полиномов Жегалкина.\\
    Пусть два различных полинома Жегалкина задают одну функцию, тогда мы получим ненулевой полином, задающий нулевую константу $\Longrightarrow$ противоречие $\Longrightarrow$  Любая функция алгебры логики представима полиномом Жегалкина, причём единственным образом.
  \end{proof}
  \begin{center}
    Билет 6
  \end{center}
  \begin{definition}
    Множество функций, которые можно пулучить из данного множества $M$ функций алгебры логики, называется замыканием множества $M$ и обозначается $[M]$.
  \end{definition}
  \begin{example}
    \begin{enumerate}
      \item $P_2=[P_2]$
      \item [{1, $x+y$}] - множество линейных функций
    \end{enumerate}
  \end{example}
  \begin{properties}
    \begin{enumerate}
      \item $M\subseteq[M]$
      \item $[[M]]=[M]$
      \item Если $M_1\subseteq M_2$, то $[M_1]\subseteq[M_2]$
      \item $[M_1]\cup[M_2]\subseteq[M_1\cup M_2]$
    \end{enumerate}
  \end{properties}
  \begin{proof}
    \begin{enumerate}
      \item По определению замыкания.
      \item Из первого следует, что $[M]\subseteq[[M]]$, а $[[M]]\subseteq[M]$, так как в противном случае существовала бы функция, которая не выражается суперпозициями функций из $M$, но выражается суперпозициями функций, которые выражаются суперпозициями функций из $M$, а значит она выражается суперпозициями из $M$ $\Longrightarrow$ противоречие.
      \item Если функция получается суперепозициями из $M_1$, то её можно получить суперпозициями из $M_2$, так как все функции $M_1$ являются функциями $M_2$.
      \item Пусть функция $f\in[M_1]\cap[M_2]$, тогда она получается суперпозициями из $M_1$ или из $M_2$, пусть для определённости она выражается суперпозициями из $M_1$, но тогда её можно получить суперпозициями из $M_1\cup M_2$, то есть $f\in[M_1\cup M_2]$
    \end{enumerate}
  \end{proof}
  \begin{definition}
    Класс функций $M$ называется замкнутым, если $[M]=M$.
  \end{definition}
  \begin{example}
    \begin{enumerate}
      \item $P_2=[P_2]$
      \item $L=[L]$, $L$ - множество линейных функций.
    \end{enumerate}
  \end{example}
  \begin{center}
    Билет 7
  \end{center}
  \begin{definition}
    Функция $f$ называется функцией, сохраняющей ноль, если на наборе из нулей она принимает значение 0. 
  \end{definition}
  \begin{definition}
    Функция $f$ называется функцией, сохраняющей единицу, если на наборе из единиц она принимает значение 1. 
  \end{definition}
  Класс функций, сохраняющих ноль, обозначим $T_0$, а класс функций, сохраняющих единицу, обозначим $T_1$.
  \begin{theorem}
    Классы $T_0$ и $T_1$ замкнуты.
  \end{theorem}
  \begin{proof}
    \begin{enumerate}
      \item Операция подстановки переменных:\\
      $g(x_1$, $\ldots$, $x_n)=f(x_{i_1}$, $\ldots$, $x_{i_n})$, если функция $f$ сохраняла ноль, то и функция $g$ будет сохранять ноль, если функция $f$ сохраняла единицу, то и функция $g$ будет сохранять единицу.
      \item Операция подстановки одной функции в другую:\\
      $h(x_1$, $\ldots$, $x_{n+m-1})=f(x_1$, $\ldots$, $x_{n-1}$, $g(x_n$, $\ldots$, $x_{n+m-1}))$, если функции $f$ и $h$ сохраняли ноль, то и функция $g$ будет сохранять ноль, если функции $f$ и $g$ сохраняли единицу, то и функция $h$ будет сохранять единицу.
      \item Операция добавления или удаления фиктивной переменной, не влияет на способность функции сохранять ноль или сохранять единицу.
    \end{enumerate}
    Следовательно суперпозициями мы не сможем получить функцию, не принадлежащую данному классу $\Longrightarrow$ классы $T_0$ и $T_1$ - замкнуты. 
  \end{proof}
  \begin{center}
    Билет 8
  \end{center}
  Класс самодвойственных функций обозначим $S$.
  \begin{theorem}
    Класс $S$ замкнут.
  \end{theorem}
  \begin{proof}
    \begin{enumerate}
      \item Операция подстановки переменных:\\
      Пусть $f(x_1$, $\ldots$, $x_n)\in S$, $g(x_1$, $\ldots$, $x_n)=f(x_{i_1}$, $\ldots$, $x_{i_n})$, тогда $\overline{g}(\overline{x}_1$, $\ldots$, $\overline{x}_n)=\overline{f}(\overline{x}_{i_1}$, $\ldots$, $\overline{x}_{i_n})=f(x_{i_1}$, $\ldots$, $x_{i_n})=g(x_1$, $\ldots$, $x_n)$ $\Longrightarrow$ $g$ - самодвойственная функция.
      \item Операция подстановки функции в функцию:\\
      Пусть $f(x_1$, $\ldots$, $x_n)\in S$, $g(x_1$, $\ldots$, $x_m)\in S$, $h(x_1$, $\ldots$, $x_n$, $x_{n+1}$, $\ldots$, $x_{n+m-1})=f(x_1$, $\ldots$, $x_{n-1}$, $g(x_n$, $\ldots$, $x_{n+m-1}))$, тогда $\overline{h}(\overline{x}_1$, $\ldots$, $\overline{x}_n$, $\overline{x}_{n+1}$, $\ldots$, $\overline{x}_{m+n-1})=\overline{f}(\overline{x}_1$, $\ldots$, $\overline{x}_{n-1}$, $g(\overline{x}_{n}$, $\ldots$, $\overline{x}_{m+n-1}))=\overline{f}(\overline{x}_1$, $\ldots$, $\overline{x}_{n-1}$, $\overline{g}(x_{n}$, $\ldots$, $x_{m+n-1}))=f(x_1$, $\ldots$, $x_{n-1}$, $g(x_{n}$, $\ldots$, $x_{m+n-1}))=h(x_1$, $\ldots$, $x_n$, $x_{n+1}$, $\ldots$, $x_{m+n-1})$ $\Longrightarrow$ $h$ - самодвойственная функция.
      \item Операция добавления или удаления фиктивных переменных:\\
      Пусть $f(x_1$, $\ldots$, $x_n)\in S$, $g(x_1$, $\ldots$, $x_{i-1}$, $0$, $x_{i+1}$, $\ldots$,  $x_n) = f(x_1$, $\ldots$, $x_n)=g(x_1$, $\ldots$, $x_{i-1}$, $1$, $x_{i+1}$, $\ldots$,  $x_n)$, тогда $\overline{g}(\overline{x}_1$, $\ldots$, $\overline{x}_{i-1}$, $1$, $\overline{x}_{i+1}$, $\ldots$,  $\overline{x}_n) = f(x_1$, $\ldots$, $x_n)=g(x_1$, $\ldots$, $x_{i-1}$, $0$, $x_{i+1}$, $\ldots$,  $x_n)$ $\Longrightarrow$ $g$ - самодвойственная функция. 
    \end{enumerate}
  \end{proof}
  \begin{theorem}
    Если функция $f$ не является самодвойственной, то с помощью неё и функции отрицания можно получить константу.
  \end{theorem}
  \begin{proof}
    Пусть $f(x_1$, $\ldots$, $x_n)\notin S$, тогда существует набор $(\alpha_1$, $\ldots$, $\alpha_n$) :\\ 
    \begin{center}
      $f(\alpha_1$, $\ldots$, $\alpha_n)=f(\overline{\alpha}_1$, $\ldots$, $\overline{\alpha}_n)$.\\
    \end{center} 
    Пусть $\phi_i=x^{\alpha_i}$, $\phi(x)=f(\phi_1(x)$, $\ldots$, $\phi_n(x))$,\\ 
    \begin{center}
      тогда $\phi(0)= f(0^{\alpha_1}, \ldots, 0^{\alpha_n})=f(\overline{\alpha}_1, \ldots, \overline{\alpha}_n)=f(\alpha_1, \ldots, \alpha_n)=f(1^{\alpha_1}, \ldots, 1^{\alpha_n})=\phi(1)\Longrightarrow$\\
    \end{center}
      $\Longrightarrow$ $\phi(x)$ - константа, полученная из несамодвойственной функции и отрицания.
  \end{proof}
  \begin{center}
    Билет 9
  \end{center}
  \begin{definition}
    Пусть $\tilde{\alpha}=(\alpha_1$, $\ldots$, $\alpha_n)$, $\tilde{\beta}=(\beta_1$, $\ldots$, $\beta_n)$ - двоичные наборы, тогда $\tilde{\alpha}\leqslant\tilde{\beta}$, если $\forall i=\overline{1,n}$ $\alpha_i\leqslant\beta_i$.
  \end{definition}
  \begin{definition}
    Функция алгебры логики называется монотонной, если $\forall$ двоичных наборов $\tilde{\alpha}$ и $\tilde{\beta}$ таких, что $\tilde{\alpha}\leqslant\tilde{\beta}$, $f(\tilde{\alpha})\leqslant f(\tilde{\beta})$.
  \end{definition}
  \begin{theorem}
    Класс $M$ монотонных функций - замкнут.
  \end{theorem}
  \begin{proof}
    \begin{enumerate}
      \item Операция подстановки переменных:\\
      $g(x_1$, $\ldots$, $x_n)=f(x_{i_1}$, $\ldots$, $x_{i_n})$, если функция $f$ монотонна, то\\ $\forall\tilde{\alpha}=(\alpha_1$, $\ldots$, $\alpha_n)$ и $\tilde{\beta}=(\beta_1$, $\ldots$, $\beta_n)$ : $\tilde{\alpha}\leqslant\tilde{\beta}$, $f(\tilde{\alpha})\leqslant f(\tilde{\beta})$ $\Longrightarrow$ $\alpha_1\leqslant\beta_1$, $\ldots$, $\alpha_n\leqslant\beta_n$ $\Longrightarrow$\\
      $\Longrightarrow$ $\alpha_{i_1}\leqslant\beta_{i_1}$, $\ldots$, $\alpha_{i_n}\leqslant\beta_{i_n}$ $\Longrightarrow$ $f(\alpha_{i_1}$, $\ldots$, $\alpha_{i_n})\leqslant f(\beta_{i_1}$, $\ldots$, $\beta_{i_n})$ $\Longrightarrow$\\
      $\Longrightarrow$ $g(\alpha_{1}$, $\ldots$, $\alpha_{n})=f(\alpha_{i_1}$, $\ldots$, $\alpha_{i_n})\leqslant f(\beta_{i_1}$, $\ldots$, $\beta_{i_n})=g(\beta_{i_1}$, $\ldots$, $\beta_{i_n})$ $\Longrightarrow$ $g$ - монотонна.
      \item Операция подстановки одной функции в другую:\\ 
      $f(x_{1}$, $\ldots$, $x_{n})$, $g(x_{1}$, $\ldots$, $x_{m})$ - монотонные функции, $h(x_1$, $\ldots$, $x_{n+m-1})=f(x_1$, $\ldots$, $x_{n-1}$, $g(x_n$, $\ldots$, $x_{n+m-1}))$, так как функции $f$ и $g$ монотонны, $\forall\tilde{\alpha}=(\alpha_1$, $\ldots$, $\alpha_{m+n-1})$ и $\tilde{\beta}=(\beta_1$, $\ldots$, $\beta_{m+n-1})$ : $\tilde{\alpha}\leqslant\tilde{\beta}$, $f(\tilde{\alpha})\leqslant f(\tilde{\beta})$ и $g(\alpha_n$, $\ldots$, $\alpha_{m+n-1})=g(\beta_n$, $\ldots$, $\alpha_{m+n-1})$ $\Longrightarrow$\\
      $(\alpha_1$, $\ldots$, $\alpha_{n-1}$, $g(\alpha_n$, $\ldots$, $\alpha_{m+n-1}))\leqslant (\beta_1$, $\ldots$, $\beta_{n-1}$, $g(\beta_n$, $\ldots$, $\beta_{n+m-1}))$ $\Longrightarrow$ $h(\alpha_1$, $\ldots$, $\alpha_{m+n-1})=f(\alpha_1$, $\ldots$, $\alpha_{n-1}$, $g(\alpha_n$, $\ldots$, $\alpha_{m+n-1}))\leqslant f(\beta_1$, $\ldots$, $\beta_{n-1}$, $g(\beta_n$, $\ldots$, $\beta_{n+m-1}))=h(\beta_1$, $\ldots$, $\beta_{m+n-1})$.
      \item Операция добавления или удаления фиктивных переменных:\\$f(x_1$, $\ldots$, $x_{i-1}$, $x_{i+1}$, $\ldots$, $x_n)=g(x_1$, $\ldots$, $x_{i-1}$, $0$, $x_{i+1}$, $\ldots$, $x_n)$, так как $f$ монотонна $\Longrightarrow$ $\forall\tilde{\alpha}=(\alpha_1$, $\ldots$, $\alpha_{i-1}$, $\alpha_{i+1}$, $\ldots$, $\alpha_n)$ и $\tilde{\beta}=(\beta_1$, $\ldots$, $\beta_{i-1}$, $\beta_{i+1}$, $\ldots$, $\beta_n)$ : $\tilde{\alpha}\leqslant\tilde{\beta}$,\\верно $f(\alpha_1$, $\ldots$, $\alpha_{i-1}$, $\alpha_{i+1}$, $\ldots$, $\alpha_n)\leqslant f(\beta_1$, $\ldots$, $\beta_{i-1}$, $\beta_{i+1}$, $\ldots$, $\beta_n)$.\\ Тогда $\tilde{\alpha}$, с добавленной фиктивной переменной, $\leqslant$ $\tilde{\beta}$, с добавленной фиктивной переменной $\Longrightarrow$ $g(\alpha_1$, $\ldots$, $\alpha_{i-1}$, $0$, $\alpha_{i+1}$, $\ldots$, $\alpha_n)=f(\alpha_1$, $\ldots$, $\alpha_{i-1}$, $\alpha_{i+1}$, $\ldots$, $\alpha_n)\leqslant f(\beta_1$, $\ldots$, $\beta_{i-1}$, $\beta_{i+1}$, $\ldots$, $\beta_n)=g(\beta_1$, $\ldots$, $\beta_{i-1}$, $0$, $\beta_{i+1}$, $\ldots$, $\beta_n)$.
    \end{enumerate}
    Следовательно, суперпозициями мы не сможем получить функцию, не принадлежащую данному классу $\Longrightarrow$ класс $M$ замкнут. 
  \end{proof}
  \begin{theorem}
    Если $f$ - немонотонная функция, то из неё и констант можно получить отрицание.
  \end{theorem}
  \begin{proof}
    Пусть $f(x_1$, $\ldots$, $x_n)$ - немонотонная функция, тогда $\exists\tilde{\alpha}$ и $\tilde{\beta}$ : $\tilde{\alpha}\leqslant\tilde{\beta}$ и $f(\tilde{\alpha})=1$, а $f(\tilde{\beta})=0$. Так как наборы различны, то $\exists\alpha_{i_1}=\ldots=\alpha_{i_k}=0$ и $\beta_{i_1}=\ldots=\beta_{i_k}=1$,\\ а $\forall j\in(1$, $\ldots$, $n)\setminus(i_1$, $\ldots$, $i_k$) $\alpha_j=\beta_j$.\\ Пусть наборы $\tilde{\gamma}_0$, $\ldots$, $\tilde{\gamma}_k$ на позициях $(1$, $\ldots$, $n)\setminus(i_1$, $\ldots$, $i_k$) совпадает со значениями  набора $\tilde{\alpha}$, на позициях $i_1$, $\ldots$, $\i_j$ набор $\tilde{\gamma}_j=1$, а на позициях $i_{j+1}$, $\ldots$, $i_k$ принимает значение 0, тогда $\tilde{\gamma}_0=\tilde{\alpha}$, а $\tilde{\gamma_k}=\tilde{\beta}$ $\Longrightarrow$ $f(\tilde{\gamma}_0)=1$, $f(\tilde{\gamma}_k)=0$ $\Longrightarrow$ $\exists\tilde{\gamma}_j$ : $f(\tilde{\gamma_j})=0$, а $f(\tilde{\gamma_{j-1}})=1$ $\Longrightarrow$\\
    $\Longrightarrow$ $\tilde{\gamma}_{j-1}=(\delta_1$, $\ldots$, $\delta_{i_j-1}$, $0$, $\delta_{i_j+1}$, $\ldots$, $\delta_n)$, $\tilde{\gamma}_j=(\delta_1$, $\ldots$, $\delta_{i_j-1}$, $1$, $\delta_{i_j+1}$, $\ldots$, $\delta_n)$.\\
    Тогда функция $\phi(f(\delta_1$, $\ldots$, $\delta_{i_j-1}$, $x$, $\delta_{i_j+1}$, $\ldots$, $\delta_n))$, при $x=0$ функция равна 1, а при $x=1$, функция равна 0, то есть $\phi=\overline{x}$, а так как она получена с помощью функции $f$ и констант, значит, это искомая функция.
  \end{proof}
  \begin{center}
    Билет 10
  \end{center}
  \begin{definition}
    Функция $f$ называется линейной, если она представима полиномом Жегалкина степени 1.
  \end{definition}
  \begin{theorem}
    Класс $L$ линейных функций замкнут.
  \end{theorem}
  \begin{proof}
    \begin{enumerate}
      \item Операция подстановки переменных:\\
      $g(x_1$, $\ldots$, $x_n)=f(x_{i_1}$, $\ldots$, $x_{i_n})$, если функция $f$ линейна, то $\forall\tilde{\alpha}=(\alpha_1$, $\ldots$, $\alpha_n)$ $f(\tilde{\alpha})=c_0+c_1\alpha_1+\ldots+c_n\alpha_n$, тогда\\
      $g(\alpha_1$, $\ldots$, $\alpha_n)=c_0+c_1\alpha_{i_1}+\ldots+c_n\alpha_{i_n}$ $\Longrightarrow$ $g$ - линейная функция.
      \item Операция подстановки одной функции в другую:\\ 
      $f(x_{1}$, $\ldots$, $x_{n})$, $g(x_{1}$, $\ldots$, $x_{m})$ - линейные функции, $h(x_1$, $\ldots$, $x_{n+m-1})=f(x_1$, $\ldots$, $x_{n-1}$, $g(x_n$, $\ldots$, $x_{n+m-1}))$, так как функции $f$ и $g$ линейны, $\forall\tilde{\alpha}=(\alpha_1$, $\ldots$, $\alpha_{m+n-1})$ $f(\alpha_1$, $\ldots$, $\alpha_n)=c_0+c_1\alpha_1+\ldots+c_n\alpha_n$, $g(\alpha_1$, $\ldots$, $\alpha_m)=c'_0+c'_1\alpha_1+\ldots+c'_n\alpha_n$ $\Longrightarrow$\\
      $\Longrightarrow$ $h(\alpha_1$, $\ldots$, $\alpha_{n+m-1})=c_0+c_1\alpha_1+\ldots+c_{n-1}\alpha_{n-1}+c_ng(\alpha_n$, $\ldots$, $\alpha_{m+n-1})=\\=c_0+c_1\alpha_1+\ldots+c_{n-1}\alpha_{n-1}+c_n(c'_1\alpha_n+\ldots+c'_m\alpha_{m+n-1})$ $\Longrightarrow$ функция $h$ является линейной.
      \item Операция добавления или удаления фиктивных переменных:\\$f(x_1$, $\ldots$, $x_{i-1}$, $x_{i+1}$, $\ldots$, $x_n)=g(x_1$, $\ldots$, $x_{i-1}$, $0$, $x_{i+1}$, $\ldots$, $x_n)$, так как $f$ линейна $\Longrightarrow$ $\forall\tilde{\alpha}=(\alpha_1$, $\ldots$, $\alpha_{i-1}$, $\alpha_{i+1}$, $\ldots$, $\alpha_n)$ $f(\tilde{\alpha})=c_0+c_1\alpha_1+\ldots+c_{i-1}\alpha_{i-1}+c_{i+1}\alpha_{i+1}+\ldots+c_n\alpha_n$, тогда очевидно, что $g(\alpha_1$, $\ldots$, $\alpha_{i-1}$, $0$, $\alpha_{i+1}$, $\ldots$, $\alpha_n)$ тоже линейная функция. 
    \end{enumerate}
    Следовательно, суперпозициями мы не сможем получить функцию, не принадлежащую данному классу $\Longrightarrow$ класс $L$ замкнут. 
  \end{proof}
  \begin{theorem}
    Если функция $f$ нелинейна, то из неё, констант и отрицания можно получить конъюнкцию.
  \end{theorem}
  \begin{proof}
    Пусть $f(x_1$, $\ldots$, $x_n)$ - нелинейная функция, тогда полином Жегалкина без ограничения общности имеет вид: $x_1x_2f_1(x_3$, $\ldots$, $x_n)+x_1f_2(x_3$, $\ldots$, $x_n)+x_2f_3(x_3$, $\ldots$, $x_n)+f_4(x_3$, $\ldots$, $x_n)$. Так как $f1$ не является тождественно нулевой функцией, существует набор $(\alpha_3$, $\ldots$, $\alpha_n)$ : $f_1(\alpha_3$, $\ldots$, $\alpha_n)=1$, тогда $f=x_1x_2+\alpha x_1+\beta x_2+\gamma$ $\Longrightarrow$\\
    $\Longrightarrow$ $f(x_1+\alpha$, $x_2+\beta)=(x_1+\alpha)(x_2+\beta)+\alpha(x_1+\alpha)+\beta(x_2+\beta)+\gamma=x_1x_2+\alpha\beta\gamma$, если $\alpha\beta\gamma=1$, то возьмём $\overline{f}(x_1+\alpha$, $x_2+\beta)=x_1x_2$,  так как данная функция получена из $f$ с помощью констант и отрцания, значит это искомая функция.
  \end{proof}
  \begin{center}
    Билет 11
  \end{center}
  \begin{theorem}
    Система функций полна тогда и только тогда, когда она не содержится ни в одном из классов $T_0$, $T_1$, $S$, $M$, $L$.
  \end{theorem}
  \begin{proof}
    $\underline{\Longrightarrow}$ Если ситсема $F$ функций алгебры логики полна, то $[F]=P_2$. Предположим, что $F\subseteq K$, где $K$ - один из этих классов, тогда $[F]\subseteq[K]\neq P_2$ - противоречие.\\
    $\underline{\Longleftarrow}$ Пусть $F$ не лежит ни в одном из этих классов, тогда $\exists f_1$, $f_2$, $f_3$, $f_4$, $f_5$ : $f_1\notin T_0$, $f_2\notin T_1$, $f_3\notin S$, $f_4\notin M$, $f_5\notin L$.\\
    Рассмотрим $f_1\notin T_0$, тогда $f_1(0$, $\ldots$, $0)=1$. Есть два случая:
    \begin{enumerate}
      \item Пусть $f_1\notin T_1$, тогда $\phi(x)=f_1(x$, $\ldots$, $x)=\overline{x}$, то есть мы получили из $f_1$ функцию отрицания. Тогда по лемме о несамодвойственной функции из $f_3$ и $\overline{x}$ можно получить константы.
      \item Пусть $f_1\in T_1$, тогда $\phi(x)=f_1(x$, $\ldots$, $x)=1$, то есть $\phi(x)$ - константа 1. Рассмотрим $f_2\notin T_1$, тогда $f_2(f_1(x$, $\ldots$, $x))=0$, то есть мы получили константу 0.
    \end{enumerate}
    Тогда по лемме о немонотонной функции из $f_4$ и констант можно получить $\overline{x}$, а по лемме о нелинейной функции из $f_5$, $\overline{x}$ и констант можно получить $x\wedge y$, то есть мы получим полную систему {$x\wedge y$, $\overline{x}$}.
  \end{proof}
  \begin{center}
    Билет 12
  \end{center}
  \begin{definition}
    Класс $K$ функций алгебры логики называется предполным, если $[K]\neq P_2$ и если $f\in P_2\setminus K$, то $[\{f\}\cup K]=P_2$.
  \end{definition}
  \begin{theorem}
    В $P_2$ нет предполных классов, отличных от $T_0$, $T_1$, $S$, $M$, $L$.
  \end{theorem}
  \begin{proof}
    Пусть класс $K$ - предполный класс, отличный от данных пяти классов. Этот класс замкнут, так как в противном случае можно было бы выбрать функцию $f$ : $f\in[K]$ и $f\notin K$, тогда $[\{f\}\cup K]=[K]$, но так как класс $K$ является предполным, то $[K]=P_2$ $\Longrightarrow$ противоречие с тем, что класс $K$ не является полным.\\
    Так как класс $K$ замкнут, то он содержится в одном из классов $T_0$, $T_1$, $S$, $M$, $L$ (обозначим этот класс $Q$), иначе по теореме Поста он был бы полным, а он по условию таким не является. Пусть класс $K$ не совпадает с классом $Q$, тогда $\exists f\in Q\setminus K$ $\Longrightarrow$ $[\{f\}\cup K]\subseteq[Q]\neq P_2$ - противоречие.\\
    Пусть $f\in P_2\setminus Q$, тогда если $[Q\cup\{f\}]=[Q']\neq P_2$, то $Q'$ содержится в одном из оставшихся классов, что невозможно, а значит, класс $Q$ является предполным.
  \end{proof}
  \begin{center}
    Билет 13
  \end{center}
  \begin{theorem}
    В любой полной системе алгебры логики можно выделить полную подсистему, состоящую из 4 функций.
  \end{theorem}
  \begin{proof}
    Пусть система $F$ полна, выберем в ней функции $f_1$, $f_2$, $f_3$, $f_4$, $f_5$ : $f_1\notin T_0$, $f_2\notin T_1$, $f_3\notin S$, $f_4\notin M$, $f_5\notin L$, по теореме Поста система из этих функций полна. Если $f_1\in T_1$, тогда $f_1\notin S$, тогда функцию $f_3$ можно выбрать равной $f_1$, а если $f_1(1$, $\ldots$, $1)=0$, то $f_1\notin M$, то есть $f_4$ можно выбрать равной $f_1$ $\Longrightarrow$ в обоих случаях мы получаем полную систему из четырёх функций.
  \end{proof}
  \begin{center}
    Билет 14
  \end{center}
  \begin{definition}
    Пусть $K$ - замкнутый класс, $F$ - система функций данного класса, тогда $F$ называется полной, если $[F]=K$.
  \end{definition}
  \begin{definition}
    Система функций некоторого класса $K$ называется базисом, если она полна в $K$, но каждая её собствееная подсистема неполна в $K$.
  \end{definition}
  \begin{example}
    \{0, 1, $x_1\cdot x_2$, $x_1\vee x_2$\} - базис в $M$
  \end{example}
  \begin{theorem}
    Каждый замкнутый класс функций алгебры логики имеет конечный базис. (Без доказательства)
  \end{theorem}
  \begin{theorem}
    Число замкнутых классов в $P_2$ счётно. (Без доказательства)
  \end{theorem}
  \begin{center}
    Билет 15
  \end{center}
  \begin{definition}
    Отображение $f:E_k\times\ldots\times E_k\to E_k$ - функция $k$-значной логики.
  \end{definition}
  Элементарные функции:
  \begin{enumerate}
    \item $\overline{x}=x+1(mod$ $k)$
    \item $\sim x=k-1-x$
    \item $J_i(x)=\begin{matrix}
      k-1\textup{, если} x=i\\
      0\textup{, если} x\neq i
    \end{matrix}$
    \item $j_i(x)=\begin{matrix}
      1\textup{, если} x=i\\
      0\textup{, если} x\neq i
    \end{matrix}$
    \item $min(x_1$, $x_2)$
    \item $max(x_1$, $x_2)$
    \item $x_1\cdot x_2(mod$ $k$)
    \item $x_1+x_2(mod$ $k$)
  \end{enumerate}
  \begin{definition}
    Отображение $\Sigma:S\to F$, где $S$ - множество символов, обозначующих функции из $P_k$, а $F$ - множество функций в $P_k$ называется сигнатурой.
  \end{definition}
  \begin{definition}
    База индукции: пусть $x_i$ - символ переменной, тогда однобуквенное слово, состоящее из $x_i$ - формула в сигнатуре.\\
    Пусть $s\in S$, $f=\Sigma(s)$ - функция от $n$ переменных, Ф$_1$, $\ldots$, Ф$_n$ - формулы в сигнатуре $\Sigma$, тогда слово $s($Ф$_1$, $\ldots$, Ф$_n$) - формула в сигнатуре $\Sigma$.
  \end{definition}
  \begin{definition}
    Пусть Ф - формула, $\tilde{x}=(x_{i_1}$, $\ldots$, $x_{i_n})$ - упорядоченный набор, содержащий все переменные формулы Ф, $\tilde{\alpha}=(\alpha_1$, $\ldots$, $\alpha_n)$ - двоичный набор.\\
    База индукции: Ф - однобуквенное слово $x_{i_j}$, тогда Ф$[\tilde{x}$, $\tilde{\alpha}]=\alpha_j$ - значение формулы на наборе.\\
    Пусть $\in S$, $f=\Sigma(s)$, Ф$_1$, $\ldots$, Ф$_n$ - формулы в сигнатуре. Обозначим Ф$_1[\tilde{x}$, $\tilde{\alpha}]=\beta_1$, $\ldots$, Ф$_n[\tilde{x}$, $\tilde{\alpha}]=\beta_n$, тогда $f(\beta_1$, $\ldots$, $\beta_n)$ - значение формулы на наборе $\tilde{\alpha}$.
  \end{definition}
  \begin{definition}
    Операции:
    \begin{enumerate}
      \item Операция подстановки переменных. Пусть $f(x_1$, $\ldots$, $x_n)\in P_k$, $g(x_1$, $\ldots$, $x_n)$ - функция, определённая на $B_n$ такая, что $g(x_1$, $\ldots$, $x_n)=f(x_{i_1}$, $\ldots$, $x_{i_n})$, где набор ($i_1$, $\ldots$, $i_n$) - набор элементов ($1$, $\ldots$, $n$) (они необязательно различны). Тогда $g$ получена из $f$ операцией подстановки переменных.
      \item Операция подстановки функции в функцию. Пусть $f(x_1$, $\ldots$, $x_n)$, $g(x_1$, $\ldots$, $x_m)$, $h$ определена на $B_{n+m-1}$ и $h(x_1$, $\ldots$, $x_{n+m-1})=f(x_1$, $\ldots$, $x_{n-1}$, $g(x_n$, $\ldots$, $x_{n+m-1}))$, тогда функция $h$ получена из функций $f$ и $g$ операцией подстановки одной функции в другую.
      \item Операция добавления или удаления фиктивных переменных. Пусть $x_i$ - фиктивная переменная, тогда если функция $f(x_1$, $\ldots$, $x_{i-1}$, $x_{i+1}$, $\ldots$, $x_n)=g(x_1$, $\ldots$, $x_{i-1}$, $0$, $x_{i+1}$, $\ldots$, $x_n)$, то функция $g$ называется полученной из $f$ добавлением фиктивной переменной. Функция удаления фиктивной переменной определяется аналогично.
    \end{enumerate}
  \end{definition}
  \begin{center}
    Билет 16
  \end{center}
  Тождества для функций в $P_k$:
  \begin{enumerate}
    \item операции $min(x_1$, $x_2)$, $max(x_1$, $x_2$), $x_1\cdot x_2(mod$ $k)$, $x_1+x_2(mod$ $k)$ ассоциативны и коммутативны
    \item $min(max(x_1$, $x_2)$, $x_3)=max(min(x_1$, $x_3)$, $min(x_2$, $x_3))$
    \item ($x_1+x_2)\cdot x_3=(x_1\cdot x_3)+(x_2\cdot x_3)$
    \item $\sim(\sim x)=x$
    \item $\sim min(x_1$, $x_2)=max(\sim x_1$, $\sim x_2)$
  \end{enumerate}
  \begin{definition}
    Выражение $\bigvee\limits_{(\sigma_1,\ldots,\sigma_n)\in (E_k)^n}min(J_{\sigma_1}(x_1)$, $\ldots$, $J_{\sigma_n}(x_n)$, $f(\sigma_1$, $\ldots$, $\sigma_n))$ - аналог совершенной дизъюнктивной нормальной формы для $P_k$.
  \end{definition}
  \begin{theorem}
    Любая функция, не являющаяся тождественно нулевой, имеет аналог совершенной д.н.ф.
  \end{theorem}
  \begin{proof}
    Рассмотрим произвольный набор $(\alpha_1$, $\ldots$, $\alpha_n)$, так как $J_{\sigma_i}(\alpha_j)=0$ $\forall j\neq i$, а для $j=i$ $J_{\sigma_i}(\alpha_i)=k-1$, значит, все члены, кроме $\alpha_1=\sigma_1$, $\ldots$, $\alpha_n=\sigma_n$, будут равны нулю, а значит, останется тольго $min(J_{\sigma_1}(\alpha_1)$, $\ldots$, $J_{\sigma_n}(\alpha_n)$, $f(\alpha_1$, $\ldots$, $\alpha_n))=f(\alpha_1$, $\ldots$, $\alpha_n)$.
  \end{proof}
  \begin{center}
    Билет 17
  \end{center}
  \begin{definition}
    Система $F$ функций в $P_k$ называется полной, если любая функция из $P_k$ получается суперпозициями из $F$.
  \end{definition}
  \begin{example}
    \begin{enumerate}
      \item $P_k$
      \item $\{0$, $1$, $\ldots$, $k-1$, $J_0(x)$, $\ldots$, $J_{k-1}(x)$, $min(x_1$, $x_2$), $max(x_1$, $x_2)\}$
      \item $max(x_1$, $x_2)$, $\overline{x}$
      \item $min(x_1$, $x_2)$, $\overline{x}$
      \item $\{0$, $1$, $\ldots$, $k-1$, $j_0(x)$, $\ldots$, $j_{k-1}(x)$, $x_1+x_2$, $x_1\cdot x_2\}$
      \item $V_k(x_1$, $x_2)=max(x_1$, $x_2)+1(mod$ $k$)
    \end{enumerate}
  \end{example}
  Докажем полноту каждой из систем.
  \begin{proof}
    \begin{enumerate}
      \item Так как в системе есть отрицание Поста, то из $\forall x$ можно получить $\{x$, $x+1$, $\ldots$, $x+k-1$\} все эти числа различны по $(mod$ $k$) $\Longrightarrow$ $max(x$, $\ldots$, $x+k-1)=k-1$, тогда из константы $k-1$ можно получить все остальные константы, используя отрицание Поста.\\
      Рассмотрим набор $\{x$, $\ldots$, $x_{j-1}$, $x_{j+1}$, $\ldots$, $x_k$\}, тогда функция $\phi_j(x)=max(x$, $\ldots$, $x+j-1$, $x+j+1$, $\ldots$, $x+k-1)=\begin{matrix}
        k-1, \textup{при} x+j\neq k-1\\
        k-2, \textup{при} x+j=k-1
      \end{matrix}$. Тогда функция $\psi_j(x)=max(x$, $\ldots$, $x+j-1$, $x+j+1$, $\ldots$, $x+k-1)+1$ (это ожно сделать благодаря отрицанию Поста) $\Longrightarrow$ $\psi_j(x)=\begin{matrix}
        0, \textup{при} x+j\neq k-1\\
        k-1, \textup{при} x+j=k-1
      \end{matrix}$. То есть мы получили все константы, $J_i(x)$ $\forall i$, а значит, получили полную систему из примера 2.
      \item Аналогично с предыдущим пунктом, с помощью отрицания Поста можно получить все константы, а значит, можем получить отрицание Лукашевича, а по одному из тождеств, $\sim min(x_1$, $x_2)=max(\sim x_1$, $\sim x_2)$, то есть мы получили поную систему из предыдущего пункта.
      \item Из $V_k(x_1$, $x_2$) получим отрицание Поста: $V_k(x$, $x)=x+1=\overline{x}$ $\Longrightarrow$ можно получить $x+i$ $\forall i$, тогда $max(x_1$, $x_2)=V_k(x_1$, $x_2)+k-1$, то есть мы получили полную систему $\{max(x_1$, $x_2)$, $\overline{x}\}$.
    \end{enumerate}
  \end{proof}
  \begin{center}
    Билет 18
  \end{center}
  \begin{definition}
    Замыканием множества $F$ в $P_k$ называется множество всех функций, которые можно получить суперпозициями из $F$.
  \end{definition}
  \begin{definition}
    Если $[F]=F$, то множество $M$ называется замкнутым.
  \end{definition}
  \begin{definition}
    Пусть $Q\subseteq E_k$. Множество функций $T_Q$ : $\forall \alpha_1$, $\ldots$, $\alpha_n\in Q$ $f(\alpha_1$, $\ldots$, $\alpha_n)\in Q$, называется функцией, сохраняющей множество $Q$.
  \end{definition}
  \begin{example}
    \begin{enumerate}
      \item $P_k$
      \item $T_Q$
    \end{enumerate}
  \end{example}
  \begin{theorem}
    Класс $T_Q$ замкнут.
  \end{theorem}
  \begin{proof}
    \begin{enumerate}
      \item Операция подстановки переменных:\\
      Пусть функция $f(x_1$, $\ldots$, $x_n)$ сохраняет множество $Q$, тогда $g(x_1$, $\ldots$, $x_n)=f(x_{i_1}$, $\ldots$, $x_{i_n})$ тоже будет сохранять множество $Q$, так как при перестановке одинаковых переменных ничего не поменяется.
      \item Операция подстановки функции в функцию:\\
      Пксть функции $f(x_1$, $\ldots$, $x_n)$ и $g(x_1$, $\ldots$, $x_m)$ сохраняют множесво $Q$, тогда $h(x_1$, $\ldots$, $x_{m+n-1})=f(x_1$, $\ldots$, $x_{n-1}$, $g(x_n$, $\ldots$, $x_{m+n-1}))$, так как функция $g$ сохраняет множество $Q$ $\Longrightarrow$ все переменные $f$ принимают одно и то же значение, а значит, и функция $h$ будет сохранять множество $Q$.
      \item Операция добавления или удаления фиктивных переменных:\\
      Очевидно.
    \end{enumerate}
  \end{proof}
  \begin{center}
    Билет 19
  \end{center}
  \begin{definition}
    Определим глубину формулы через индукцию по определению формулы в сигнатуре:\\
    База индукции: пусть $x_i$ - символ переменной, тогда глубина формулы $x_i$ равна 0.\\
    Пусть $s\in S$, $f=\Sigma(s)$, Ф$_1$, $\ldots$, Ф$_n$ - формулы сигнатуре, причём $m$ - наибольшая из глубин из этих формул, тогда глубина формулы $s($Ф$_1$, $\ldots$, Ф$_n)$ равна $m+1$.
  \end{definition}
  \begin{theorem}
    Существует алгаритм, распознающий поноту конечных систем функций в $P_k$. Он заключается в построении последовательности Кузнецова и проверке вхождения в её предел фунции Вебба.
  \end{theorem}
  \begin{proof}
    Пусть $F\subseteq P_k$ - конечное множество функций в $P_k$, $\Sigma:S\to F$ - сигнатура. Рассмотрим последовательность $G_1$, $G_2$, $\ldots$ такую, что $G_i$ - множество функций, определяемых невырожденными формулами в сигнатуре $\Sigma$, содержащими только переменные $x_1$, $x_2$ и имеющими глубину, меньшую $i$. Данную последовательность назовём последовательностью Кузнецова. Так как все формулы в соответствующем множестве $G_i$ имеют глубину, меньшую $i$ $\Longrightarrow$ $\varnothing\subseteq G_1\subseteq\ldots$. Так как число функций в $P_k$ от двух переменных равно $k^{k^2}$ $\Longrightarrow$ $|G_i|\leqslant k^{k^2}$ $\Longrightarrow$ последовательность Кузнецова стабилизируется на некотором шаге $G_m=G$, $G$ называется пределом последовательности Кузнецова. Свяжем с каждой функцией из $G_i$ некоторую формулу Ф'$_j$, содержащую тольо переменные $x_1$, $x_2$ и имеющая глубину, меньшую $i$. Рассмотрим функцию $f\in G_{i+1}\setminus G_i$, она определяется формулой Ф$=s($Ф$_1$, $\ldots$, Ф$_n$), где формулы Ф$_1$, $\ldots$, Ф$_n$ либо являются переменными, либо определяют некоторые функции в $G_i$, но эти функции мы уже определили формулами Ф'$_j$, тогда елси заменить в формуле Ф формулы Ф$_j$ на Ф'$_j$, то мы получим формулу Ф', определяющую ту же самую функцию $f$ $\Longrightarrow$ для получения из $G_i$ $G_{i+1}$ достаточно рассмотреть все формулы Ф'$=s($Ф'$_1$, $\ldots$, Ф'$_n)$. Значит данную последовательность имеет смысл проверять до первого совпадения $G_i$ и $G_{i+1}$.\\
    \begin{lemma}
      Система фнкций в $P_k$ полна тогда и только тогда, когда в предел последовательности входит функция Вебба.
    \end{lemma}
    \begin{proof}
      $\underline{\Longrightarrow}$ Пусть $V_k(x_1$, $x_2)\in G$, тогда функция Вебба получается суперпозициями из функций данной системы $\Longrightarrow$ эта система полна.\\
      $\underline{\Longleftarrow}$ Пусть система функций $F$ полна, тогда функция Вебба определяется некоторой формулой в сигнатуре $\Sigma$, существенно зависящей от двух переменных и имеющей глубину, меньшую $i$, то есть $V_k\in G_i$, переобозначим переменные так, чтобы существенными стали только переменные $x_1$, $x_2$, а все остальные несущественные переменные заменим на $x_1$, тогда эта формула определяет функцию из $G_{i+1}$ (так как она получена из формул, сопоставленных функциям из $G_i$) $\Longrightarrow$ $V_k\in G_{i+1}$ $\Longrightarrow$ $V_k\in G$.
    \end{proof}
  \end{proof}
  \begin{center}
    Билет 20
  \end{center}
  \begin{theorem}
    Из любой полной системы функций в $P_k$ можно выделить конечную полную подсистему.
  \end{theorem}
  \begin{proof}
    Пусть $F$ - полная система в $P_k$, тогда суперпозициями из $F$ можно получить функцию Вебба, то есть полную подсистему, а так как она получается суперпозициями из конечного числа функций, значит, подсистема из этих функций конечна и полна.
  \end{proof}
  \begin{center}
    Билет 21
  \end{center}
  \begin{definition}
    Функции $g_i^p(x_1$, $\ldots$, $x_p)=x_i$, где $i=\overline{1,p}$,  называются селекторными функциями.
  \end{definition}
  \begin{definition}
    Пусть $K$ - множество функций $h(x_1$, $\ldots$, $x_p)$, зависящих от $p$ переменных и содержащих все селекторные функции от $p$ переменных. Если для любых функций $h_1(x_1$, $\ldots$, $x_p)$, $\ldots$, $h_n(x_1$, $\ldots$, $x_p)$ функция $f(h_1$, $\ldots$, $h_n)\in K$, то скажем, что функция $f$ сохраняет множество $K$.
  \end{definition}
  Рассмотрим класс функций в алгебре логики, сохраняющих множество $K=\{x$, $\overline{x}\}$, то есть в $K$ входят функции $\{x^{\sigma}\}$, где $\sigma=\{0, 1\}$. Тогда функция $f$ сохраняет $K$, если $f(x_1^{\sigma_1}$, $\ldots$, $x_n^{\sigma_n})=x^{\sigma}$, то есть
  $$\begin{cases}
    f(1^{\sigma_1}, \ldots, 1^{\sigma_n})=1^{\sigma}=\sigma=f(\sigma_1, \ldots, \sigma_n)\\
    f(0^{\sigma_1}, \ldots, 0^{\sigma_n})=0^{\sigma}=\overline{\sigma}=f(\overline{\sigma}_1, \ldots, \overline{\sigma}_n)
  \end{cases}$$\\
  $\Longrightarrow$ $f(\sigma_1$, $\ldots$, $\sigma_n)=\overline{f(\overline{\sigma}_1 \ldots \overline{\sigma}_n)}$, то есть мы получили класс $S$ самодвойственных функций.
  \begin{definition}
    Множество всех функций, сохраняющих множество $K$, называется классом сохранения множества $K$. Данный класс обозначим $U(K)$.
  \end{definition}
  \begin{theorem}
    Класс $U(K)$ замкнут.
  \end{theorem}
  \begin{proof}
    \begin{enumerate}
      \item Опреация подстановки переменных:\\
      Пусть функция $f$ сохраняет множество $K$, тогда функция $g(x_1$, $\ldots$, $x_n)=f(x_{i_1}$, $\ldots$, $x_{i_n})$, $f(h_1(x_1$, $\ldots$, $x_p)$, $\ldots$, $h_n(x_1$, $\ldots$, $x_p))\in K$ $\forall h_1$, $\ldots$, $h_n\in K$, а значит, $f(h_{i_1}(x_1$, $\ldots$, $x_p)$, $\ldots$, $h_{i_n}(x_1$, $\ldots$, $x_p))\in K$ $\Longrightarrow$ $g$ сохраняет множество $K$.
      \item Операция подстановки функции в функцию:\\
      Аналогично с предыдущим пунктом.
      \item Операция добавления или удаления фиктивных переменных:\\
      Пусть $f(x_1$, $\ldots$, $x_{i-1}$, $x_{i+1}$, $\ldots$, $x_n)$ сохраняет множество $K$,  $g(x_1$, $\ldots$, $x_{i-1}$, $0$, $x_{i+1}$, $\ldots$, $x_n)$ получена из $f$ добвлением фиктивной переменной, тогда $g$ будет сохранять множество $K$, так как при подстановке функций $h_j(x_1$, $\ldots$, $x_p)$ в функцию $g$ мы получим $g(h_1(x_1$, $\ldots$, $x_p)$, $\ldots$, $h_{i-1}(x_1$, $\ldots$, $x_p)$, $0$, $h_{i+1}(x_1$, $\ldots$, $x_p)$, $\ldots$, $h_n(x_1$, $\ldots$, $x_p))=f(h_1(x_1$, $\ldots$, $x_p)$, $\ldots$, $h_{i-1}(x_1$, $\ldots$, $x_p)$, $h_{i+1}(x_1$, $\ldots$, $x_p)$, $\ldots$, $h_n(x_1$, $\ldots$, $x_p))\in K$.
    \end{enumerate}
    Значит, суперпозициями мы не сможем получить функцию, не сохраняющая множество $K$.
  \end{proof}
\end{document}