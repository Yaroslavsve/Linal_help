\documentclass[a4paper, 12pt]{article}

\usepackage{import}

% Корректность отображения всех шрифтов, кодировок и мат. символов
\usepackage[T2A]{fontenc}
\usepackage[utf8]{inputenc}
\usepackage[english, russian]{babel}
\usepackage{amssymb, amsmath, amsthm, mathtools}

% Отображение содержания
\usepackage{tocloft}

% Вставка картинок
\usepackage{graphicx}
\usepackage{tikz}
\usepackage{tkz-euclide}
\usepackage{asymptote}

\usepackage{wrapfig}        % Огибание картинок текстом
\usepackage{cancel}         % Зачёркивания
\usepackage{indentfirst}    % Отступ у первого абзаца
\usepackage{xcolor}         % Цвета
\setlength{\parskip}{.5ex}  % Отступы между абзацами
\usepackage{enumitem}       % Работа со списками
% \usepackage{minted}       % Вставка блоков кода

\usepackage{hyperref}       % гиперссылки
\definecolor{linkcolor}{HTML}{225ae2} % Цвет ссылок
\definecolor{urlcolor}{HTML}{225ae2} % Цвет гиперссылок
\hypersetup{
    pdfstartview=FitH, 
    linkcolor=linkcolor,
    urlcolor=urlcolor,
    colorlinks=true}
\setlength{\arrayrulewidth}{0.5mm} %Толщина линейки в таблицах
\setlength{\tabcolsep}{18pt} %Разделение между столбцами в таблице

% Отступы на странице
\usepackage[left=2cm, right=1.5cm, top=2cm, bottom=2cm]{geometry}

\usepackage{cmap}            % Русский поиск в PDF документе
\usepackage{etoolbox}
\usepackage{soul}            % Разряженный текст \so{} и подчеркивание \ul{}
\usepackage{soulutf8}        % Поддержка UTF8 в soul

\usepackage{titlesec}        % Форматирование заголовков
\titleformat{\section}{\LARGE \bfseries}{\thesection}{1em}{}
\titleformat{\subsection}{\Large\bfseries}{\thesubsection}{1em}{}
\titleformat{\subsubsection}{\large\bfseries}{\thesubsubsection}{1em}{}

\newcommand{\R}{\mathbb R}
\newcommand{\Q}{\mathbb Q}
\newcommand{\Z}{\mathbb Z}
\newcommand{\N}{\mathbb N}
\newcommand{\CC}{\mathbb C}
\newcommand{\F}{\mathbb F}
\newcommand{\aug}{\fboxsep=-\fboxrule\!\!\!\fbox{\strut}\!\!\!}
\newcommand{\sgn}{\operatorname{sgn}}
\newcommand{\id}{\mathrm{id}}
\renewcommand{\phi}{\varphi}
\renewcommand{\epsilon}{\varepsilon}

\newsavebox{\boxedalignbox}
\newenvironment{boxedalign*}
  {\begin{equation*}\begin{lrbox}{\boxedalignbox}$\begin{aligned}}
  {\end{aligned}$\end{lrbox}\fbox{\usebox{\boxedalignbox}}\end{equation*}}

\newcommand\tab[1][.5cm]{\hspace*{#1}}

% Подписи для матриц
\newcommand\undermat[2]{\makebox[0pt][l]{$\smash{\underbrace
{\phantom{\begin{matrix}#2\end{matrix}}}_{\text{$#1$}}}$}#2}
\newcommand\overmat[2]{\makebox[0pt][l]{$\smash{\overbrace
{\phantom{\begin{matrix}#2\end{matrix}}}^{\text{$#1$}}}$}#2}

% Значек "пусть"
\newlength{\tempheight}  
\newcommand{\Let}[0]{  
\mathbin{\text{\settoheight{\tempheight}{\mathstrut}\raisebox{0.5\pgflinewidth}{%
\tikz[baseline,line cap=round,line join=round] \draw (0,0) --++ (0.4em,0) --++ (0,1.5ex) --++ (-0.4em,0);
}}}}


% \newcounter{lemcount}
% \newcounter{thcount}
% \newcounter{offercount}
% \newcounter{concount}
% \newcounter{subthcount}
% \newcounter{defcount}

\theoremstyle{definition}
\newtheorem*{definition}{Определение}
% \newtheorem{definitionnum}[defcount]{Определение}
\newtheorem*{example}{Примеры}
\newtheorem*{example1}{Пример}
\newtheorem*{exercise}{Упражнение}


\theoremstyle{plain}
\newtheorem*{theorem}{Теорема}
% \newtheorem{theoremnum}[thcount]{Теорема}
\newtheorem*{consequense}{Следствие}
\newtheorem*{consequenses}{Следствия}
% \newtheorem{consequensenum}[concount]{Следствие}
\newtheorem*{lemma}{Лемма}
% \newtheorem{lemmanum}[lemcount]{Лемма}
\newtheorem*{subtheorem}{Утверждение}
% \newtheorem{subtheoremnum}[subthcount]{Утверждение}
\newtheorem*{algorithm}{Алгоритм}
\newtheorem*{properties}{Свойства}
\newtheorem*{properties1}{Свойство}


\theoremstyle{remark}
\newtheorem*{remark}{Замечание}
\newtheorem*{offer}{Предложение}
% \newtheorem{offernum}[offercount]{Предложение}
\begin{document}
  \newpage
  \begin{exercise} $A_\phi$ - матрица линейного отображения $\phi: V_1 \to V_2$, dim($V_1$) = n, dim($V_2$) = m. Какой должна быть матрица $A_\phi$, чтобы 
    а) отображение $\phi$ было инъективным
    б) отображение $\phi$ было сюръективным
    в) отображение $\phi$ было биекцией
  \end{exercise}
  \begin{theorem}
    Пусть векторы $a_1$, ..., $a_n$ - л. н. з. векторы в $V_1$, $dimV_1$ = n, $b_1$, ..., $b_n$ - случайные векторы в $V_2$. Тогда $\exists$! \отображение $\phi: V_1 \to V_2$ : $\phi(a_j)$ = $b_j$
  \end{theorem}
  \begin{proof}
    Пусть в базисе $E$ пространства $V_1$ $a_j$ = \begin{pmatrix}
      $a_1$
      $a_n$
    \end{pmatrix}
  \end{proof}
  \begin{theorem}
    Если $dimV_1$ = n и $\phi: V_1 \to V_2$ - линейное отображение, то $\phi(V_1)\cong$ фактор-пространству.
  \end{theorem}
  \begin{proof}

  \end{proof}
  \begin{definition}
    Линейное отображение $\phi: V \to V$ называется линейным оператором.
  \end{definition}
  Если $\phi: V \to V$ - линейной оператор, то Ker$\phi$ и Im$\phi$ являются подпространствами V.
  $\Longrightarrow$ Если $U$ - подпространство $V$ и $\phi: V \to V$ - линейный оператор, то Im$\phi$ - подпространство $V$.
  \begin{definition}
    Пусть $U$ - подпространство $V$, отображение $\phi: V \to V$ - линейный оператор, тогда если $\forall u \in U \phi(u) \in U$ 
  \begin{definition}
    Матрицей линейного оператора $\phi$ в базисе {$e_1$...$e_n$} называется матрица $A_{\phi}$ = ($a_{ij}$) такая, что ($\phi(e_1$), ..., $\phi(e_n$)) = ($e_1$, ..., $e_n$)$A_{\phi}$
  \end{definition}
  \begin{subtheorem}
    Пусть $C$ - матрица перехода от базиса {$e_1$...$e_n$} к базису {$e'_1$...$e'_n$}, тогда матрица $A_{\phi, e'}$ линейного оператора $\phi$ в базисе {$e'_1$...$e'_n$} выражается через матрицу $A_{\phi, e}$ линейного оператора $\phi$ в базисе {$e_1$...$e_n$} и матрицу перехода.
  \end{subtheorem}
  \begin{proof}
    ($e'_1$, ..., $e'_n$) = ($e_1$, ..., $e_n$)$C$
    $\Longrightarrow$ в силу линейности оператора $\phi$ ($\phi(e'_1$), ..., ($\phi(e'_n$))) = ($\phi(e_1$), ..., ($\phi(e_n$)))$C$ = ($e_1$...$e_n$)$AC$ = ($e'_1$...$e'_n$)$C^{-1}A_{\phi, e}C$
    $\Longrightarrow$ $A_{\phi, e'}$ = $C^{-1}A_{\phi, e}C$.
  \end{proof}
\end{document}
